% Обязательный слайд: четкая формулировка цели данной работы и постановка задачи
% Описание выносимых на защиту результатов, процесса или особенностей их достижения и т.д.
\begin{frame}
	\transwipe[direction=90]
	\frametitle{\faCheckSquareO\ Результаты}
	\begin{itemize}
		\item[\checkmark] Модернизирована архитектура набора данных
		\item[\checkmark] Добавлены новые возможности работы с данными
		\begin{itemize}
            \item[\checkmark] Загрузка конкретных графов из набора данных
            \item[\checkmark] Преобразование графов в другие форматы
            \item[\checkmark] Получение информации о графе
            \item[\checkmark] Трансформация графов
        \end{itemize}
		\item[\checkmark] Опубликован Python пакет\footnote{Python пакет: \url{https://pypi.org/project/cfpq-data}, дата последнего доступа --- 04.06.2021} для работы с набором данных и документация к нему\footnote{Веб-сайт с документацией: \url{https://jetbrains-research.github.io/CFPQ_Data/}, дата последнего доступа --- 04.06.2021}
	\end{itemize}
\end{frame}