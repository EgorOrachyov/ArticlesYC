\begin{frame}[fragile]
	\transwipe[direction=90]
	\frametitle{\faConnectdevelop\ Задача поиска путей с контекстно-свободными ограничениями}
	\begin{columns}
		\begin{column}{0.5\textwidth}
			\begin{figure}[h]{\textwidth}
				\centering
				\begin{tikzpicture}[shorten >=1pt,node distance=2cm,on grid,auto]
					\node[state] (q_1)   {$1$};
					\node[state] (q_2) [above=of q_1] {$2$};
					\node[state] (q_3) [above right=of q_1, below right=of q_2] {$0$};
					\node[state] (q_4) [right=of q_3] {$3$};
					\path[->]
					(q_1) edge[red]  node[black] {a} (q_2)
					(q_2) edge[red]  node[black] {a} (q_3)
					(q_3) edge  node {a} (q_1)
					(q_3) edge[bend left, above, red]  node[black] {b} (q_4)
					(q_4) edge[bend left, below, red]  node[black] {b} (q_3);
				\end{tikzpicture}
				\caption*{Рис. 1. Граф} \label{fig:Graph}
			\end{figure}
							    
			\begin{figure}[h]
				\centering
				\begin{subfigure}[h]{\textwidth}
					\centering
					\[
						\begin{array}{l}
							S \rightarrow a \ S \ b \\
							S \rightarrow a \ b     
						\end{array}
					\]
				\end{subfigure}
				\caption*{Рис. 2. Грамматика} \label{fig:Grammar}
			\end{figure}
					
		\end{column}
			
		\begin{column}{0.5\textwidth}
			\textbf{CFPQ алгоритмы}
						    
			\begin{itemize}
				\item[\bullet] \textbf{Граф}: на рёбрах метки
				\item[\bullet] \textbf{Запрос}: контекстно-свободная грамматика
				\item[\bullet] \textbf{Путь в графе}: слово из языка этой грамматики
				\item[\bullet] \textbf{Задача}: искать такие пути
			\end{itemize}
						    
			\newline
						    
			\textbf{Области применения}
							
			\begin{itemize}
				\item[\bullet] Базы данных
				\item[\bullet] Биоинформатика
				\item[\bullet] Статический анализ кода
			\end{itemize}
					
		\end{column}
	\end{columns}
\end{frame}
