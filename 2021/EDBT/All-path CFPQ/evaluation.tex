\section{Evaluation}
The goal of this evaluation is to investigate the applicability of the proposed matrix-based algorithm to CFPQ with all-path query semantics and to provide the comparation of the most performant linear algebra-based CFPQ algorithms. We will compare the following CFPQ implementations:
\begin{itemize}
	\item $MtxSingle$ --- the implementation from~\cite{10.1145/3398682.3399163} of the matrix-based CFPQ algorithm for single-path query semantics,
	\item $Tns$ --- the implementation from~\cite{} of the Kronecker product-based CFPQ algorithm for all-path query semantics which all include the single-path query semantics,
	\item $MtxAll$ --- the implementation of the proposed matrix-based CFPQ algorithm for all-path query semantics which utilizes SuiteSparse\footnote{SuiteSparse is a sparse matrix software which includes GraphBLAS API implementation. Project web page: \url{http://faculty.cse.tamu.edu/davis/suitesparse.html}. Access date: 14.01.2021.}~\cite{Davis2018Algorithm9S} implementation of GraphBLAS API for matrix manipulations.
\end{itemize}

First, we measured the execution time and required memory of the index creation. Then we compared the practical applicability of the paths extraction for both implementations $MtxAll$ and $Tns$ of the CFPQ with all-path query semantics.

For evaluation, we used a PC with Ubuntu 18.04 installed.
It has Intel core i7-6700 CPU, 3.4GHz, and DDR4 64Gb RAM.
We only measure the execution time of the algorithms themselves, thus we assume an input graph is loaded into RAM in the form of its adjacency matrix in the sparse format.

\subsection{Dataset Description}

We use the graphs and respective queries from the CFPQ\_Data dataset\footnote{CFPQ\_Data dataset GitHub repository: \url{https://github.com/JetBrains-Research/CFPQ_Data}. Access date: 14.01.2021.} provided in~\cite{10.1145/3398682.3399163}. This dataset contains the real-world RDFs with properties presented in table~\ref{tbl:propRDF}, and queries $g_1, g_2, geo$ 
which are different variations of the \textit{same-generation query}~\cite{FndDB} --- an important example of real-world queries that are context-free but not regular.




{\setlength{\tabcolsep}{0.25em}
	\begin{table}
		{
			\caption{RDFs properties}
			\label{tbl:propRDF}
			\small
			\rowcolors{2}{black!2}{black!10}
			\begin{tabular}{|l|c|c|c|}
				\hline
				Graph & \#V & \#E & Queries \\
				\hline
				\hline
				pathways               & 6238                 & 18 598               & $g_1, g_2$ \\
				gohierarchy            & 45 007               & 980 218              & $g_1, g_2$ \\
				enzyme                 & 48 815               & 109 695              & $g_1, g_2$ \\
				eclass\_514en          & 239 111              & 523 727              & $g_1, g_2$ \\
				go                     & 272 770              & 534 311              & $g_1, g_2$ \\
				geospecies             & 450 609              & 2 311 461            & $g_1, g_2, geo$  \\
				taxonomy                   & 5 728 398                 & 14 922 125                 & $g_1, g_2$ \\
				\hline
			\end{tabular}
		}
	\end{table}
}


\subsection{Evaluation Results}
The results of the index creation for all three implementations are presented in table~\ref{tbl:index_creation}. We can see that the running time of both CPU
and GPGPU versions for the relational query semantics is
small even for graphs with a big number of vertices and
edges. The relatively small number of edges of interest may
be the reason for such behavior. We believe it is necessary
to extend the dataset with new queries that involve more
different types of edges. Also, we can see, that $RG_CUSPrel$
implementation which uses CUSP requires more memory.
As we can see, the matrix-based algorithm for relational
query semantics implemented for RedisGraph is more than
1000 times faster than the one based on annotated grammar
implemented for Neo4j [17] and uses more than 4 times less
memory. We can conclude that the matrix-based algorithm
is more performant than other CFPQ algorithms for query
evaluation under a relational semantics for real-world data
processing.
Also, we can see, that the GPGPU version which utilizes
sparse matrices is significantly faster than the other implementations especially on big graphs. For example, for
Geospacies it more than 7 times faster in both relational and
single-path scenarios. Note, that for GPGPU versions we
include the time required for data transferring and format
conversions.
We can conclude, that the cost of computing matrices
with PathIndexes for single-path query semantics is not high.
On average, it is about 2 times slower than the reachability
matrix calculation. The additional running time of the path
extraction is presented in figure 7 (boxplots are standard,
outliers are omitted). As we can see, this time is small and
linear in the length of the path.
Finally, we conclude that the matrix-based algorithm paired
with a suitable database and employing appropriate libraries
for linear algebra is a promising way to make CFPQ with
relational and single-path query semantics applicable for
real-world data analysis. We show that the SuiteSparse-based
CPU implementation is performant enough to be comparable
with GPGPU-based implementations on real-world data.

{\setlength{\tabcolsep}{0.25em}
	\begin{table*}[t]
		{
			\caption{Index creation}
			\label{tbl:index_creation}
			\small
			\rowcolors{4}{black!2}{black!10}
			\begin{tabular}{|l|l|l|l|l|l|l|l|l|l|l|l|l|l|l|l|l|l|l|}
				\hline
				\multicolumn{1}{|c|}{\multirow{3}{*}{Graph}} & \multicolumn{6}{c|}{G1}                                                           & \multicolumn{6}{c|}{G2}                                                           & \multicolumn{6}{c|}{Geo}                                                          \\ \cline{2-19}
				\multicolumn{1}{|c|}{}                       & \multicolumn{2}{c|}{MtxAll} & \multicolumn{2}{c|}{Tns} & \multicolumn{2}{c|}{MtxSingle} & \multicolumn{2}{c|}{MtxAll} & \multicolumn{2}{c|}{Tns} & \multicolumn{2}{c|}{MtxSingle} & \multicolumn{2}{c|}{MtxAll} & \multicolumn{2}{c|}{Tns} & \multicolumn{2}{c|}{MtxSingle} \\ \cline{2-19}
				\multicolumn{1}{|c|}{}                       & Time          & Mem         & Time        & Mem        & Time        & Mem        & Time          & Mem         & Time        & Mem        & Time        & Mem        & Time          & Mem         & Time        & Mem        & Time        & Mem        \\ \hline \hline
				\hline
				pathways                                     & 0.04          & 91          & 0.02        & 123        & 0.01        & 671        & 0.01          & 49          & 0.01        & 122        & 0.01        & 671        & ---           & ---         & ---         & ---        & ---         & ---        \\ \hline
				go-hierarchy                                 & 22.12         & 38797       & 0.17        & 265        & 1.41        & 660        & 15.66         & 28447       & 0.24        & 252        & 0.84        & 671        & ---           & ---         & ---         & ---        & ---         & ---        \\ \hline
				enzyme                                       & 0.4           & 307         & 0.04        & 137        & 0.01        & 216        & 0.02          & 61          & 0.02        & 132        & 0.01        & 217        & ---           & ---         & ---         & ---        & ---         & ---        \\ \hline
				eclass\_514en                                & 25.02         & 14416       & 0.24        & 205        & 0.23        & 216        & 0.22          & 126         & 0.27        & 193        & 0.16        & 216        & ---           & ---         & ---         & ---        & ---         & ---        \\ \hline
				go                                           & 11.8          & 8290        & 1.58        & 282        & 1.45        & 215        & 1.13          & 990         & 1.27        & 243        & 0.93        & 217        & ---           & ---         & ---         & ---        & ---         & ---        \\ \hline
				geospecies                                   & 4.45          & 2691        & 0.08        & 218        & 0.06        & 2250       & 0.34          & 156         & 0.01        & 196        & 0.01        & 2251       & 32.06         & 44235       & 26.32       & 19537      & 15.54       & 22941      \\ \hline
				taxonomy                                     & ---           & ---         & 4.42        & 2018       & 2.73        & 1962       & 19.13         & 27232       & 3.56        & 1776       & 1.15        & 2250       & ---           & ---         & ---         & ---        & ---         & ---        \\ \hline
			\end{tabular}
		}
	\end{table*}
}
