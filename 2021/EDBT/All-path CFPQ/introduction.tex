\section{Introduction}

Formal language-constrained path querying~\cite{doi:10.1137/S0097539798337716} is a graph analysis problem in which formal languages are used as constraints for
navigational path queries. In this problem a path in an edge-labeled graph is viewed
as a word constructed by concatenation of edge labels. The formal languages are used to constrain the paths of interest: a query should find only paths labeled by words from the language. The most popular class of constraints used as navigational queries in graph databases are the regular ones.
However, in some cases, regular languages are not expressive enough and context-free languages are used instead. The context-free path querying (CFPQ), can be used in many areas, for example, RDF analysis~\cite{10.1007/978-3-319-46523-4_38}, static code analysis~\cite{Zheng,10.1145/373243.360208}, biological data analysis~\cite{SubgraphQueriesbyContextfreeGrammars}.

CFPQ have been studied a lot since the problem was first stated by Mihalis Yannakakis in 1990~\cite{Yannakakis}.
Jelle Hellings investigates various aspects of CFPQ in~\cite{hellingsPathQuerying,hellingsRelational,DBLP:journals/corr/Hellings15} and formulate three possible querying semantics: relational that requires to find all vertex pairs reachable by some path of interest, single-path query semantics also requires to return the example of such path for all vertex pairs, and all-path query semantics that requires to return all such paths for all vertex pairs.

A number of CFPQ algorithms based on parsing techniques were proposed: (G)LL and (G)LR-based algorithms by Ciro M. Medeiros et al.~\cite{Medeiros:2018:EEC:3167132.3167265}, Fred C. Santos et al.~\cite{10.1007/978-3-319-91662-0_17}, Semyon Grigorev et al.~\cite{Grigorev:2017:CPQ:3166094.3166104}, and Ekaterina Verbitskaia et al.~\cite{10.1007/978-3-319-41579-6_22}; CYK-based algorithm by Xiaowang Zhang et al.~\cite{10.1007/978-3-319-46523-4_38}; combinators-based approach to CFPQ by Ekaterina Verbitskaia et al.~\cite{Verbitskaia:2018:PCC:3241653.3241655}.
Yet recent research by Jochem Kuijpers et al.~\cite{Kuijpers:2019:ESC:3335783.3335791} shows that existing solutions are not applicable for real-world graph analysis because of significant
running time and memory consumption.

One promising way to achieve high-performance solutions for graph querying problems is to reduce them to linear algebra operations. Rustam Azimov and Semyon Grigorev proposed a matrix-based algorithm for CFPQ with the relational query semantics in~\cite{Azimov:2018:CPQ:3210259.3210264} and with the single-path query semantics in~\cite{10.1145/3398682.3399163}.
This algorithms provides a solution performant enough for real-world data analyses.
However, in some cases, this is important to provide all founded paths and recently this matrix-based algorithm does not support the all-path query semantics. Another linear algebra-based CFPQ algorithm was proposed by Orachev et al. in~\cite{}. This Kronecker product-based CFPQ algorithms creates more expressive and complex index that can be used for answering all three query semantics. Thus, the matrix-based CFPQ algorithm cannot be truly compared with the Kronecker product-based algorithm yet.

In this work we propose a matrix-based CFPQ algorithm for all-path query semantics and provide the comparation of the most performant linear algebra-based CFPQ algorithms.

To sum up, we make the following contributions in this paper.
\begin{enumerate}
	\item We modify the Azimov's matrix-based CFPQ algorithm and provide a matrix-based CFPQ algorithm for all-path query semantics.
	Our modification is still based on linear algebra, hence it still allows one to use high-performance libraries and utilize modern parallel hardware for CFPQ evaluation.
	\item We provide the comparation of the most performant linear algebra-based CFPQ algorithms. For this, we implement the proposed algorithm and compare it with other linear algebra-based implementations using real-world RDF dataset. We show that proposed algorithm is performant enough for real-world data analysis and comparable with other most performant CFPQ implementations.
\end{enumerate}