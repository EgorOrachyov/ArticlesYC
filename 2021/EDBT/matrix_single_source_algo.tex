\section{Matrix-based multiple-source CFPQ algorithm}
\label{sec:multiple-source-algo}

 In this section we introduce two versions of the multiple-source matrix-based CFPQ algorithm.
 This algorithm is a modification of Azimov's matrix-based algorithm for CFPQ and its core idea is to cut off those vertices which are not in the selected set of start vertices.


In order to simplify Azimov's algorithm modification and the final algorithm description, we assume the input graph to have only edge labels.
Note, that we always can convert the original graph into such form.
To do it, we should add loops into vertices in the following way: for the vertex $i$ we add an edge $i \xrightarrow{x} i$ iff $\lambda_V(i) = x$ and $x\neq \varnothing$.
This way we can switch to an edge-labeled graph with the same number of vertices while preserving the defined semantics of CFPQ.

\begin{figure}[h]
    \centering
    \begin{tikzpicture}[shorten >=1pt,auto]
       \node[state] (q_0)                        {$0$};
       \node[state] (q_1) [right=of q_0]         {$1$};
       \node[state] (q_2) [below left=of q_1]    {$2$};
       \node[state] (q_3) [below left=of q_2]    {$3$};
       \node[state] (q_4) [below right=of q_2]   {$4$};
       \node[state] (q_5) [below right=of q_1]   {$5$};
       \path[->]
        (q_0) edge[loop left]  node {$\{x, y\}$} (q_0)
        (q_0) edge  node {$\{a\}$} (q_1)
        (q_2) edge[loop left]  node {$\{x\}$} (q_2)
        (q_1) edge  node {$\{b\}$} (q_5)
        (q_1) edge  node {$\{a, b\}$} (q_2)
        (q_3) edge[above]  node {$\{c\}$} (q_2)
        (q_4) edge  node {$\{c\}$} (q_3)
        (q_2) edge[above]  node {$\{c\}$} (q_4)
        (q_4) edge[loop below]  node {$\{y\}$} (q_4)
        (q_5) edge[bend left, below]  node {$\{d\}$} (q_4)
        (q_4) edge[bend left, above]  node {$\{d\}$} (q_5);
    \end{tikzpicture}
    \caption{The example of $D_1'$: the modified input graph $D_1$}
    \label{fig:example_modified_input_graph}
\end{figure}

The adjacency matrix $M$ of the graph $D_1'$ is
{
    \renewcommand{\arraystretch}{0.7}
    \setlength\arraycolsep{2pt}
$$
    M =
    \begin{pmatrix}
    \{x, y\}     & \{a\}       & \varnothing & \varnothing & \varnothing & \varnothing \\
    \varnothing  & \varnothing & \{a, b\}    & \varnothing & \varnothing & \{b\}       \\
    \varnothing  & \varnothing & \{x\}       & \varnothing & \{c\}       & \varnothing \\
    \varnothing  & \varnothing & \{c\}       & \varnothing & \varnothing & \varnothing \\
    \varnothing  & \varnothing & \varnothing & \{c\}       & \{y\}       & \{d\}       \\
    \varnothing  & \varnothing & \varnothing & \varnothing & \{d\}       & \varnothing
    \end{pmatrix}.
$$
}
Note that this transformation is impractical for real-world graphs, thus we use it only for algorithm description.

The first version of multiple-source algorithm is the Azimov's algorithm equipped with vertices filtering.
Let $G = (N, \Sigma, P, S)$ be the input context-free grammar, $D = (V, E, \Sigma_V, \Sigma_E, \lambda_V, \lambda_E)$ be the input graph and $Src$ be the input set of start vertices.
The result of the algorithm is a Boolean matrix which represents relation $R_{S,D}^{Src}$.

\begin{algorithm}
\small
\begin{algorithmic}[1]
\caption{Multiple-source CFPQ algorithm}
\label{alg:algo1}
\Function{MultiSrcCFPQNaive}{\par
\hskip\algorithmicindent $D = (V, E, \Sigma_V, \Sigma_E, \lambda_V, \lambda_E)$, \par
\hskip\algorithmicindent $G=(N,\Sigma,P,S)$, \par
\hskip\algorithmicindent
$Src$}
    \State{$T \gets \{T^A \mid  A \in N, T^A[i,j] \gets \textit{false} \text{, for all $i,j$}\} $}

    \State{$TSrc \gets \{TSrc^A \mid  A \in N, TSrc^A[i,j] \gets \textit{false} \text{, for all $i,j$}\}$}

    \ForAll{$ v \in Src$} \Comment{Input matrix initialization}
        \State{$TSrc^S[v,v] \gets true$}
    \EndFor
    
    \State $MSrc \gets TSrc^S$

    \ForAll{$A \to x \in P$} \Comment{Simple rules initialization}
        \ForAll{$(v, to) \in E, \lambda_E(v,to) = x$}
            \State{$T^A[v,to] \gets true$}
        \EndFor
    \EndFor

    \While{$T\ or\ TSrc\ is\ changing$} \Comment{Algorithm's body}
        \ForAll{$A \to B C \in P$}
            \State{$M \gets TSrc^A*T^B$}
            \State{$T^A \gets T^A + M*T^C$}
            \State{$TSrc^B \gets TSrc^B + TSrc^A$}
            \State{$TSrc^C \gets TSrc^C + $ \Call{getDst}{$M$}}
        \EndFor
    \EndWhile
    \State \Return $MSrc * T^S$
\EndFunction



\Function{getDst}{$M$}
    \State{$A[i,j] \gets \textit{false}$}
    \ForAll{$(v,to) \in V^2 \mid M[v,to] = true$}
        \State{$A[to,to] \gets true$}
    \EndFor
    \State \Return A
\EndFunction
\end{algorithmic}
\end{algorithm}

In order to solve the single-source and multiple-source CFPQ problem, we modified the Azimov's algorithm.
Each time, when a grammar rule is applied (see line \textbf{8} of Algorithm~\ref{alg:algo0}: Boolean matrix multiplication $T^A = T^A + T^B \cdot T^C$ for each $A \rightarrow BC \in P$), only vertices of interest should be stored.
To do it, we added one more matrix multiplication: $T^A = T^A + (TSrc^A \cdot T^B) \cdot T^ C$, where $TSrc^A$---matrix of start vertices for the current iteration (lines \textbf{11-13} of the Algorithm~\ref{alg:algo1}).
In the end of every iteration of for loop, it is necessary to update the set of source vertices paths from which we need to calculate.
To do it, the we call the function \textsc{getDst} (see lines \textbf{17-21}), in line \textbf{15}.
Thus, the modified algorithm supports the frontier of the vertices of interest and updates it on each iteration.
As a result, it only computes the paths which starts from the small set of selected vertices.

In case when one has a sequence of similar queries to the single graph, it may be useful to cache results of the query evaluation and share them between queries.
This may help to avoid recalculation of the already computed results.
To introduce sharing of results between queries, we modify the previous version of the algorithm.
The modified version stores all the vertices the paths from which have already been calculated in cache \textit{index}, which is used to filter such vertices in line \textbf{11} of Algorithm~\ref{alg:algo2}.
Thus, the modified algorithm calculates paths from the particular vertex only once.
Note, that \textsc{CreateIndex} function should be called first, and the created index can be shared between multiple calls of \textsc{MultiSrcCFPQCache} after that.

%\begin{algorithm}
%\small
%\begin{algorithmic}[1]
%\caption{Multiple-source context-free path querying algorithm with caching}
%\label{alg:algo2}
%\Function{CreateIndex}{ \par
%\hskip\algorithmicindent $D = (V, E, \Sigma_V, \Sigma_E, \lambda_V, \lambda_E)$, \par
%\hskip\algorithmicindent $G=(N,\Sigma,P,S)$}
%    \State{$T \gets \{T^A \mid  A \in N, T^A[i,j] \gets \textit{false} \text{, for all $i,j$}\}$}
%
%    \State{$TSrc \gets \{TSrc^A \mid  A \in N, TSrc^A[i,j] \gets \textit{false} \text{, for all $i,j$}\}$}
%
%    \ForAll{$A \to x \in P$} \Comment{Simple rules initialization}
%        \ForAll{$(v, to) \in E, \lambda_E(v,to) = x$}
%            \State{$T^A[v,to] \gets \textit{true}$}
%        \EndFor
%    \EndFor
%
%    \State \Return $(T, TSrc)$
%\EndFunction
%\State
%\Function{MultiSrcCFPQCache}{ \par
%\hskip\algorithmicindent $D = (V, E, \Sigma_V, \Sigma_E, \lambda_V, \lambda_E)$, \par
%\hskip\algorithmicindent $G=(N,\Sigma,P,S)$, \par
%\hskip\algorithmicindent $Src$, \par
%\hskip\algorithmicindent $Index = (T, TSrc)$}
%
%    \State{$TNewSrc \gets \{TNewSrc^A \mid  A \in N, TNewSrc^A \gets \textit{false} \text{, for all $i,j$}\}\}$}
%
%    \ForAll{$v \in Src \mid TSrc[v,v] = false$}
%        \State{$TNewSrc^S[v,v] \gets true$}
%    \EndFor
%    
%    \State $MSrc \gets TNewSrc^S$
%
%    \While{\textit{T or TNewSrc is changing}}
%        \ForAll{$A \to B C \in P$}
%            \State{$M \gets TNewSrc^A*T^B$}
%            \State{$T^A \gets T^A + M*T^C$}
%
%            \State{$TNewSrc^B \gets TNewSrc^B + TNewSrc^A \setminus TSrc^B$}
%            \State{$TNewSrc^C \gets TNewSrc^C + $ \Call{getDst}{M}} $\setminus$ $TSrc^C$
%        \EndFor
%    \EndWhile
%    \State \Return $MSrc*T^S$
%    \Comment{We want to return only relevant data, not all cached results}
%\EndFunction
%\end{algorithmic}
%\end{algorithm}



\subsection{Implementation Notes}

All of the versions presented have been implemented$\footnote{GitHub repository with implemented algorithms: \url{https://github.com/JetBrains-Research/CFPQ_PyAlgo}, last accessed 28.08.2020}$ using GraphBLAS framework that allows one to represent graphs as matrices and to work with them in terms of linear algebra.
For~convenience, the code is written in Python using pygraphblas\footnote{GitHub repository of PyGraphBLAS library: \url{https://github.com/michelp/pygraphblas}}, which is a Python wrapper for GraphBLAS API and is based on SuiteSparse:GraphBLAS\footnote{GitHub repository of SuiteSparse:GraphBLAS library: \url{https://github.com/DrTimothyAldenDavis/SuiteSparse}}~\cite{10.1145/3322125} --- the complete implementation of the GraphBLAS standard.
This library is specialized to work with sparse matrices which appear in real graphs most often.
