\section{Matrix-based multiple-source CFPQ algorithm}
\label{sec:multiple-source-algo}

 In this section we introduce a multiple-source matrix-based CFPQ algorithm.
 This algorithm is a modification of Azimov's matrix-based algorithm for CFPQ and its core idea is to cut off those vertices which are not in the selected set of start vertices.


Let $G = (N, \Sigma, P, S)$ be the input context-free grammar, $D = (V, E, \Sigma_V, \Sigma_E, \lambda_V, \lambda_E)$ be the input graph and $Src$ be the input set of start vertices.
The result of the algorithm is a Boolean matrix which represents relation $R_{G,D}^{Src}$.

\begin{algorithm}
\small
\begin{algorithmic}[1]
\caption{Multiple-source CFPQ algorithm}
\label{alg:algo1}
\Function{MultiSrcCFPQNaive}{\par
\hskip\algorithmicindent $D = (V, E, \Sigma_V, \Sigma_E, \lambda_V, \lambda_E)$, \par
\hskip\algorithmicindent $G=(N,\Sigma,P,S)$, \Comment{Grammar in WCNF}\par
\hskip\algorithmicindent
$Src$}
    \State{$T \gets \{T^A \mid  A \in N, T^A[i,j] \gets \textit{false} \text{, for all $i,j$}\} $}

    \State{$TSrc \gets \{TSrc^A \mid  A \in N, TSrc^A[i,j] \gets \textit{false} \text{, for all $i,j$}\}$}

    \ForAll{$ v \in Src$} \Comment{Input matrix initialization}
        \State{$TSrc^S[v,v] \gets true$}
    \EndFor

    \State $MSrc \gets TSrc^S$

    \ForAll{$A \to x \in P \mid x \in \Sigma_E$} \Comment{Simple rules initialization}
        \ForAll{$(v, to) \in E \mid x \in \lambda_E(v,to)$}
            \State{$T^A[v,to] \gets true$}
        \EndFor
    \EndFor

    \ForAll{$A \to x \in P~\mid x \in \Sigma_V$}
        \ForAll{$v \in V~|~ x \in \lambda_V(v)$}
            \State{$T^A[v,v] \gets true$}
        \EndFor
    \EndFor

    \While{$T\ or\ TSrc\ is\ changing$} \Comment{Algorithm's body}
        \ForAll{$A \to B C \in P$}
            \State{$M \gets TSrc^A*T^B$}
            \State{$T^A \gets T^A + M*T^C$}
            \State{$TSrc^B \gets TSrc^B + TSrc^A$}
            \State{$TSrc^C \gets TSrc^C + $ \Call{getDst}{$M$}}
        \EndFor
    \EndWhile
    \State \Return $MSrc * T^S$
\EndFunction

\Function{getDst}{$M$}
    \State{$A[i,j] \gets \textit{false}$}
    \ForAll{$(v,to) \in V^2 \mid M[v,to] = true$}
        \State{$A[to,to] \gets true$}
    \EndFor
    \State \Return A
\EndFunction
\end{algorithmic}
\end{algorithm}

In order to solve the single-source and multiple-source CFPQ problem, we modified the Azimov's algorithm.
Each time, when a grammar rule is applied, only vertices of interest should be stored.
To do it, we added one more matrix multiplication: $T^A = T^A + (TSrc^A \cdot T^B) \cdot T^ C$, where $TSrc^A$ is a matrix of start vertices for the current iteration (lines \textbf{15-16} of the Algorithm~\ref{alg:algo1}).
In the end of each iteration of for loop, it is necessary to update the set of source vertices paths from which we need to calculate.
To do it, we call the function \textsc{getDst} (see lines \textbf{20-24}), in line \textbf{18}.
Thus, the modified algorithm supports the frontier of the vertices of interest and updates it on each iteration.
Thus, it only computes the paths which starts from the small set of selected vertices.


%\subsection{Implementation Notes}

%The proposed algorithm has been implemented$\footnote{GitHub repository with implemented algorithms: \url{https://github.com/JetBrains-Research/CFPQ_PyAlgo}, last accessed 28.08.2020}$ using GraphBLAS framework that allows one to represent graphs as matrices and to work with them in terms of linear algebra.
%For~convenience, the code is written in Python using pygraphblas\footnote{GitHub repository of PyGraphBLAS library: \url{https://github.com/michelp/pygraphblas}}, which is a Python wrapper for GraphBLAS API and is based on SuiteSparse:GraphBLAS\footnote{GitHub repository of SuiteSparse:GraphBLAS library: \url{https://github.com/DrTimothyAldenDavis/SuiteSparse}}~\cite{10.1145/3322125} --- the complete implementation of the GraphBLAS standard.
%This library is specialized to work with sparse matrices which appear in real graphs most often.
