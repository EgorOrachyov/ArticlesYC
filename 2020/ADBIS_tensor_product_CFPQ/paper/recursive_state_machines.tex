\section{Recursive State Machines}
\label{section:rsm}

In this section, we introduce recursive state machines (RSM)~\cite{rsm:analysis:10.1007/3-540-44585-4_18}.
This kind of computational machine extends the definition of finite state machines and increases the computational capabilities of this formalism.

A recursive state machine $R$ over a finite alphabet $\Sigma$ is defined as a tuple of elements $(M,m,\{C_i\}_{i \in M})$, where:

\begin{itemize}
    \item $M$ is a finite set of labels of boxes.
    \item $m \in M$ is an initial box label.
    \item Set of \textit{component state machines} or \textit{boxes},
          where $C_i=(\Sigma \cup M, Q_i,q_i^0,F_i,\delta_i)$:
    \begin{itemize}
        \item $\Sigma \cup M$ is a set of symbols, $\Sigma \cap M = \emptyset$
        \item $Q_i$ is a finite set of states,
              where $Q_i \cap Q_j = \emptyset, \forall i \neq j$
        \item $q_i^0$ is an initial state for the component state machine $C_i$
        \item $F_i$ is a set of final states for $C_i$, where $F_i \subseteq Q_i$
        \item $\delta_i$ is a transition function for $C_i$,
              where $\delta_i: Q_i \times (\Sigma \cup M)
              \to Q_i$
    \end{itemize}
\end{itemize}

RSM behaves as a set of finite state machines (or FSM).
Each FSM is called a \textit{box} or a \textit{component state machine}~\cite{rsm:analysis:10.1007/3-540-44585-4_18}.
A box works almost the same as a classical FSM, but it also handles additional \textit{recursive calls} and employs an implicit \textit{call stack} to \textit{call} one component from another and then return execution flow back.

The execution of an RSM could be defined as a sequence of the configuration transitions, which are done on input symbols reading. 
The pair $(q_i,S)$, where $q_i$ is current state for box $C_i$ and $S$ is stack of \textit{return states}, describes execution configurations. 

The RSM execution starts form configuration $(q_m^0, \langle\rangle)$. 
The following list of rules defines the machine transition from configuration $(q_i,S)$ to $(q',S')$ on some input symbol $a$ from the input sequence, which is read as usual for FSA:

\begin{itemize}
    \item $q' \gets q_i^t, S' \gets S$, where $q_i^t = \delta_i (q_i^k, a), q_i^k = q$
    \item $q' \gets q_j^0, S' \gets q_i^t \circ S$, where $q_i^t = \delta_i (q_i^k, j),  q_i^k = q, j \in M$
    \item $q' \gets q_i^t, S' \gets S_{tail}$, where $S = q_i^t \circ S_{tail}, q_j^k = q, q_j^k \in F_j$
\end{itemize}

Some input sequence of the symbols $a_1 ... a_n$, which forms some input word, accepted, if machine reaches configuration $(q,\langle\rangle)$, where $q \in F_m$. It is also worth noting that the  RSM makes not deterministic transitions, without reading the input character when it \textit{calls} some component or makes a \textit{return}.

%The execution of RSM $R$ is a sequance of transitions between configurations of form $(q,S)$, where $S$ is global stack of \textit{return states} from $\bigcup_{i \in M}Q_i$ such as $S=\langle ...q_r \rangle$, and $q \in \bigcup_{i \in M}Q_i$ is a current state of the machine. The execution starts from box $m$ initial state $q_m^0$ and empty stack as follows $(q_m^0,\langle \rangle)$. Transitions for a given global machine state $(q,S)$ to some new state $(q',S')$ are defined as follows:

%\begin{itemize}
%    \item $q':=q_b', S':=S$,
%    where $\delta_b (q_b,s) \to q_b', s \in \Sigma,
%    q=q_b, S = \langle ...q_r \rangle$
%    \item $q':=q_t^0, S':=\langle ...q_r,q_b'\rangle$,
%    where $\delta_b (q_b,t) \to q_b', t \in M, q=q_b, S = \langle ...q_r \rangle$
%    \item $q':=q_b', S':=\langle ...q_r \rangle$,
%    where $q=q_t^i, q_t^i \in F_t, S=\langle ...q_r,q_b' \rangle$
%\end{itemize}

According to~\cite{rsm:analysis:10.1007/3-540-44585-4_18}, recursive state machines are equivalent to pushdown systems.
Since pushdown systems are capable of accepting context-free languages~\cite{automata:theory:10.5555/1177300}, it is clear that RSMs are equivalent to context-free languages.
Thus RSMs suit to encode query grammars.
Any CFG can be easily converted to an RSM with one box per nonterminal.
The box which corresponds to a nonterminal $A$ is constructed using the right-hand side of each rule for $A$.
An example of such RSM $R$ constructed for the grammar $G$ with rules $S \to a S b \mid a b$ is provided in figure~\ref{example:automata}.

%We escape detailed discussion of the conversion of a context free grammar $G$ to a RSM $R$. In the basic case, an algorithm for constructing such RSM could be composed from several stages, where each stage involves building of finite state machine for each non-terminal $N$ from grammar $G$.

Since $R$ is a set of FSMs, it is useful to represent $R$ as an adjacency matrix for the graph where vertices are states from $\bigcup_{i \in M}Q_i$ and edges are transitions between $q_i^a$ and $q_i^b$ with label $l \in \Sigma \cup M$, if $\delta_i (q_i^a, l) = q_i^b$.
An example of such adjacency matrix $M_R$ for the machine $R$ is provided in section~\ref{example:section}.

\begin{figure}[h]
    \begin{tikzpicture}[shorten >=1pt,auto]
        \node[state, initial] (q_0)   {$q_S^0$};
        \node[state] (q_1) [right=of q_0] {$q_S^1$};
        \node[state] (q_2) [right=of q_1] {$q_S^2$};
        \node[state, accepting] (q_3) [right=of q_2] {$q_S^3$};
        \path[->]
            (q_0) edge node {a} (q_1)
            (q_1) edge node {S} (q_2)
            (q_2) edge node {b} (q_3)
            (q_1) edge [bend left, above]  node {b} (q_3);
        \node[draw=black, fit= (q_0) (q_1) (q_2) (q_3), xshift=-4.5ex,inner sep=0.75cm, label=Box S] {};
    \end{tikzpicture}
    \centering
    \caption{The recursive state machine $R$ for grammar $G$}
    \label{example:automata}
\end{figure}

