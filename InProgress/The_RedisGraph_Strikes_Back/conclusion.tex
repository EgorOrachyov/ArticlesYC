\section{Conclusion and Future Work}
% Two is a number, yeah, but it is hardly a number...
In this paper we propose two version of multiple-source modifications of Azimov's CFPQ algorithm.
The evaluation of the proposed modifications on the real-world examples shows that caching of the queries results is not useful in the evaluated scenarios and the naive implementation is the best choice for the integration with a real-world graph database.
We also provide the full-stack support of CFPQ.
For our solution, we implement a Cypher extension as a part of \texttt{libcypher-parser}, integrate the proposed algorithm into RedisGraph, and extend RedisGraph execution plan builder to support the extended Cypher queries.
We demonstrate that our solution is applicable for real-world graph analysis.

In the future, it is necessary to provide formal translation of Cypher to linear algebra, or to determine a maximal subset of Cypher which can be translated to linear algebra.
There is a number of works on a subset of SPARQL to linear algebra translation, such as~\cite{10.14778/3229863.3236239,10.1007/978-3-642-34002-4_36,10.1145/3302424.3303962,DBLP:journals/corr/MetzlerM15a}.
Most of them are practical-oriented and do not provide full theoretical basis to translate querying language to linear algebra.
Others discuss only partial cases and should be extended to cover real-world query languages.
Deep investigation of this topic can help to determine the restrictions of linear algebra utilization for graph databases.
Moreover, it can also improve the existing solutions.

The evaluation of the regular queries is possible in practice by using the CFPQ algorithm, since the regular queries are a partial case of the context-free queries.
But it seems that the proposed solution is not optimal.
It is important to provide an optimal unified algorithm for both RPQ and CFPQ to create a tool applicable to a real-world tasks.
One of possible way to solve this problem is to use the tensor-based algorithm~\cite{10.1007/978-3-030-54832-2_6}.

Another important task is to compare non-linear-algebra-based approaches to the multiple-source CFPQ with the proposed solution.
In~\cite{Kuijpers:2019:ESC:3335783.3335791} Jochem Kuijpers et al. show that all-pairs CFPQ algorithms implemented in Neo4j demonstrate unreasonable performance on real-world data.
At the same time, Arseniy Terekhov et.al. shows that matrix-based all-pairs CFPQ algorithm implemented in the appropriate linear algebra based graph database (RedisGraph) demonstrates good performance.
But in the case of multiple-source scenario, when a number of start vertices is relatively small, non-linear-algebra-based solutions can be better, because such solutions naturally handle only subgraph required to answer the query.
Thus detailed investigation and comparison of other approaches to evaluate multiple-source CFPQ is required in the future.