\section{The reduction}

Suppose we have $\Phi$ --- an instance of 3-SAT problem contains $m$ clauses over $k$ variables.

First of all, we should to constract a graph. 
To do it we follow the next steps.
\begin{enumerate}
	\item 
	Let $\gamma_i = \{v_1 \leftarrow b_1, v_2 \leftarrow b_2, \cdots, v_k \leftarrow b_k\}$ where $b_k \in \{0,1\}$.
	For each substitution $\gamma_i$ a vertex $V_{\gamma_i}$ should be created.
	\item For each $V_{\gamma_i}$ the following edges should be added: $\{ (V_{\gamma_i}, [v_j \leftarrow b_l]^+ ,V_{\gamma_i}) \mid v_j \leftarrow b_l \in \gamma_i \}$.
	\item For each clause $(l_1 \vee l_2 \vee l_3)$ the following subgraph should be created.
	First, two new vertices are edded: $c_1$ and $c_2$.
	After that, the following edges for each $l_p$ and for each $\gamma_i$ should be added $$\{(c_1, [v_j \leftarrow b_l]^- ,c_2) \mid b_l = \systeme*{1 \text{ if } l_p = v_j, 0  \text{ if } l_p = \neg v_j} \}.$$
	\item Subgraph for all clauses should be connected sequencially. 
	Suppose we have seqence of subgraps with vertices $$\{(c_1^1,c_2^1),(c_1^2,c_2^2),\cdots,(c_1^m,c_2^m)\}.$$ To connect them we should merge vertices $c_2^{i}$ and $c_1^{i+1}$ for all $i$ except $i=m$.
	After that we fix $c_1^1$ as a start vertex of formula subgraph, and $c_2^m$ as a final vertex of formula subgraph.
    \item Finally, for all $V_{\gamma_i}$ we should add the following edge
    $$
    (V_{\gamma_i}, q ,c_1^1)
    $$
\end{enumerate}

The second part is a query.
Suppose, we have p different substitutions.
The gramamr is following
\begin{align*}
S & \to q \\
S & \to [v_1 \leftarrow b_1]^+ \ S \ [v_1 \leftarrow b_1]^-  \\
  & \mid \cdots \\
  & \mid  [v_k \leftarrow b_k]^+ \ S \ [v_k \leftarrow b_k]^- \\ 
\end{align*}

After that we should applay transformation which is described in the section~\ref{sec:cfpq_to_dyck}. 
As a result we get h-Dyck reachability problem (yes, we can reduce it to 2-Dyck reachability).

\subsection{An Example of Reduction}

Suppose we have the following instance of 3-SAT problem. 
$$
\Phi = (\neg x_1 \vee x_2 \vee \neg x_3) \wedge (\neg x_2 \vee x_1 \vee x_3) \wedge (x_1 \vee \neg x_3 \vee x_2)
$$

Substitutions:
\begin{align*}
\gamma_1 = \{x_1 \leftarrow 0, x_2 \leftarrow 0, x_3 \leftarrow 0\} \\
\gamma_2 = \{x_1 \leftarrow 1, x_2 \leftarrow 0, x_3 \leftarrow 0\} \\
\gamma_3 = \{x_1 \leftarrow 0, x_2 \leftarrow 1, x_3 \leftarrow 0\} \\
\gamma_4 = \{x_1 \leftarrow 0, x_2 \leftarrow 0, x_3 \leftarrow 1\} \\
\gamma_5 = \{x_1 \leftarrow 1, x_2 \leftarrow 1, x_3 \leftarrow 0\} \\
\gamma_6 = \{x_1 \leftarrow 1, x_2 \leftarrow 0, x_3 \leftarrow 1\} \\
\gamma_7 = \{x_1 \leftarrow 0, x_2 \leftarrow 1, x_3 \leftarrow 1\} \\
\gamma_8 = \{x_1 \leftarrow 1, x_2 \leftarrow 1, x_3 \leftarrow 1\} 
\end{align*}

Graph for $\Phi$ is presented in figure~\ref{fig:sat_to_cfpq_graph_example}.

The grammar:
\begin{align*}
S & \to  [x_1 \leftarrow 0]^+ \ S \ [x_1 \leftarrow 0]^- \\
  & \mid [x_2 \leftarrow 0]^+ \ S \ [x_2 \leftarrow 0]^- \\ 
  & \mid [x_3 \leftarrow 0]^+ \ S \ [x_3 \leftarrow 0]^- \\
  & \mid [x_1 \leftarrow 1]^+ \ S \ [x_1 \leftarrow 1]^- \\ 
  & \mid [x_2 \leftarrow 1]^+ \ S \ [x_2 \leftarrow 1]^- \\ 
  & \mid [x_3 \leftarrow 1]^+ \ S \ [x_3 \leftarrow 1]^- \\
  & \mid q \\
\end{align*}

The intuition of such path finding is that substitution vertex ($V_{\gamma_i}$) should provide appropriate values for respective variable in appropriate order to satisfy the given formula.
It can be done by appropriate traversing of loops. 
After that, each edge from $c_i^j$ to  $c_l^k$ ``uses'' provided values to satisfie respective closure, and it can be done if and only if the respective vertex provides value required.
This fact is expressed by usung balanced-bracket grammar.
So, if there exists a path from $V_{\gamma_i}$ to $c_2^3$, such that the corresponded word is derivable from $S$, then $V_{\gamma_i}$ satisfy the given formula. 


\begin{figure*}
\begin{tikzpicture}[-,>=stealth',shorten >=1pt,auto,node distance=3.5cm,
  thick,main node/.style={circle,fill=black!20,draw}]

  
  \node[main node] (7) [] {$V_{\gamma_3}$};
  \node[main node] (8) [below of=7] {$V_{\gamma_4}$};
  \node[main node] (9) [below of=8] {$V_{\gamma_5}$};


  \node[main node] (1) [right of=8] {$c_1^1$};
  \node[main node] (2) [right of=1] {$c_2^1$};
  \node[main node] (3) [right of=2] {$c_2^2$};
  \node[main node] (4) [right of=3] {$c_2^3$};

  \node[main node] (5) [above of=2] {$V_{\gamma_1}$};
  \node[main node] (6) [above of=1] {$V_{\gamma_2}$};
  \node[main node] (10) [below of=1] {$V_{\gamma_6}$};
  \node[main node] (11) [below of=2] {$V_{\gamma_7}$};
  \node[main node] (12) [below of=3] {$V_{\gamma_8}$};


  \path[-> , every node/.style={font=\sffamily\small}]
    (5) edge [loop right, right] node  {$[x_1 \leftarrow 0]^+$}   (5)    
    (5) edge [loop above, right] node  {$[x_2 \leftarrow 0]^+$}  (5)    
    (5) edge [loop below, right] node  {$[x_3 \leftarrow 0]^+$}  (5)

    (5) edge [] node  {q}  (1)    
    

    (6) edge [loop left, left] node   {$[x_1 \leftarrow 1]^+$}   (6)    
    (6) edge [loop above, left] node  {$[x_2 \leftarrow 0]^+$}  (6)    
    (6) edge [loop right, right] node  {$[x_3 \leftarrow 0]^+$}  (6)

    (6) edge [] node  {q}  (1)    
         


    (7) edge [loop left, left] node   {$[x_1 \leftarrow 0]^+$}   (7)    
    (7) edge [loop above, left] node  {$[x_2 \leftarrow 1]^+$}  (7)    
    (7) edge [loop below, left] node  {$[x_3 \leftarrow 0]^+$}  (7)

    (7) edge [] node   {q}  (1)    
    


    (8) edge [loop left, left] node   {$[x_1 \leftarrow 0]^+$}   (8)    
    (8) edge [loop above, left] node  {$[x_2 \leftarrow 0]^+$}  (8)    
    (8) edge [loop below, left] node  {$[x_3 \leftarrow 1]^+$}  (8)

    (8) edge [] node  {q}  (1)    
    

    (9) edge [loop left, left] node   {$[x_1 \leftarrow 1]^+$}   (9)    
    (9) edge [loop above, left] node  {$[x_2 \leftarrow 1]^+$}  (9)    
    (9) edge [loop below, left] node  {$[x_3 \leftarrow 0]^+$}  (9)

    (9) edge [] node  {q}  (1)    

    
    (10) edge [loop left, left] node   {$[x_1 \leftarrow 1]^+$}   (10)    
    (10) edge [loop right, right] node  {$[x_2 \leftarrow 0]^+$}  (10)    
    (10) edge [loop below, left] node  {$[x_3 \leftarrow 1]^+$}  (10)

    (10) edge [] node   {q}  (1)    

    
    (11) edge [loop above, right] node    {$[x_1 \leftarrow 0]^+$}   (11)    
    (11) edge [loop below, below] node  {$[x_2 \leftarrow 1]^+$}  (11)    
    (11) edge [loop right, right] node  {$[x_3 \leftarrow 1]^+$}  (11)

    (11) edge [] node  {q}  (1)


    (12) edge [loop right, right] node  {$[x_1 \leftarrow 1]^+$}   (12)    
    (12) edge [loop above, right] node  {$[x_2 \leftarrow 1]^+$}  (12)    
    (12) edge [loop below, right] node  {$[x_3 \leftarrow 1]^+$}  (12)

    (12) edge [] node  {q}  (1)    
 

    (1) edge [bend left=30, above] node   {$[x_1 \leftarrow 0]^-$}  (2)
    (1) edge [bend left=0, above] node   {$[x_2 \leftarrow 1]^-$}  (2)
    (1) edge [bend right=30, above] node   {$[x_3 \leftarrow 0]^-$}  (2)
      

    (2) edge [bend left=30, above] node  {$[x_1 \leftarrow 1]^-$}  (3)
    (2) edge [bend left=0, above] node  {$[x_2 \leftarrow 0]^-$}  (3)
    (2) edge [bend right=30, above] node  {$[x_3 \leftarrow 1]^-$}  (3)
   

    (3) edge [bend left=30, above] node  {$[x_1 \leftarrow 1]^-$}  (4)
    (3) edge [bend left=0, above] node  {$[x_2 \leftarrow 1]^-$}  (4)
    (3) edge [bend right=30, above] node  {$[x_3 \leftarrow 0]^-$}  (4)
    
    ;

\end{tikzpicture}
\caption{Example of graph for $\Phi$}
\label{fig:sat_to_cfpq_graph_example}
\end{figure*}

\section{Reduction to $k/3$}

	
Firs step is to split variables into three group of the same size.
Suppose this splitting preserves the order.
So, we have a set of partial substitution. 

By the same way we create vertices for partial subctitutions.


\section{An Example of Reduction $k/3$}

For our example:
\begin{align*}
\gamma_1^1 = \{x_1 \leftarrow 1 \} \\
\gamma_1^2 = \{x_1 \leftarrow 0 \} \\
\gamma_2^1 = \{x_2 \leftarrow 1 \} \\
\gamma_2^2 = \{x_2 \leftarrow 0 \} \\
\gamma_3^1 = \{x_3 \leftarrow 1 \} \\
\gamma_3^2 = \{x_3 \leftarrow 0 \} 
\end{align*}

Grammar:
\begin{align*}
S & \to S_1 \mid S_2 \mid S_3 \mid S_4 \mid S_5 \mid S_6 \mid S_7 \mid S_8\\
S_1 & \to q \mid p^+ \ S_1 \ p^- 
      \mid [x_1 \leftarrow 0]^+ \ S_1 \ [x_1 \leftarrow 0]^- \\ 
    & \mid [x_2 \leftarrow 0]^+ \ S_1 \ [x_2 \leftarrow 0]^- 
      \mid [x_3 \leftarrow 0]^+ \ S_1 \ [x_3 \leftarrow 0]^- \\
S_2 & \to q \\
    & \mid p^+ \ S_2 \ p^- 
      \mid [x_1 \leftarrow 1]^+ \ S_2 \ [x_1 \leftarrow 1]^- \\ 
    & \mid [x_2 \leftarrow 0]^+ \ S_2 \ [x_2 \leftarrow 0]^- 
      \mid [x_3 \leftarrow 0]^+ \ S_2 \ [x_3 \leftarrow 0]^- \\
S_3 & \to q \mid p^+ \ S_3 \ p^- 
      \mid [x_1 \leftarrow 0]^+ \ S_3 \ [x_1 \leftarrow 0]^- \\ 
    & \mid [x_2 \leftarrow 1]^+ \ S_3 \ [x_2 \leftarrow 1]^- 
      \mid [x_3 \leftarrow 0]^+ \ S_3 \ [x_3 \leftarrow 0]^- \\
S_4 & \to q \mid p^+ \ S_4 \ p^- 
      \mid [x_1 \leftarrow 0]^+ \ S_4 \ [x_1 \leftarrow 0]^- \\ 
    & \mid [x_2 \leftarrow 0]^+ \ S_4 \ [x_2 \leftarrow 0]^- 
      \mid [x_3 \leftarrow 1]^+ \ S_4 \ [x_3 \leftarrow 1]^- \\
S_5 & \to q \mid p^+ \ S_5 \ p^- 
      \mid [x_1 \leftarrow 1]^+ \ S_5 \ [x_1 \leftarrow 1]^- \\ 
    & \mid [x_2 \leftarrow 1]^+ \ S_5 \ [x_2 \leftarrow 1]^- 
      \mid [x_3 \leftarrow 0]^+ \ S_5 \ [x_3 \leftarrow 0]^- \\
S_6 & \to q \mid p^+ \ S_6 \ p^- 
      \mid [x_1 \leftarrow 1]^+ \ S_6 \ [x_1 \leftarrow 1]^- \\ 
    & \mid [x_2 \leftarrow 0]^+ \ S_6 \ [x_2 \leftarrow 0]^- 
      \mid [x_3 \leftarrow 1]^+ \ S_6 \ [x_3 \leftarrow 1]^- \\
S_3 & \to q \mid p^+ \ S_7 \ p^- 
      \mid [x_1 \leftarrow 0]^+ \ S_7 \ [x_1 \leftarrow 0]^- \\ 
    & \mid [x_2 \leftarrow 1]^+ \ S_7 \ [x_2 \leftarrow 1]^- 
      \mid [x_3 \leftarrow 1]^+ \ S_7 \ [x_3 \leftarrow 1]^- \\
S_8 & \to q \mid p^+ \ S_8 \ p^- 
      \mid [x_1 \leftarrow 1]^+ \ S_8 \ [x_1 \leftarrow 1]^- \\ 
    & \mid [x_2 \leftarrow 1]^+ \ S_8 \ [x_2 \leftarrow 1]^- 
      \mid [x_3 \leftarrow 1]^+ \ S_8 \ [x_3 \leftarrow 1]^- \\
\end{align*}

\begin{figure*}
\begin{tikzpicture}[-,>=stealth',shorten >=1pt,auto,node distance=3.5cm,
  thick,main node/.style={circle,fill=black!20,draw}]

  
  \node[main node] (6) [] {$V_{\gamma_1^1}$};
  \node[main node] (7) [right of =6] {$V_{\gamma_1^2}$};
  \node[main node] (8) [below of=6] {$V_{\gamma_2^1}$};
  \node[main node] (9) [right of=8] {$V_{\gamma_2^2}$};
  \node[main node] (10) [below of=8] {$V_{\gamma_3^1}$};
  \node[main node] (11) [right of=10] {$V_{\gamma_3^2}$};
  


  \node[main node] (1) [right of=9] {$c_1^1$};
  \node[main node] (2) [right of=1] {$c_2^1$};
  \node[main node] (3) [right of=2] {$c_2^2$};
  \node[main node] (4) [right of=3] {$c_2^3$};

  
  \path[-> , every node/.style={font=\sffamily\small}]
    (6) edge [loop above, above] node  {$[x_1 \leftarrow 1]^+$}   (6)    
    
    (6) edge [bend left = 80, color=red] node  {q}  (1)        
    (6) edge [color=blue] node  {$p^+$}  (8)
    (6) edge [color=blue, sloped, bend left=15] node  {$p^+$}  (9)                


    (7) edge [loop above, left] node  {$[x_1 \leftarrow 0]^+$}   (7)    
    
    (7) edge [color=red] node {q}  (1)        
    (7) edge [color=blue, sloped, above, bend right=20] node  {$p^+$}  (8)
    (7) edge [color=blue] node  {$p^+$}  (9) 


    (8) edge [loop left, sloped, above] node  {$[x_2 \leftarrow 1]^+$}   (8)    
    
    (8) edge [bend right=35, sloped, above, color=red] node  {q}  (1)        
    (8) edge [color=blue] node  {$p^+$}  (10)
    (8) edge [color=blue, sloped, bend right=20] node  {$p^+$}  (11) 

    (9) edge [loop left, above, sloped] node  {$[x_2 \leftarrow 0]^+$}   (9)    
    
    (9) edge [color=red,bend left] node  {q}  (1)        
    (9) edge [color=blue, sloped, above] node  {$p^+$}  (10)
    (9) edge [color=blue] node  {$p^+$}  (11)     
    
    (10) edge [loop below, below] node  {$[x_3 \leftarrow 1]^+$}   (10)    
    
    (10) edge [bend right=80,color=red] node  {q}  (1)        
    (10) edge [bend left=45, color=blue] node  {$p^+$}  (6)
    (10) edge [bend left=3, color=blue, above, sloped] node  {$p^+$}  (7) 
    
    (11) edge [loop below, left] node  {$[x_3 \leftarrow 0]^+$}   (11)    
    
    (11) edge [color=red] node  {q}  (1) 
    (11) edge [bend left=13, color=blue, below, sloped] node  {$p^+$}  (6)
    (11) edge [bend right, color=blue] node  {$p^+$}  (7)


    (1) edge [bend left=60, above] node  {$[x_1 \leftarrow 0]^-$}  (2)
    (1) edge [bend left=30, above] node  {$[x_2 \leftarrow 1]^-$}  (2)
    (1) edge [bend right=30, above] node  {$[x_3 \leftarrow 0]^-$}  (2)
    (1) edge [loop right, right, color=blue] node {$p^-$} (1)
    


    (2) edge [bend left=60, above] node  {$[x_1 \leftarrow 1]^-$}  (3)
    (2) edge [bend left=30, above] node  {$[x_2 \leftarrow 0]^-$}  (3)
    (2) edge [bend right=30, above] node  {$[x_3 \leftarrow 1]^-$}  (3)
    (2) edge [loop right, right, color=blue] node {$p^-$} (2)
    

    (3) edge [bend left=60, above] node  {$[x_1 \leftarrow 1]^-$}  (4)
    (3) edge [bend left=30, above] node  {$[x_2 \leftarrow 1]^-$}  (4)
    (3) edge [bend right=30, above] node  {$[x_3 \leftarrow 0]^-$}  (4)
    (3) edge [loop right, right, color=blue] node {$p^-$} (3)
    
    (4) edge [loop left, left, color=blue] node {$p^-$} (4)
    

    ;

\end{tikzpicture}
\caption{Example of graph for $\Phi$}
\label{fig:sat_to_cfpq_graph_example}
\end{figure*}


\section{Improved Reduction to $k/3$}

	
Firs step is to split variables into three group of the same size.
Suppose this splitting preserves the order.
So, we have a set of partial substitution. 

By the same way we create vertices for partial subctitutions.


\section{An Example of Improved Reduction $k/3$}

For our example:
\begin{align*}
\gamma_1^1 = \{x_1 \leftarrow 1 \} \\
\gamma_1^2 = \{x_1 \leftarrow 0 \} \\
\gamma_2^1 = \{x_2 \leftarrow 1 \} \\
\gamma_2^2 = \{x_2 \leftarrow 0 \} \\
\gamma_3^1 = \{x_3 \leftarrow 1 \} \\
\gamma_3^2 = \{x_3 \leftarrow 0 \} 
\end{align*}

Grammar:
\begin{align*}
S & \to S_{\gamma_1} \mid S_{\gamma_2} \mid S_{\gamma_3} \mid S_{\gamma_4} \mid S_{\gamma_5} \mid S_{\gamma_6} \mid S_{\gamma_7} \mid S_{\gamma_8}\\
S_1 & \to q  
      \mid [x_1 \leftarrow 1]^+ \ S_1 \ [x_1 \leftarrow 1]^- \\ 
S_2 & \to q  
      \mid [x_1 \leftarrow 0]^+ \ S_2 \ [x_1 \leftarrow 0]^- \\ 
S_3 & \to q  
      \mid [x_2 \leftarrow 1]^+ \ S_3 \ [x_2 \leftarrow 1]^- \\ 
S_4 & \to q 
      \mid [x_2 \leftarrow 0]^+ \ S_4 \ [x_2 \leftarrow 0]^- \\ 
S_5 & \to q
      \mid [x_3 \leftarrow 1]^+ \ S_5 \ [x_3 \leftarrow 1]^- \\ 
S_6 & \to q
      \mid [x_3 \leftarrow 0]^+ \ S_6 \ [x_3 \leftarrow 0]^- \\ 
S_{\gamma_1} & \to S_2 \mid S_4 \mid S_6 \mid S_{\gamma_1} \ p \ S_2 \mid S_{\gamma_1} \ p \ S_4 \mid S_{\gamma_1} \ p \ S_6\\
S_{\gamma_2} & \to S_1 \mid S_4 \mid S_6 \mid S_{\gamma_2} \ p \ S_1 \mid S_{\gamma_2} \ p \ S_4 \mid S_{\gamma_2} \ p \ S_6\\
S_{\gamma_3} & \to S_2 \mid S_3 \mid S_6 \mid S_{\gamma_3} \ p \ S_2 \mid S_{\gamma_3} \ p \ S_3 \mid S_{\gamma_3} \ p \ S_6\\
S_{\gamma_4} & \to S_2 \mid S_4 \mid S_5 \mid S_{\gamma_4} \ p \ S_2 \mid S_{\gamma_4} \ p \ S_4 \mid S_{\gamma_4} \ p \ S_5\\
S_{\gamma_5} & \to S_1 \mid S_3 \mid S_6 \mid S_{\gamma_5} \ p \ S_1 \mid S_{\gamma_5} \ p \ S_3 \mid S_{\gamma_5} \ p \ S_6\\
S_{\gamma_6} & \to S_1 \mid S_4 \mid S_5 \mid S_{\gamma_6} \ p \ S_1 \mid S_{\gamma_6} \ p \ S_4 \mid S_{\gamma_6} \ p \ S_5\\
S_{\gamma_7} & \to S_2 \mid S_3 \mid S_5 \mid S_{\gamma_7} \ p \ S_2 \mid S_{\gamma_7} \ p \ S_3 \mid S_{\gamma_7} \ p \ S_5\\
S_{\gamma_8} & \to S_1 \mid S_3 \mid S_5 \mid S_{\gamma_8} \ p \ S_1 \mid S_{\gamma_8} \ p \ S_3 \mid S_{\gamma_8} \ p \ S_5\\
\end{align*}


Grammar in EBNF (better for tensor-based algorithm):
\begin{align*}
S & \to S_{\gamma_1} \mid S_{\gamma_2} \mid S_{\gamma_3} \mid S_{\gamma_4} \mid S_{\gamma_5} \mid S_{\gamma_6} \mid S_{\gamma_7} \mid S_{\gamma_8}\\
S_1 & \to q  
      \mid [x_1 \leftarrow 1]^+ \ S_1 \ [x_1 \leftarrow 1]^- \\ 
S_2 & \to q  
      \mid [x_1 \leftarrow 0]^+ \ S_2 \ [x_1 \leftarrow 0]^- \\ 
S_3 & \to q  
      \mid [x_2 \leftarrow 1]^+ \ S_3 \ [x_2 \leftarrow 1]^- \\ 
S_4 & \to q 
      \mid [x_2 \leftarrow 0]^+ \ S_4 \ [x_2 \leftarrow 0]^- \\ 
S_5 & \to q
      \mid [x_3 \leftarrow 1]^+ \ S_5 \ [x_3 \leftarrow 1]^- \\ 
S_6 & \to q
      \mid [x_3 \leftarrow 0]^+ \ S_6 \ [x_3 \leftarrow 0]^- \\ 
S_{\gamma_1} & \to (S_2 \mid S_4 \mid S_6) (p \ (S_2 \mid S_4 \mid S_6))^*\\
S_{\gamma_2} & \to (S_1 \mid S_4 \mid S_6) (p \ (S_1 \mid S_4 \mid S_6))^*\\
S_{\gamma_3} & \to (S_2 \mid S_3 \mid S_6) (p \ (S_2 \mid S_3 \mid S_6))^*\\
S_{\gamma_4} & \to (S_2 \mid S_4 \mid S_5) (p \ (S_2 \mid S_4 \mid S_5))^*\\
S_{\gamma_5} & \to (S_1 \mid S_3 \mid S_6) (p \ (S_1 \mid S_3 \mid S_6))^*\\
S_{\gamma_6} & \to (S_1 \mid S_4 \mid S_5) (p \ (S_1 \mid S_4 \mid S_5))^*\\
S_{\gamma_7} & \to (S_2 \mid S_3 \mid S_5) (p \ (S_2 \mid S_3 \mid S_5))^*\\
S_{\gamma_8} & \to (S_1 \mid S_3 \mid S_5) (p \ (S_1 \mid S_3 \mid S_5))^*\\
\end{align*}


\begin{figure*}
\begin{tikzpicture}[-,>=stealth',shorten >=1pt,auto,node distance=3.5cm,
  thick,main node/.style={circle,fill=black!20,draw}]

  \node[main node] (1) [right of=9] {$c_1^1$};
  \node[main node] (2) [right of=1] {$c_2^1$};
  \node[main node] (3) [right of=2] {$c_2^2$};
  \node[main node] (4) [right of=3] {$c_2^3$};


  \node[main node] (6) [above of=3] {$V_{\gamma_1^1}$};
  \node[main node] (7) [above of=2] {$V_{\gamma_1^2}$};
  \node[main node] (8) [above of=1] {$V_{\gamma_2^1}$};
  \node[main node] (9) [below of=1] {$V_{\gamma_2^2}$};
  \node[main node] (10) [below of=2] {$V_{\gamma_3^1}$};
  \node[main node] (11) [below of=3] {$V_{\gamma_3^2}$};
  

  
  \path[-> , every node/.style={font=\sffamily\small}]
    (6) edge [loop above, above] node  {$[x_1 \leftarrow 1]^+$}   (6)    
    (6) edge [bend left=5, color=red] node  {q}  (1)        
    (6) edge [bend left=5, color=red] node  {q}  (2)        
    (6) edge [bend left=5, color=red] node  {q}  (3)        
    (1) edge [bend left=5, color=blue] node  {p}  (6)        
    (2) edge [bend left=5, color=blue] node  {p}  (6)        
    (3) edge [bend left=5, color=blue] node  {p}  (6)    

    (7) edge [loop above, above] node  {$[x_1 \leftarrow 0]^+$}   (7)    
    (7) edge [bend left=5, color=red] node  {q}  (1)        
    (7) edge [bend left=5, color=red] node  {q}  (2)        
    (7) edge [bend left=5, color=red] node  {q}  (3)        
    (1) edge [bend left=5, color=blue] node  {p}  (7)        
    (2) edge [bend left=5, color=blue] node  {p}  (7)        
    (3) edge [bend left=5, color=blue] node  {p}  (7)    

    (8) edge [loop above, above] node  {$[x_2 \leftarrow 1]^+$}   (8)    
    (8) edge [bend left=5, color=red] node  {q}  (1)        
    (8) edge [bend left=5, color=red] node  {q}  (2)        
    (8) edge [bend left=5, color=red] node  {q}  (3)        
    (1) edge [bend left=5, color=blue] node  {p}  (8)        
    (2) edge [bend left=5, color=blue] node  {p}  (8)        
    (3) edge [bend left=5, color=blue] node  {p}  (8)    

    (9) edge [loop below, below] node  {$[x_2 \leftarrow 0]^+$}   (9)    
    (9) edge [bend left=5, color=red] node  {q}  (1)        
    (9) edge [bend left=5, color=red] node  {q}  (2)        
    (9) edge [bend left=5, color=red] node  {q}  (3)        
    (1) edge [bend left=5, color=blue] node  {p}  (9)        
    (2) edge [bend left=5, color=blue] node  {p}  (9)        
    (3) edge [bend left=5, color=blue] node  {p}  (9)    
    
    (10) edge [loop below, below] node  {$[x_3 \leftarrow 1]^+$}   (10)    
    (10) edge [bend left=5, color=red] node  {q}  (1)        
    (10) edge [bend left=5, color=red] node  {q}  (2)        
    (10) edge [bend left=5, color=red] node  {q}  (3)        
    (1) edge [bend left=5, color=blue] node  {p}  (10)        
    (2) edge [bend left=5, color=blue] node  {p}  (10)        
    (3) edge [bend left=5, color=blue] node  {p}  (10)    
    
    (11) edge [loop below, below] node  {$[x_3 \leftarrow 0]^+$}   (11)    
    (11) edge [bend left=5, color=red] node  {q}  (1)        
    (11) edge [bend left=5, color=red] node  {q}  (2)        
    (11) edge [bend left=5, color=red] node  {q}  (3)        
    (1) edge [bend left=5, color=blue] node  {p}  (11)        
    (2) edge [bend left=5, color=blue] node  {p}  (11)        
    (3) edge [bend left=5, color=blue] node  {p}  (11)    
    

    (1) edge [bend left=30, above] node  {$[x_1 \leftarrow 0]^-$}  (2)
    (1) edge [bend left=0, above] node  {$[x_2 \leftarrow 1]^-$}  (2)
    (1) edge [bend right=30, above] node  {$[x_3 \leftarrow 0]^-$}  (2)
    
    (2) edge [bend left=30, above] node  {$[x_1 \leftarrow 1]^-$}  (3)
    (2) edge [bend left=0, above] node  {$[x_2 \leftarrow 0]^-$}  (3)
    (2) edge [bend right=30, above] node  {$[x_3 \leftarrow 1]^-$}  (3)
    
    (3) edge [bend left=30, above] node  {$[x_1 \leftarrow 1]^-$}  (4)
    (3) edge [bend left=0, above] node  {$[x_2 \leftarrow 1]^-$}  (4)
    (3) edge [bend right=30, above] node  {$[x_3 \leftarrow 0]^-$}  (4)

    ;

\end{tikzpicture}
\caption{Example of graph for $\Phi$}
\label{fig:sat_to_cfpq_graph_example}
\end{figure*}