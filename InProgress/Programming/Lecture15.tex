\section{Лекуия 15. Основы парарллельного программирования}

Раздел 4: Параллельное программирование
    1. Архитектуры, подходы, парадигмы. SIMD, MIMD, SPMD. Асинхронное программирование, параллельное программирование. Процессы и потоки: многопроцессорность и многопоточность. Гонки по данным, блокировки.  
    2. Базовые примитивы работы с потоками и разделяемыми ресурсами в F\#. Функция lock. Запуск функции в отдельном потоке. Особенности работы с исключениями. Общее состояние. Плюсы и минусы неизменяемости.
    3. Array.Parallel, ParallelSeq и другие высокоуровневые средства параллельного программирования на F\#. Линейная алгебра и параллелизм: бонусы, проблемы, возможные решения.
