\section{Домашняя работа 8}

При создании новых структур данных необходимо расширять библиотеку, созданную в предыдущей домашней работе.

\begin{enumerate}

    \item \textbf{(5 баллов)} Используя тип MyList из предыдущей домашней работы, реализовать целочисленную длинную арифметику: операции сложения, умножения, вычитания, целочисленного деления. 

    \item \textbf{(2 балла)} Реализовать представление разреженных матриц в виде дерева квадрантов. Реализовать функцию поэлементного сложения двух матриц в таком формате.
    
    \item \textbf{(3 балла)} Реализовать функцию умножения двух матриц в формате дерева квадрантов.
    
    \item \textbf{(3 балла)} Реализовать функцию тензорного умножения двух матриц в формате дерева квадрантов.
    
    \item \textbf{(5 баллов)} Реализовать построение транзитивного замыкания ориентированного графа через произведение матриц. Использовать представление матриц из второй задачи. Визуализировать результат с помощью GraphViz: исходный граф и выделенные рёбра, появившиеся в результате транзитивного замыкания. Для задания графа использовать формат из задачи 2 6-й домашней работы.
    
    \item \textbf{(5 баллов)} Реализовать вычисление кратчайших путей между всеми парами вершин в ориентированном графе. Использовать представление матриц из второй задачи. Визуализировать результат с помощью GraphViz: исходный граф и выделенные рёбра со значением кратчайшего пути между соответствующей парой вершин. Для задания графа использовать формат, аналогичный формату из задачи 2 6-й домашней работы.
\end{enumerate}