\section{САКОД 1: Основы обработки изображений}

Форматы изображений.

Векторный, растровый.

Нас интересует растровый.

И для простоты сразу битмапа (bmp): двумерный массив пикселей, где для каждого хранится цвет.
Естественно, ещё и метаданные вокруг, но они, в основном, про то, как ситать файл, а не про само изображение.
Цвет --- либо RGB, либо градации серого (Grayscale). На пиксель от 1 до 64 бит. В grayscale 16 или 32.

\href{Bitmap in .NET}{https://docs.microsoft.com/en-us/dotnet/api/system.drawing.bitmap?view=dotnet-plat-ext-5.0}


Фильтры.

Собель для поиска границ, размытие по Гауссу, машинное обучение


Реализовать приложение с графическим интерфейсом пользователя, позволяющее открыть папку с изображениями, выбрать изображение, просмотреть его, просмотреть информацию о нём (размер в пикселях, размер в мегабайтах). 

    1. (1 балл) Расширить приложение из предыдущей домашней работы графической компонентой задания матричного фильтра. Необходимо предусмотреть возможность выбора типа фильтра, дефолтных значений, размера фильтра, корректировку весов.
    2. (1 балл) Расширить приложение из предыдущей работы возможностью отображать одновременно два изображения: до и после применения фильтра. Предусмотреть возможность сохранять результат применения фильтра.
    3. (3 балла) Реализовать применение матричных фильтров с использованием GPGPU. Интегрировать с разработанным графическим интерфейсом. Предусмотреть возможность применения нескольких фильтров последовательно.
    4. (6 баллов) Расширить разрабатываемое приложение возможностью потоковой обработки изображений: выбираем папку с изображениями и ко всем применяем заданный фильтр. Результаты применения фильтров сохраняются.
    5. (5 баллов) Подготовить отчёт с анализом производительности и масштабируемости полученного решения.


    Домашки:
    --- гитхаб
    --- ридми
    --- тесты
    --- CI
    --- build-script
    --- ubuntu