\section{Практика 12}

\begin{enumerate}

    \item \textbf{[4 балла]} Реализовать консольный генератор матриц. На вход принимается размер матрицы, тип данных, количество матриц, метрика разреженности, возможно другие необходимые параметры. В результате генерируется набор файлов с матрицами в формате из первого семестра. 
    \item \textbf{[9 баллов]} Реализовать параллельное умножение для плотных матриц. Исследовать варианты с распараллеливанием различных циклов. Реализовать как с использованием async, так и с использованием Array.Parallel. Для исследования использовать созданный ранее генератор. Оформить соответствующий отчёт.
    \item \textbf{[6 баллов]} Реализовать параллельное умножение матриц, представленных в виде дерева квадрантов.
    \item \textbf{[10 баллов]} Сравнить производительность решений из первых двух пунктов на разных типах матриц. Для исследования использовать созданный ранее генератор. Оформить соответствующий отчёт. 
    
\end{enumerate}

