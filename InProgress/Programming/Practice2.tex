\section{Домашняя работа 2}

В задачах, связанных с обработкой массивов на вход необходимо принимать длину массива и затем создавать случайный массив соответствующей длины. Для всех задач необходимо реализовать чтение входных данных из консоли и вывод результата в консоль.

Задачи:
\begin{enumerate}
    \item \textbf{(1 балл)} Реализовать функцию, вычисляющую значение выражения $x^4+x^3+x^2+x+1$ ``наивным'' способом. 
    \item \textbf{(1 балл)} Реализовать функцию, вычисляющую значение выражения $x^4+x^3+x^2+x+1$, применив минимальное число умножений и сложений.
    \item \textbf{(1 балл)} Вычислить индексы элементов массива, не больших, чем заданное число.
    \item \textbf{(1 балл)} Вычислить индексы элементов массива, лежащих вне диапазона, заданного двумя числами.
    \item \textbf{(1 балл)} Дан массив длины 2. Поменять местами нулевой и первый элементы, не используя дополнительной памяти/переменных.
    \item \textbf{(1 балл)} Поменять местами $i$-й и $j$-й элементы массива не используя дополнительной памяти/переменных.
\end{enumerate}