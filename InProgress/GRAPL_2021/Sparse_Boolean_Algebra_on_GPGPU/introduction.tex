\section{introduction}

% TODO:
% - links to problems sovled
% - sp boolean linear algebra applications links
% - possible fpga utilization

One of the techniques to effectively solve a data analysis problem is to reduce it to linear
algebra operations over vectors and matrices for appropriate values set. That gives one well 
studied for years mathematical apparatus, as well as the possibility to evaluate this problem with 
\textit{zero-cost} by linear algebra libraries, which utilize modern hardware, provide various 
optimization techniques and allow quickly and safely prototype solution in code with predefined 
building blocks.

Particularly, in graph data analysis such reduction is well presented, since a graph could be 
converted to the matrix and with specially defined \textit{semiring} it could be effectively 
process in linear algebra fashion. Examples of problems solved in this way are all-pairs shortest 
path, breadth-first search, maximal independent set problems~\cite{todo}. Since practical data 
often come with huge size and sparse form, what is also applicable for graphs, it requires special 
processing tools for analysis, what appeals to sparse linear algebra libraries.

Huge amount data analysis typically processed as small chunks with the fixed or rarely changed set 
of instructions, what relates to the \textit{single instruction, multiple data} (SIMD) model. 
Although modern CPUs exploit SIMD optimizations, GPGPU gives much more power in such kind data 
processing at cost of implementation challenges. GPGPU programming introduces heterogeneous device 
model into system, memory traffic and data operations limitations, as well as requires to take into
account vendor-specific capabilities. 

At this moment there are presented modern open-source and proprietary sparse linear algebra 
libraries on GPGPU for general operations and common types of values. However, sparse boolean 
linear algebra on GPGPU is still not presented, because of its high specificity. Boolean algebra 
allows to address problems over finite set of values, for example, reachability or relational 
queries for some graphs~\cite{todo}.

In this work we present the sparse boolean linear algebra operations implementation as sand-alone 
self-sufficient programming libraries for the two most popular GPGPU platforms: NVIDIA Cuda and 
OpenCL. Cuda is a GPGPU technology for NVIDIA devices, which allows to employ some 
platform-specific facilities, such as unified memory mechanism, and make an architectural 
assumptions, what gives more optimizations space at cost of portability. OpenCL is platform 
agnostic API standard, which allows to run computations on different platforms, such as 
multi-threaded CPUs, GPUs, and FPGA. Our implementation relies on modern sparse matrices processing
techniques, as well as exploits some optimizations, related to the boolean data processing.