\section{Introduction}
\textbf{Фокус статьи.}
Во введении важно ввести читателя в курс дела, объяснить что такое CFPQ, рассказать про известные работы.
Рассказать, что в каждой статье был свой кустарный подход к экспериментальному анализу алгоритмов и зачастую из-за этого не получается сопоставить алгоритмы по результатам.

\textbf{Параграф первый - что такое CFPQ. Нужно переписать другими словами.}

Formal language constrained path querying, or formal language constrained path problem~\cite{barrett2000formal}, is a graph analysis problem in which formal languages are used as constraints for navigational path queries.
In this approach, a path is viewed as a word constructed by concatenation of edge labels.
Paths of interest are constrained with some formal language: a query should find only paths labeled by words from the language.
The class of language constraints which is most widely spread is regular: it is used in various graph query languages and engines.
Context-free path querying (CFPQ)~\cite{Yannakakis}, while being more expressive, is still at the early stage of development.
Context-free constraints allow one to express such important class of queries as \textit{same-generation queries}~\cite{FndDB} which cannot be expressed in terms of regular constraints.

\textbf{Параграф второй - работы в области CFPQ.}

Several algorithms for CFPQ based on such parsing techniques as (G)LL, (G)LR, and CYK were proposed recently~\cite{bradford2007quickest,ward2008distributed,bradford2016fast,hellingsPathQuerying,Grigorev:2017:CPQ:3166094.3166104,Verbitskaia:2018:PCC:3241653.3241655,RDF,10.1007/978-3-319-91662-0_17,Medeiros:2018:EEC:3167132.3167265}.
Yet recent research by Jochem Kuijpers et al.~\cite{Kuijpers:2019:ESC:3335783.3335791} shows that existing solutions are not applicable for real-world graph analysis because of significant running time and memory consumption.
At the same time, Nikita Mishin et al. show in~\cite{Mishin:2019:ECP:3327964.3328503} that the matrix-based CFPQ algorithm demonstrates good performance on real-world data.
A matrix-based algorithm proposed by Rustam Azimov~\cite{Azimov:2018:CPQ:3210259.3210264} offloads the most critical computations onto Boolean matrices multiplication.

All discussed matrix-based algorithms correspond to the CFPQ with relational query semantics and solve the reachability problem (according to Hellings~\cite{hellingsRelational}). 
However, in some areas, it is important to have a proof of existence of certain paths. 
This problem can be solved using CFPQ algorithms with single-path query semantics (according to Hellings~\cite{hellingsPathQuerying}), which provide some path for each node pair if one exists. 
There are many results on the CFPQ with single-path query semantics which use the shortest paths to return~\cite{hellingsPathQuerying,barrett2000formal,bradford2007quickest,ward2010complexity}.

\textbf{Параграф третий - почему нужно стандартизировать экспериментальное исследование.}

Due to the wide applicability of context-free path queries, the need to measure the performance of algorithms that implement these queries becomes critical. 
In order to show the applicability of the CFPQ algorithm in practice, it becomes necessary to conduct an experimental analysis on data simulating real scenarios. 
It also allows researchers to compare the performance of their proposed solution with existing ones.
However, finding and preparing the necessary data for conducting an experimental study can cause many different problems.
One solution to these problems in many areas of research is the use of a standardized dataset.
Unfortunately, in the field of CFPQ algorithms, at the moment in most works, even recent ones, experiments are carried out on a fixed set of small-scale, non-diverse graphs, using non-standardized experimental protocols and baselines, which makes it difficult to compare results from different publications.

\subsection{Present work}
\textbf{Поверхностный обзор того, что есть в CFPQ\_Data.}
The collection consists of over 40 graphs of varying sizes.
All graphs are presented in standard RDF format on the collection site \href{https://github.com/JetBrains-Research/CFPQ_Data}{https://github.com/JetBrains-Research/CFPQ\_Data}.

\textbf{RDF.} Где-то нужно написать почему именно RDF используется. 

The Resource Description Framework (RDF) is a family of World Wide Web Consortium (W3C) specifications~\cite{semanticweb} originally designed as a metadata data model.
RDF has features that facilitate data merging even if the underlying schemas differ, and it specifically supports the evolution of schemas over time without requiring all the data consumers to be changed. 
RDF extends the linking structure of the Web to use URIs to name the relationship between things as well as the two ends of the link (this is usually referred to as a “triple”). 
Using this simple model, it allows structured and semi-structured data to be mixed, exposed, and shared across different applications. 
This linking structure forms a directed, labeled graph, where the edges represent the named link between two resources, represented by the graph nodes. 
This graph view is the easiest possible mental model for RDF and is often used in easy-to-understand visual explanations.
We provide Python-based data loaders and implementations of the well-known CFPQ algorithms.
We also give standardized evaluation procedures, and provide baseline experiments.
Moreover, we report results on an experimental study comparing well-known algorithms in CFPQ area.

\subsection{Related work}
\textbf{Обзор существующих работ.} И того, как в этих работах проводились эксперименты.

Как проводились эксперименты by Xiaowang Zhang in "Context-Free Path Queries on RDF Graphs"~\cite{RDF}.
Как проводились эксперименты by Jochem Kuijpers et al.~\cite{Kuijpers:2019:ESC:3335783.3335791}.
Как проводились эксперименты by Nikita Mishin et al. show in~\cite{Mishin:2019:ECP:3327964.3328503}.
Как проводились эксперименты by Rustam Azimov~\cite{Azimov:2018:CPQ:3210259.3210264}. The paper measures the performance of the algorithm in isolation while J.~Kuijpers provides the evaluation of the algorithms which are integrated with Neo4j\footnote{Neo4j graph database web page: \url{https://neo4j.com/}. Access date: 15-March-2021.} graph database.

Указать на то, что они все проводились на разных данных и физически не сравнить алгоритмы между собой (если это так).
Указать на то, что все данные для экспериментов в каждой из статей были сфокусированы только на одной области (если это так).