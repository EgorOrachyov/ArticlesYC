%%
%% The "author" command and its associated commands are used to define
%% the authors and their affiliations.
%% Of note is the shared affiliation of the first two authors, and the
%% "authornote" and "authornotemark" commands
%% used to denote shared contribution to the research.
\author{Author 1}
\authornote{Both authors contributed equally to this research.}
\email{author_1@samplemail.com}
\orcid{1234-5678-9012}
\author{Author 2}
\authornotemark[1]
\email{author_2@samplemail.com}
\affiliation{%
  \institution{institution}
  \streetaddress{streetaddress}
  \city{city}
  \state{state}
  \country{country}
  \postcode{43017-6221}
}

\author{Author 3}
\email{author_3@samplemail.com}
\orcid{1234-5678-9012}
\affiliation{%
  \institution{institution}
  \streetaddress{streetaddress}
  \city{city}
  \state{state}
  \country{country}
  \postcode{43017-6221}
}

%%
%% By default, the full list of authors will be used in the page
%% headers. Often, this list is too long, and will overlap
%% other information printed in the page headers. This command allows
%% the author to define a more concise list
%% of authors' names for this purpose.
% \renewcommand{\shortauthors}{Trovato and Tobin, et al.}