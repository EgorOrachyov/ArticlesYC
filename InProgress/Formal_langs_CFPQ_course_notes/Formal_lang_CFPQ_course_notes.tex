%\documentclass[a4paper,12pt]{article}  % standard LaTeX, 12 point type
\documentclass[12pt, a4paper, table]{book}

\usepackage{algpseudocode}
\usepackage{algorithm}
\usepackage{algorithmicx}

\usepackage{geometry}
\usepackage{amsfonts,latexsym}
\usepackage{amsthm}
\usepackage{amssymb}
\usepackage[utf8]{inputenc} % Кодировка
\usepackage[english,russian]{babel} % Многоязычность
\usepackage{mathtools}
\usepackage{hyperref}
\usepackage{tikz}
\usepackage{dsfont}
\usepackage{multicol}
\usepackage[bb=boondox]{mathalfa}

\usetikzlibrary{fit,calc,automata,positioning}

\theoremstyle{definition}
\newtheorem{definition}{Определение}[section]
\newtheorem{example}{Пример}[section]
\newtheorem{theorem}{Теорема}[section]
\newtheorem{proposition}[theorem]{Proposition}
\newtheorem{lemma}[theorem]{Лемма}
\newtheorem{corollary}[theorem]{Corollary}
\newtheorem{conjecture}[theorem]{Conjecture}
\newtheorem{note}[theorem]{Утверждение}


% unnumbered environments:

\theoremstyle{remark}
\newtheorem*{remark}{Remark}
%\newtheorem*{notation}{Notation}

\setlength{\parskip}{5pt plus 2pt minus 1pt}
%\setlength{\parindent}{0pt}


\algtext*{EndWhile}% Remove "end while" text
\algtext*{EndIf}% Remove "end if" text
\algtext*{EndFor}% Remove "end for" text
\algtext*{EndFunction}% Remove "end function" text


\usepackage{color}
\usepackage{listings}
\usepackage{caption}
\usepackage{graphicx}
\usepackage{ucs}

\graphicspath{{pics/}}

%\geometry{left=2cm}
%\geometry{right=1.5cm}
%\geometry{top=2cm}
%\geometry{bottom=2cm}




%\lstnewenvironment{algorithm}[1][]
%{
%    \lstset{
%        frame=tB,
%        numbers=left,
%        mathescape=true,
%        numberstyle=\small,
%        basicstyle=\small,
%        inputencoding=utf8,
%        extendedchars=\true,
%        keywordstyle=\color{black}\bfseries,
%        keywords={,function, procedure, return, datatype, function, in, if, else, for, foreach, while, denote, do, and, then, assert,}
%        numbers=left,
%        xleftmargin=.04\textwidth,
%        #1 % this is to add specific settings to an usage of this environment (for instnce, the caption and referable label)
%    }
%}
%{}

\newcommand{\tab}[1][0.3cm]{\ensuremath{\hspace*{#1}}}

\newcommand{\rvline}{\hspace*{-\arraycolsep}\vline\hspace*{-\arraycolsep}}

\newcommand{\derives}[1][*]{\xRightarrow[]{#1}}
\newcommand{\first}[1][1]{\textsc{first}_{#1}}
\newcommand{\follow}[1][1]{\textsc{follow}_{#1}}

\setcounter{MaxMatrixCols}{20}


\tikzset{
%->, % makes the edges directed
%>=stealth’, % makes the arrow heads bold
node distance=4cm, % specifies the minimum distance between two nodes. Change if necessary.
%every state/.style={thick, fill=gray!10}, % sets the properties for each ’state’ node
initial text=$ $, % sets the text that appears on the start arrow
}

\tikzstyle{symbol_node} = [shape=rectangle, rounded corners, draw, align=center]

\tikzstyle{r_state} = [shape=rectangle, draw, minimum size=0.2cm]

\tikzstyle{prod_node} = [shape=rectangle, draw, align=center]

\tikzset{
    between/.style args={#1 and #2}{
         at = ($(#1)!0.5!(#2)$)
    }
}

%every node/.style = {shape=rectangle, rounded corners,
%      draw, align=center,
%      top color=white, bottom color=blue!20}

\newcommand{\bfgray}[1]{\cellcolor{lightgray}\textbf{#1}}

\newenvironment{scaledalign}[4]
  {
    \begingroup
    #1
    \setlength\arraycolsep{#2}
    \renewcommand{\arraystretch}{#3}
    \begin{center}
    \begin{equation}
    \begin{aligned}
    #4
  }
  {
    \end{aligned}
    \end{equation}
    \end{center}
    \endgroup
  }

\title{Приложения теории формальных языков и синтаксического анализа}
\author{Семён Григорьев}
\date{\today}

\begin{document}
\maketitle
\newpage
\tableofcontents
\newpage

\input{List_of_contributors}
\chapter*{Введение}

Теория формальных языков находит применение не только для ставших уже классическими задач синтаксического анализа кода (языков программирования, искусственных языков) и естественных языков, но и в других областях, таких как статический анализ кода, графовые базы данных, биоинформатика, машинное обучение.

Например, в машинном обучении использование формальных грамматик позволяет передать искусственной нейронной сети, предназначенной для генерации цепочек с определёнными свойствами (генеративной нейронной сети), знания о синтаксической структуре этих цепочек, что позволяет существенно упростить процесс обучения и повысить качество результата~\cite{10.5555/3305381.3305582}.
Вместе с этим, развиваются подходы, позволяющие нейронным сетям наоборот извлекать синтаксическую структуру (строить дерево вывода) для входных цепочек~\cite{kasai-etal-2017-tag,kasai-etal-2018-end}.

В биоинформатике формальные грамматики нашли широкое применение для описания особенностей вторичной структуры геномных и белковых последовательностей~\cite{Dyrka2019,WJAnderson2012,zier2013rna}.
Соответствующие алгоритмы синтаксического анализа используются при создании инструментов обработки данных.

Таким образом, теория формальных языков выступает в качестве основы для многих прикладных областей, а алгоритмы синтаксического анлиза применимы не только для обработки естественных языков или языков программирования.
Нас же в данной работе будет интересовать применение теории формальных языков и алгоритмов синтаксического анализа для анализа графовых баз данных и для статического анализа кода.

Одна из классических задач, связанных с анализом графов --- это поиск путей в графе.
Возможны различные формулировки этой задачи.
В некоторых случаях необходимо выяснить, существует ли путь с определёнными свойствами между двумя выбранными вершинами.
В других же ситуациях необходимо найти все пути в графе, удовлетворяющие некоторым свойствам или ограничениям. 
Например, в качестве ограничений можно указать, что искомый путь должен быть простым, кратчайшим, гамильтоновым и так далее.

Один из способов задавать ограничения на пути в графе основан на использовании формальных языков.
Базовое определение языка говорит нам, что язык --- это множество слов над некоторым алфавитом.
Если рассмотреть граф, рёбра которого помечены символами из алфавита, то путь в таком графе будет задавать слово: достаточно соединить последовательно символы, лежащие на рёбрах пути.
Множество же таких путей будет задавать множество слов или язык.
Таким образом, если мы хотим найти некоторое множество путей в графе, то в качестве ограничения можно описать язык, который должно задавать это множество.
Иными словами, задача поиска путей может быть сформулирована следующим образом: необходимо найти такие пути в графе, что слова, получаемые конкатенацей меток их рёбер, принадлежат заданному языку.
Такой класс задач будем называть задачами поиска путей с ограничениям в терминах формальных языков.

Подобный класс задач часто возникает в областях, связанных с анализом граф-структурированных данных и активно исследуется~\cite{doi:10.1137/S0097539798337716,axelsson2011formal,10.1007/978-3-642-22321-1_24,Ward:2010:CRL:1710158.1710234,barrett2007label,doi:10.1137/S0097539798337716}.
Исследуются как классы языков, применяемых для задания ограничений, так и различные постановки задачи.

Граф-структурированные данные встречаются не только в графовых базах данных, но и при статическом анализе кода: по программе можно построить различные графы отображающие её свойства.
Скажем, граф вызовов, граф потока данных и так далее.
Оказывается, что поиск путей в специального вида графах с использованием ограничений в терминах формальных языков позволяет исследовать некоторые свойства программы.
Например проводить межпроцедурный анализ указателей или анализ алиасов~\cite{Zheng,10.1145/2001420.2001440,10.1145/2714064.2660213}, строить срезы программ~\cite{10.1145/193173.195287}, проводить анализ типов~\cite{10.1145/373243.360208}.

В данной работе представлен ряд алгоритмов для поиска путей с ограничениями в терминах формальных языков.
Основной акцент будет сделан на контекстно-свободных языках, однако будут затронуты и другие классы: регулярные, многокомпонентные контекстно-свободные (Multiple Context-Free Languages, MCFL~\cite{!!!}) и конъюнктивные языки.
Будет показано, что теория формальных языков и алгоритмы синтаксического анализа применимы не только для анализа языков программирования или естественных языков, а также для анализа графовых баз данных и статического анализа кода, что приводит к возникновению новых задач и переосмыслению старых.


Структура данной работы такова.
Сперва, в главе~\ref{chpt:GraphTheoryIntro} мы рассмотрим основные понятия из теории графов, необходимые в данной работе.
Затем, в главе~\ref{chpt:FormalLanguageTheoryIntro} мы введём основные понятия из теории формальных языков.
Далее, в главе~\ref{chpt:CFPQ} рассмотрим различные варианты постановки задачи поиска путей с ограничениями в терминах формальнх языков, обсудим базовые свойства задач, её разрешимость в различных постановках и т.д..
И в итоге зафиксируем постановку, которую будем изучать далее.
После этого, в главах~\ref{chpt:CFPQ_CYK}--\ref{chpt:CFPQ_Derivatives} мы будем подробно рассматривать различные алгоритмы решения этой задачи, попутно вводя специфичные для рассматриваемого алгоритма структуры данных.
Большинство алгоритмов будут основаны на классических алгоритмах синтаксического анализа, таких как CYK или LR.
Все главы, начиная с~\ref{chpt:GraphTheoryIntro}, снабжены списком вопросов и задач для самостоятельного решения и закрепления материала.
\chapter{Некоторые понятия линейной алгебры}

При изложении некоторых алгоритмов будут активно использоваться некоторые понятия и инструмены линейной алгебры, такие как различные полукольца.
В данном разделе необходимые понятия будут определены и приведены некоторые примеры соответствующих конструкций.

$$
\oplus
\otimes
\mathbb{1}
\mathbb{0}
$$

\section{!!!}
Про коммутативность, ассоциативность, идемпотентность и другие свойства операций.
Стобы потом полукольца проще определять было.


\section{Полукольцо}
Понятие полукольца.

\section{Матрицы}
Про матричное произведение, тензорное произведение, ещё что-то.
\chapter{Общие сведения теории графов}\label{chpt:GraphTheoryIntro}

В данном разделе мы дадим определения базовым понятиям из теории графов, рассмотрим несколько классических задач из области анализа графов и алгоритмы их решения.
Всё это понадобится нам при последующей работе.

\section{Основные определения}

\begin{definition}
  \textit{Граф} $\mathcal{G} = \langle V, E, L \rangle$, где $V$ --- конечное множество вершин, $E$ --- конечное множество рёбер, т.ч. $E \subseteq V \times L \times V$, $L$ --- конечное множество меток на рёбрах.
\end{definition}

В дальнейшем речь будет идти о конечных ориентированных помеченных графах.
Мы будем использовать термин \textit{граф} подразумевая именно конечный ориентированный помеченный граф, если только не оговорено противное.

Также мы будем считать, что все вершины занумерованы подряд с нуля.
То есть можно считать, что $V$ --- это отрезок $[0, |V| - 1]$ неотрицательных целых чисел, где $|V|$ --- размер множества $V$.

\begin{example}[Пример графа и его графического представления]
  Пусть дан граф $$\mathcal{G}_1 = \langle \{0,1,2,3\}, \{(0,a,1), (1,a,2), (2,a,0), (2,b,3), (3,b,2)\}, \{a,b\} \rangle.$$
  Графическое представление графа $\mathcal{G}_1$:
  \begin{center}
  \begin{tikzpicture}[on grid, auto]
     \node[state] (q_0)   {$0$};
     \node[state] (q_1) [above right=1.4cm and 1.0cm of q_0] {$1$};
     \node[state] (q_2) [right=2.0cm of q_0] {$2$};
     \node[state] (q_3) [right=2.0cm of q_2] {$3$};
      \path[->]
      (q_0) edge  node {a} (q_1)
      (q_1) edge  node {a} (q_2)
      (q_2) edge  node {a} (q_0)
      (q_2) edge[bend left, above]  node {b} (q_3)
      (q_3) edge[bend left, below]  node {b} (q_2);
  \end{tikzpicture}
  \end{center}
\end{example}

\begin{definition}
  \textit{Ребро} ориентированного помеченного графа $\mathcal{G} = \langle V, E, L \rangle$ это упорядоченная тройка $e = (v_i,l,v_j) \in V \times L \times V$.
\end{definition}

\begin{example}[Пример рёбер графа]
$(0,a,1)$  и $(3,b,2)$ --- это рёбра графа $\mathcal{G}_1$. При этом, $(3,b,2)$ $(2,b,3)$ --- это разные рёбра, что видно из рисунка.
\end{example}

\begin{definition}
  \textit{Путём} $\pi$ в графе $\mathcal{G}$ будем называть последовательность рёбер такую, что для любых двух последовательных рёбер $e_1=(u_1,l_1,v_1)$ и $e_2=(u_2,l_2,v_2)$ в этой последовательности, конечная вершина первого ребра является начальной вершиной второго, то есть $v_1 = u_2$. Будем обозначать путь из вершины $v_0$ в вершину $v_n$ как $$v_0 \pi v_n = e_0,e_1, \dots, e_{n-1} = (v_0, l_0, v_1),(v_1,l_1,v_2),\dots,(v_{n-1},l_n,v_n).$$

\begin{center}
  \begin{tikzpicture}[on grid, auto]
     \node[state] (v_1)   {$v_1$};
     \node[state] (v_n) [right=2.0cm of v_1] {$v_n$};
      \path[->]
      (v_1) edge [out=45] node {$\pi$} (v_n);
  \end{tikzpicture}
  \end{center}
\end{definition}

\begin{example}[Пример путей графа]
$(0,a,1),(1,a,2) = 0\pi_1 2$  --- путь из вершины 0 в вершину 2 в графе $\mathcal{G}_1$.
При этом, $(0,a,1),(1,a,2),(2,b,3),(3,b,2) = 0\pi_2 2$ --- это тоже путь из вершины 0 в вершину 2 в графе $\mathcal{G}_1$, но он не равен $0\pi_1 2$.
\end{example}

Кроме того, нам потребуется отношение, отражающее факт существования пути между двумя вершинами.

\begin{definition}\label{def:reach}
  \textit{Отношение достижимости} в графе:
  $(v_i,v_j) \in P \iff \exists v_i \pi v_j$.
\end{definition}

Отметим, что рефлексивность этого отношения часто зависит от контекста.
В некоторых задачах по-умолчанию $(v_i,v_i) \notin P$, а чтобы это было верно, требуется явное наличие ребра-петли.

Один из способов задать граф --- это задать его \textit{матрицу смежности}.

\begin{definition}
  \textit{Матрица смежности} графа $\mathcal{G}=\langle V,E,L \rangle$ --- это квадратная матрица $M$ размера $n \times n$, где $|V| = n$ и ячейки которой содержат множества.
  При этом $l \in M[i,j] \iff \exists e = (i,l,j) \in E$.
\end{definition}

Заметим, что наше определение матрицы смежности отличается от классического, в котором матрица отражает лишь факт наличия хотя бы одного ребра и, соответственно, является булевой. То есть $M[i,j] = 1 \iff \exists e = (i,\_,j) \in E$.


Также можно встретить матрицы смежности в ячейках которых всё же хранится некоторая информация, однако, в единственном экземпляре. То есть запрещены параллельные рёбра.
Такой подход часто можно встретить в задачах о кратчайших путях: в этом случае в ячейке хранится расстояние между двумя вершинами.
При этом, так как в качестве весов часто рассматривают произвольные (в том числе отрицательные) числа, то в задачах о кратчайших путях отдельно вводят значение ``бесконечность'' для обозначения ситуации, когда между двумя вершинами нет пути или его длина ещё не известна.


\begin{example}[Пример матрицы смежности неориентированного графа]
  Неориентированный граф:
  \begin{center}
  \begin{tikzpicture}[on grid,auto]
     \node[state] (q_0)   {$0$};
     \node[state] (q_1) [above right = 1.4cm and 1cm of q_0] {$1$};
     \node[state] (q_2) [right = 2cm of q_0] {$2$};
     \node[state] (q_3) [right = 2cm of q_2] {$3$};
      \path[-]
      (q_0) edge  node {} (q_1)
      (q_1) edge  node {} (q_2)
      (q_2) edge  node {} (q_0)
      (q_2) edge  node {} (q_3);
  \end{tikzpicture}
  \end{center}

  И его матрица смежности:
  $$
  \begin{pmatrix}
    1 & 1 & 1 & 0 \\
    1 & 1 & 1 & 0 \\
    1 & 1 & 1 & 1 \\
    0 & 0 & 1 & 1
  \end{pmatrix}
  $$
\end{example}

\begin{example}[Пример матрицы смежности ориентированного графа]
  Ориентированный граф:
  \begin{center}
  \begin{tikzpicture}[shorten >=1pt,on grid,auto]
     \node[state] (q_0)   {$0$};
     \node[state] (q_1) [above right = 1.4cm and 1cm of q_0] {$1$};
     \node[state] (q_2) [right = 2cm of q_0] {$2$};
     \node[state] (q_3) [right = 2cm of q_2] {$3$};
      \path[->]
      (q_0) edge  node {} (q_1)
      (q_1) edge  node {} (q_2)
      (q_2) edge  node {} (q_0)
      (q_2) edge[bend left, above]  node {} (q_3)
      (q_3) edge[bend left, below]  node {} (q_2);
  \end{tikzpicture}
  \end{center}

  И его матрица смежности:
  $$
  \begin{pmatrix}
    1 & 1 & 0 & 0 \\
    0 & 1 & 1 & 0 \\
    1 & 0 & 1 & 1 \\
    0 & 0 & 1 & 1
  \end{pmatrix}
  $$
\end{example}

\begin{example}[Пример матрицы смежности помеченного графа]
  Помеченный граф:
  \begin{center}
  \begin{tikzpicture}[shorten >=1pt,on grid,auto]
     \node[state] (q_0)   {$0$};
     \node[state] (q_1) [above right = 1.4cm and 1cm of q_0] {$1$};
     \node[state] (q_2) [right = 2cm of q_0] {$2$};
     \node[state] (q_3) [right = 2cm of q_2] {$3$};
      \path[->]
      (q_0) edge  node {a} (q_1)
      (q_1) edge  node {a} (q_2)
      (q_2) edge  node {a} (q_0)
      (q_2) edge[bend left = 20]  node {a} (q_3)
      (q_2) edge[bend left = 60]  node {b} (q_3)
      (q_3) edge[bend left, below]  node {b} (q_2);
  \end{tikzpicture}
  \end{center}

  И его матрица смежности:
  $$
  \begin{pmatrix}
    \varnothing   & \{a\}       & \varnothing & \varnothing \\
    \varnothing   & \varnothing & \{a\}       & \varnothing \\
    \{a\}         & \varnothing & \varnothing & \{a,b\} \\
    \varnothing   & \varnothing & \{b\}       & \varnothing
  \end{pmatrix}
  $$
\end{example}

\begin{example}[Пример матрицы смежности взвешенного графа]
  Взвешенный граф для задачи о кратчайших путях:
  \begin{center}
  \begin{tikzpicture}[shorten >=1pt,on grid,auto]
     \node[state] (q_0)   {$0$};
     \node[state] (q_1) [above right = 1.4cm and 1cm of q_0] {$1$};
     \node[state] (q_2) [right = 2cm of q_0] {$2$};
     \node[state] (q_3) [right = 2cm of q_2] {$3$};
      \path[->]
      (q_0) edge  node {-1.4} (q_1)
      (q_1) edge  node {2.2} (q_2)
      (q_2) edge  node {0.5} (q_0)
      (q_2) edge[bend left, above]  node {1.85} (q_3)
      (q_3) edge[bend left, below]  node {-0.76} (q_2);
  \end{tikzpicture}
  \end{center}

  И его матрица смежности (для задачи о кратчайших путях):
  $$
  \begin{pmatrix}
    0 & -1.4 & \infty & \infty \\
    \infty & 0 & 2.2 & \infty \\
    0.5 & \infty & 0 & 1.85 \\
    \infty & \infty & -0.76 & 0
  \end{pmatrix}
  $$
\end{example}

Мы ввели лишь общие понятия.
Специальные понятия, необходимые для изложения конкретного материала, будут даны в соответствующих главах.

\section{Задачи поиска путей}

Одна из классических задач анализа графов --- это задача поиска путей между вершинами с различными ограничениями.

При этом, возможны различные постановки задачи.
С одной стороны, по тому, что именно мы хотим получить в качестве результата:
\begin{itemize}
\item Наличие хотя бы одного пути, удовлетворяющего ограничениям, в графе. В данном случае не важно, между какими вершинами существует путь, важно лишь наличие его в графе.
\item Наличие пути, удовлетворяющего ограничениям, между некоторыми вершинами: задача достижимости.
      При данной постановке задачи, нас интересует ответ на вопрос достижимости вершина $v_1$ из вершины $v_2$ по пути, удовлетворяющему ограничениям.
      Такая постановка требует лишь проверить существование пути, но не обязательно его предоставлять в явном виде.
\item Поиск одного пути, удовлетворяющего ограничениям: необходимо не только установить факт наличия пути, но и  предъявить его.
\item Поиск всех путей: необходимо предоставить все пути, удовлетворяющеие заданным ограничениям.
\end{itemize}

С другой стороны, задачи различаются ещё и по тому, как фиксируются множества стартовых и конечных вершин.
Здесь возможны следующие варианты:
\begin{itemize}
\item из одной вершины до всех,
\item между всеми парами вершин,
\item межу фиксированной парой вершин,
\item между двумя множествами вершин.
\end{itemize}

Стоит отметить, что последний вариант является самым общим и сотальные --- лишь его частные случаи. 
Однако этот вариант часто выделяют отдельно, подразумевая, что остальные, выделенные, варианты в него не включаются. В итоге мы можем сформулировать прямое произведение различных постановок задач о поиске путей, перебирая возможные варианты желаемого результата и фиксируя разные стартоыве и финальные множетсва.

Ограничение, имеющее важное прикладное значение, --- минимальность длины.
Иными словами, важная прикладная задача --- \textit{поиск кратчайших путей в графе (англ. APSP --- all-pairs shortest paths)}

Часто добавляют ограничения, что путь должен быть простым и другие.

\section{APSP и транзитивное замыкание графа}

Заметим, что отношение достижимости (\ref{def:reach}) является транзитивным.
Действительно, если существует путь из $v_i$ в $v_j$ и путь из $v_j$ в $v_k$, то существует путь из $v_i$ в $v_k$.

\begin{definition}
  \textit{Транзитивным замыканием графа} называется транзитивное замыкание отношения достижимости по всему графу.
\end{definition}

Как несложно заметить, транзитивное замыкание графа --- это тоже граф.
Более того, если решить задачу поиска кратчайших путей между всеми парами вершин, то мы построим транзитивное замыкание.
Поэтому сразу рассмотрим алгоритм Флойда-Уоршелла~\cite{Floyd1962, Bernard1959, Warshall1962}, который является общим алгоритмом поиска кратчайших путей (умеет обрабатывать рёбра с отрицательными весами, чего не может, например, алгоритм Дейкстры). Его сложность: $O(n^3)$, где $n$ --- количество вершин в графе.
При этом, данный алгоритм легко упростить до алгоритма построения транзитивного замыкания.

\begin{algorithm}
  \floatname{algorithm}{Listing}
\begin{algorithmic}[1]
\caption{Алгоритм Флойда-Уоршелла}
\label{lst:algoFloydWarxhall}
\Function{FloydWarshall}{$\mathcal{G}$}
    \State{$M \gets$ матрица смежности $\mathcal{G}$}
    \State{$n \gets$ $|V(\mathcal{G})|$}
    \For{k = 0; k < n; k++}
      \For{i = 0; i < n; i++}
        \For{j = 0; j < n; j++}
          \State{$M[i,j] \gets$ min$(M[i,j], M[i,k] + M[k,j])$}
        \EndFor
      \EndFor
    \EndFor
\State \Return $M$
\EndFunction
\end{algorithmic}
\end{algorithm}


\begin{example}
  Пусть дан следующий граф:
  \begin{center}
  \begin{tikzpicture}[shorten >=1pt,on grid,auto]
     \node[state] (q_0)   {$0$};
     \node[state] (q_1) [above right = 1.4cm and 1cm of q_0] {$1$};
     \node[state] (q_2) [right = 2cm of q_0] {$2$};
     \node[state] (q_3) [right = 2cm of q_2] {$3$};
      \path[->]
      (q_0) edge  node {} (q_1)
      (q_1) edge  node {} (q_2)
      (q_2) edge  node {} (q_0)
      (q_2) edge[bend left, above]  node {} (q_3)
      (q_3) edge[bend left, below]  node {} (q_2);
  \end{tikzpicture}
  \end{center}

  Построим его транзитивное замыкание:
  \begin{center}
  \begin{tikzpicture}[shorten >=1pt,on grid,auto]
     \node[state] (q_0)   {$0$};
     \node[state] (q_1) [above right = 1.4cm and 1cm of q_0] {$1$};
     \node[state] (q_2) [right = 2cm of q_0] {$2$};
     \node[state] (q_3) [right = 2cm of q_2] {$3$};
      \path[->]
      (q_0) edge[loop below] node {} ()
      (q_1) edge[loop above] node {} ()
      (q_2) edge[loop below] node {} ()
      (q_3) edge[loop below] node {} ()

      (q_0) edge  node {} (q_1)
      (q_1) edge[bend right] node {} (q_0)
      (q_1) edge  node {} (q_2)
      (q_2) edge[bend right] node {} (q_1)
      (q_2) edge  node {} (q_0)
      (q_0) edge[bend right] node {} (q_2)
      (q_2) edge[bend left, above]  node {} (q_3)
      (q_3) edge[bend left, below]  node {} (q_2)
      (q_0) edge[bend right = 60]  node {} (q_3)
      (q_1) edge[bend left, above]  node {} (q_3);
  \end{tikzpicture}
  \end{center}
  \begin{remark}
    На самом деле, петли у вершины 3 может и не быть, т.к. это зависит от определения.
  \end{remark}
\end{example}

\begin{remark}
Приведем список интересных работ по APSP для ориентированных графов с вещественными весами (здесь $n$ --- количество вершин в графе):
\begin{itemize}
    \item M.L. Fredman (1976) --- $O(n^3(\log \log n / \log n)^\frac{1}{3})$ --- использование дерева решений~\cite{FredmanAPSP1976}
    \item W. Dobosiewicz (1990) --- $O(n^3 / \sqrt{\log n})$ --- использование операций на Word Random Access Machine~\cite{Dobosiewicz1990}
    \item T. Takaoka (1992) --- $O(n^3 \sqrt{\log \log n / \log n})$ --- использование таблицы поиска~\cite{Takaoka1992}
    \item Y. Han (2004) --- $O(n^3 (\log \log n / \log n)^\frac{5}{7})$~\cite{Han2004}
    \item T. Takoaka (2004) --- $O(n^3 (\log \log n)^2 / \log n)$~\cite{Takaoka2004}
    \item T. Takoaka (2005) --- $O(n^3 \log \log n / \log n)$~\cite{Takaoka2005}
    \item U. Zwick (2004) --- $O(n^3 \sqrt{\log \log n} / \log n)$~\cite{Zwick2004}
    \item T.M. Chan (2006) --- $O(n^3 / \log n)$ --- многомерный принцип ``разделяй и властвуй''~\cite{Chan2008}
    \item и др.
\end{itemize}
\end{remark}

\section{APSP и произведение матриц}
Алгоритм Флойда-Уоршелла можно переформулировать в терминах перемножения матриц. Для этого введём обозначение.


\begin{definition}
Пусть даны матрицы $A$ и $B$ размера $n \times n$. Определим операцию $\star$, которую называют \textit{Min-plus matrix multiplication}:

    $A \star B = C$ --- матрица размера $n \times n$, т.ч.
    $C[i,j] = \min_{k = 1 \dots n} \{ A[i,k] + B[k,j] \}$
\end{definition}

Также, обозначим за $D[i,j](k)$ минимальную длину пути из вершины $i$ в $j$, содержащий максимум $k$ рёбер.
Таким образом, $D(1) = M$, где $M$ --- это матрица смежности, а решением APSP является $D(n-1)$, т.к. чтобы соединить $n$ вершин требуется не больше $n-1$ рёбер.

\begin{center}
    $D(1) = M$ \\
    $D(2) = D(1) \star M = M^2$ \\
    $D(3) = D(2) \star M = M^3$ \\
    $\dots$ \\
    $D(n-1) = D(n-2) \star M = M^{n-1}$ \\
\end{center}

Итак, мы можем решить APSP за $O(n K(n))$, где $K(n)$ --- сложность алгоритма умножения матриц.
Сразу заметим, что, например, $D(3)$ считать не обязательно, т.к. можем посчитать $D(4)$ как $D(2) \star D(2)$.
Поэтому:

\begin{center}
    $D(1) = M$ \\
    $D(2) = M^2 = M \star M$ \\
    $D(4) = M^4 = M^2 \star M^2$ \\
    $D(8) = M^8 = M^4 \star M^4$ \\
    $\dots$ \\
    $D(2^{\log(n-1)}) = M^{2^{\log(n-1)}} = M^{2^{\log(n-1)} - 1} \star M^{2^{\log(n-1)} - 1}$ \\
    $D(n-1) = D(2^{\log(n-1)})$ \\
\end{center}

Теперь вместо $n$ итераций нам нужно $\log{n}$. В итоге, сложность --- $O(\log{n} K(n))$.
Данный алгоритм называется \textit{repeated squaring}\footnote{Пример решения APSP с помощью repeated squaring: \url{http://users.cecs.anu.edu.au/~Alistair.Rendell/Teaching/apac_comp3600/module4/all_pairs_shortest_paths.xhtml}}.

\section{Умножение матриц с субкубической сложностью}
В предыдущем подразделе мы свели проблему APSP к проблеме min-plus matrix multiplication, поэтому взглянем на эффективные алгоритмы матричного умножения.

Сложность наивного произведения двух матриц составляет $O(n^3)$, это приемлемо только для малых матриц, для больших же лучше использовать алгоритмы с субкубической сложностью.
Отметим, что мы называем сложность субкубической, если она равна $O(n^{3-\varepsilon})$, где $\varepsilon > 0$, иначе говоря, меньшей, чем $O(n^3)$.

Первый субкубический алгоритм опубликовал Ф. Штрассен в 1969 году, его сложность --- $O(n^{\log_2 7}) \approx O(n^{2.81})$~\cite{Strassen1969}. Основная идея --- рекурсивное разбиение на блоки $2 \times 2$ и вычисление их произведения с помощью только 7 умножений, а не 8.
Впоследствии алгоритмы усовершенствовались до ${\approx} O(n^{2.5})$~\cite{Pan1978,BiniCapoRoma1979,Schonhage1981,CoppWino1982}. В настоящее время наиболее асимптотически быстрым является алгоритм Копперсмита --- Винограда со сложностью $O(n^{2.376})$~\cite{CoppWino1990}.

Несмотря на то, что у приведенных алгоритмов неплохая алгоритмическая сложность, мы всё же не можем использовать их для вычисления min-plus matrix multiplication, так как в субкубических алгоритмах требуется, чтобы элементы образовывали кольцо. Покажем, что $\mathbb{R} \cup \{+\infty\}$ с операциями min и + являются полукольцом, а не кольцом:
\begin{enumerate}
    \item $min(a, b) = min(b, a)$
    \item $min(min(a, b)), c) = min(a, min(b, c)))$
    \item $min(a, +\infty) = min(+\infty, a) = a$, т.е. $+\infty$ --- нейтральный элемент относительно операции min

    \item $(a + b) + c = a + (b + c)$

    \item $min(a, b) + c = min(a + c, b + c)$
    \item $a + min(b, c) = min(a + b, a + c)$

    \item $a + \infty = \infty + a = \infty$
    \item Но для произвольного элемента $a$: $\nexists d$, т.ч. $min(a, d) = min(d, a) = +\infty$, т.е. для любого элемента нет обратного относительно операции min
\end{enumerate}

Таким образом, вопрос о субкубических алгоритмах решения APSP всё ещё открыт~\cite{Chan2010}.
Кроме того, более простая задача APSP с булевыми матрицами также не решена за субкубическую сложность. Приведем обоснование этого факта.

\begin{definition}
  Матрица называется \textit{булевой}, если она состоит из 0 и 1.
\end{definition}

Булевы матрицы с поэлементными операциями дизъюнкции и конъюнкции являются полукольцом. Покажем это: пусть $A$, $B$ и $C$ --- булевы матрицы, тогда:
\begin{enumerate}
    \item $A \vee B = B \vee A$
    \item $(A \vee B) \vee C = A \vee (B \vee C)$
    \item $A \vee N = N \vee A = A$, где $N$ --- матрица, полностью состоящая из 0

    \item $(A \wedge B) \wedge C = A \wedge (B \wedge C)$

    \item $(A \vee B) \wedge C = (A \wedge C) \vee (B \wedge C)$
    \item $A \wedge (B \vee C) = (A \wedge B) \vee (A \wedge C)$

    \item $A \wedge N = N \wedge A = N$
\end{enumerate}

Булевы матрицы тоже не являются кольцом, т.к. не имеют обратный элемент относительно операции дизъюнкции (т.е. для произвольной булевой матрицы $A$: $\nexists D$, т.ч. $D$ --- булева матрица и $A \vee D = D \vee A = N$). Следовательно, субкубические алгоритмы не подходят для перемножения булевых матриц, т.к. в них используются обратные элементы (например, для разности). В частности, они не применимы к классической матрице смежности, которая ведёт себя как булева матрица.

\section{Вопросы и задачи}
\begin{enumerate}
  \item Реализуйте абстракцию полукльца, позволяющую конструировать полукольца с произвольными операциями.
  \item Реализуйте алгоритм произведения матриц над произвольным полукольцом. Используйте результат решения предыдущей задачи.
  \item Используя результаты предыдущих задач, реализуйте алгоритм построения транзитивного замыкания через произведение матриц.
  \item Используя результаты предыдущих задач, реализуйте алгоритм решения задачи APSP для ориентированного через произведение матриц.
  \item Используя существующую библиотеку линейной алгебры для CPU, решите задачу построения транзитивного замыкания графа. 
  \item Используя существующую библиотеку линейной алгебры для CPU, решите задачу APSP для ориентированного графа.
  \item Используя существующую библиотеку линейной алгебры для GPGPU, решите задачу построения транзитивного замыкания графа. 
  \item Используя существующую библиотеку линейной алгебры для GPGPU, решите задачу APSP для ориентированного графа.
  \item Сравните произволительность решений предыдущих задач
\end{enumerate}

\chapter{Общие сведения теории формальных языков}\label{chpt:FormalLanguageTheoryIntro}

В данной главе мы рассмотрим основные понятия из теории формальных языков, которые пригодятся нам в дальнейшем изложении.

\begin{definition}
\textit{Алфавит} --- это конечное множество.
Элементы этого множества будем называть \textit{символами}.
\end{definition}

\begin{example}
  Примеры алфавитов

  \begin{itemize}
    \item Латинский алфавит $\Sigma = \{ a, b, c, \dots, z\}$
    \item Кириллический алфавит $\Sigma = \{ \text{а, б, в, \dots, я}\}$
    \item Алфавит чисел в шестнадцатеричной записи 
    $$\Sigma = \{0, 1, 2, 3, 4, 5, 6, 7 ,8,9, A, B, C, D, E, F \}$$
  \end{itemize}
\end{example}

Традиционное обозначение для алфавита --- $\Sigma$.
Также мы будем использовать различные прописные буквы латинского алфавита. Для обозначения символов алфавита будем использовать строчные буквы латинского алфавита: $a, b, \dots, x, y, z$.

Будем считать, что над алфавитом $\Sigma$ всегда определена операция конкатенации $(\cdot): \Sigma^* \times \Sigma^* \to \Sigma^*$.
При записи выражений символ точки (обозначение операции конкатенации) часто будем опускать: $a \cdot b = ab$.

\begin{definition}
\textit{Слово} над алфавитом $\Sigma$ --- это конечная конкатенация символов алфавита $\Sigma$: $\omega = a_0 \cdot a_1 \cdot \ldots \cdot a_m$, где $\omega$ --- слово, а для любого $i$ $a_i \in \Sigma$.
\end{definition}

\begin{definition}
Пусть $\omega = a_0 \cdot a_1 \cdot \ldots \cdot a_m$ --- слово над алфавитом $\Sigma$.
Будем называть $m + 1$ \textit{длиной слова} и обозначать как $|\omega|$.
\end{definition}

\begin{definition}
\textit{Язык} над алфавитом $\Sigma$ --- это множество слов над алфавитом $\Sigma$.
\end{definition}

\begin{example}

Примеры языков.

  \begin{itemize}
    \item Язык целых чисел в двоичной записи $\{0, 1, -1, 10, 11, -10, -11, \dots\}.$
    \item Язык всех правильных скобочных последовательностей $$\{(), (()), ()(), (())(), \dots\}.$$
  \end{itemize}
\end{example}

Любой язык над алфавитом $\Sigma$ является подмножеством $\Sigma^*$ --- множества всех слов над алфавитом $\Sigma$.

Заметим, что язык не обязан быть конечным множеством, в то время как алфавит всегда конечен и изучаем мы конечные слова.

%\begin{definition}
\textit{Способы задания языков}
\begin{itemize}
\item Перечислить все элементы. Такой способ работает только для конечных языков. Перечислить бесконечное множество не получится.
\item Задать генератор --- процедуру, которая возвращает очередное слово языка.
\item Задать распознователь --- процедуру, которая по данному слову может определить, принадлежит оно заданному языку или нет.
\end{itemize}


Теоретико-множественные задачи над языками и их применение. 
О том, что моногое --- про пересечение, проверку пустоты, вложенность.





\section{Вопросы и задачи}
\begin{enumerate}
  \item !!! 
  \item !!!
\end{enumerate}

\chapter{Регулярные языки}

Регулярные языки, конечные автоматы, взяимные конвертации, замкнутость.

Лемма о накачке

Линейная алгебра для работы с регулярными языками: пересечение, замыкание.



\section{Задача поиска путей с ограничениями в терминах регулярных языков}

Графовая база данных --- автомат.

Задача --- пересечение автоматов.

Линейная алгебра, производные, построение атомата, выяснение существования путей.

\section{Вопросы и задачи}

Построить базу.

Научиться выполнять запросы через линейку. 
\chapter{Контекстно-свободные грамматики и языки}\label{CFG}

Из всего многообразия нас будут интересовать прежде всего контекстно-свободные грамматики.

\begin{definition}
\textit{Контекстно-свободная грамматика} --- это четвёрка вида $\langle \Sigma, N, P, S \rangle$, где
\begin{itemize}
  \item $\Sigma$ --- это терминальный алфавит;
  \item $N$ --- это нетерминальный алфавит;
  \item $P$ --- это множество правил или продукций, таких что каждая продукция имеет вид $N_i \to \alpha$, где $N_i \in N$ и $\alpha \in \{\Sigma \cup N\}^* \cup {\varepsilon}$;
  \item $S$ --- стартовый нетерминал.
  Отметим, что $\Sigma \cap N = \varnothing$.
\end{itemize}
\end{definition}

\begin{example}
Грамматика, задающая язык целых чисел в двоичной записи без лидирующих нулей: $G = \langle \{0, 1, -\}, \{S, N, A\}, P, S \rangle$, где $P$ определено следующим образом:

\[
\begin{array}{rcl}
S& \rightarrow & 0 \mid N \mid - N  \\
N& \rightarrow & 1 A \\
A& \rightarrow & 0 A \mid 1 A  \mid \varepsilon\\
\end{array}
\]
\end{example}

При спецификации грамматики часто опускают множество терминалов и нетерминалов, оставляя только множество правил. При этом нетерминалы часто обозначаются прописными латинскими буквами, терминалы --- строчными, а стартовый нетерминал обозначается буквой~$S$. Мы будем следовать этим обозначениям, если не указано иное.


\begin{definition}\label{def derivability in CFG}
  \textit{Отношение непосредственной выводимости}. Мы говорим, что последовательность терминалов и нетерминалов $\gamma \alpha \delta$ \textit{непосредственно выводится из} $\gamma \beta \delta$ \textit{при помощи правила} $\alpha \rightarrow \beta$ ($\gamma \alpha \delta \Rightarrow \gamma \beta \delta$), если
  \begin{itemize}
    \item $\alpha \rightarrow \beta \in P$
    \item $\gamma, \delta \in \{\Sigma \cup N\}^* \cup {\varepsilon}$
  \end{itemize}
\end{definition}

\begin{definition}
  \textit{Рефлексивно-транзитивное замыкание отношения} --- это наименьшее рефлексивное и транзитивное отношение, содержащее исходное.
\end{definition}

\begin{definition}
\textit{Отношение выводимости} является рефлексивно-транзитивным замыканием отношения непосредственной выводимости
\begin{itemize}
  \item $\alpha \derives \beta$ означает $\exists \gamma_0, \dots \gamma_k: \ \alpha \derives[] \gamma_0 \derives[] \gamma_1 \derives[] \dots \derives[] \gamma_{k-1} \derives[] \gamma_{k} \derives[] \beta$
  \item Транзитивность: $\forall \alpha, \beta, \gamma \in \{\Sigma \cup N\}^* \cup {\varepsilon}: \ \alpha \derives \beta, \beta \derives \gamma \Rightarrow \alpha \derives \gamma$
  \item Рефлексивность: $\forall \alpha \in \{\Sigma \cup N\}^* \cup {\varepsilon}: \ \alpha \derives \alpha$
  \item $\alpha \derives \beta$ --- $\alpha$ выводится из $\beta$
  \item $\alpha \derives[k] \beta$ --- $\alpha$ выводится из $\beta$ за $k$ шагов
  \item $\alpha \derives[+] \beta$ --- при выводе использовалось хотя бы одно правило грамматики
\end{itemize}
\end{definition}


\begin{example}
Пример вывода цепочки $-1101$ в грамматике:

  \[
  \begin{array}{rcl}
  S& \rightarrow & 0 \mid N \mid - N  \\
  N& \rightarrow & 1 A \\
  A& \rightarrow & 0 A \mid 1 A  \mid \varepsilon\\
  \end{array}
  \]

  \[ S \Rightarrow - N \Rightarrow - 1 A \Rightarrow - 1 1 A \derives - 1 1 0 1 A \Rightarrow - 1 1 0 1 \]
\end{example}


\begin{definition}[Вывод слова в грамматике]
Слово $\omega \in \Sigma^*$ \textit{выводимо в грамматике} $\langle \Sigma, N, P, S \rangle$, если существует некоторый вывод этого слова из начального нетерминала $S \derives \omega$.

\end{definition}

\begin{definition}
\textit{Левосторонний вывод}. На каждом шаге вывода заменяется самый левый нетерминал.
\end{definition}

\begin{example}
Пример левостороннего вывода цепочки в грамматике

  \[
    \begin{array}{rcl}
    S& \rightarrow & A A \mid s  \\
    A& \rightarrow & A A \mid B b \mid a \\
    B& \rightarrow & c \mid d
    \end{array}
  \]

  \[ \boldsymbol{S} \derives[] \boldsymbol{A} A \derives[] \boldsymbol{B} b A \derives[] c b \boldsymbol{A} \derives[] c b \boldsymbol{A} A \derives[] c b a \boldsymbol{A} \derives[] c b a a \]
\end{example}

Аналогично можно определить правосторонний вывод.

\begin{definition}
\textit{Язык, задаваемый грамматикой} --- множество строк, выводимых в грамматике $L(G) = \{ \omega \in \Sigma^* \mid S \derives \omega \}$.
\end{definition}

\begin{definition}
  Грамматики $G_1$ и $G_2$ называются \textit{эквивалентными}, если они задают один и тот же язык: $L(G_1) = L(G_2)$
\end{definition}


\begin{example}  Пример эквивалентных грамматик для языка целых чисел в двоичной системе счисления.

  \begin{tabular}{p{0.4\textwidth} | p{0.5\textwidth}}

    \[
      \begin{array}{rcl}
      \Sigma &=& \{ 0, 1, - \} \\
      N &=& \{ S, N, A \} \\~\\
      S& \rightarrow & 0 \mid N \mid - N  \\
      N& \rightarrow & 1 A \\
      A& \rightarrow & 0 A \mid 1 A  \mid \varepsilon\\
      \end{array}
    \]

    &

    \[
      \begin{array}{rcl}
      \Sigma &=& \{ 0, 1, - \} \\
      N &=& \{ S, A \} \\~\\
      S& \rightarrow & 0 \mid 1 A  \mid - 1 A  \\
      A& \rightarrow &  0 A \mid 1 A  \mid \varepsilon\\
      \end{array}
    \]
    \end{tabular}

\end{example}


\begin{definition}
  \textit{Неоднозначная грамматика} --- грамматика, в которой существует 2 и более левосторонних (правосторонних) выводов для одного слова.
\end{definition}

\begin{example}
  Неоднозначная грамматика для правильных скобочных последовательностей

\[
    S \to (S) \mid S S \mid \varepsilon
\]
\end{example}

\begin{definition}
  \textit{Однозначная грамматика} --- грамматика, в которой существует не более одного левостороннего (правостороннего) вывода для каждого слова.
\end{definition}

\begin{example}
  Однозначная грамматика для правильных скобочных последовательностей

\[
    S \to (S)S \mid \varepsilon
\]
\end{example}

\begin{definition}
  \textit{Существенно неоднозначные языки} --- языки, для которых невозможно построить однозначную грамматику.
\end{definition}

\begin{example}
  Пример существенно неоднозначного языка

\[\{a^n b^n c^m \mid n, m \in \mathds{Z}\} \cup \{a^n b^m c^m \mid n,m \in \mathds{Z}\}\]
\end{example}

\section{Дерево вывода}\label{sect:DerivTree}
В некоторых случаях не достаточно знать порядок применения правил.
Необходимо структурное представление вывода цепочки в грамматике.
Таким представлением является \textit{дерево вывода}.
\begin{definition}
Деревом вывода цепочки $\omega$ в грамматике $G=\langle \Sigma, N, S, P \rangle$ называется дерево, удовлетворяющее следующим свойствам.

\begin{enumerate}
  \item Помеченное: метка каждого внутреннего узла --- нетерминал, метка каждого листа --- терминал или $\varepsilon$.
  \item Корневое: корень помечен стартовым нетерминалом.
  \item Упорядоченное.
  \item В дереве может существовать узел с меткой $N_i$ и сыновьями $M_j \dots M_k$ только тогда, когда в грамматике есть правило вида $N_i \to M_j \dots M_k$.
  \item Крона образует исходную цепочку $\omega$.
\end{enumerate}
\end{definition}

\begin{example}
  Построим дерево вывода цепочки $ababab$ в грамматике

  \[ G = \langle \{a,b\}, \{S\}, S, \{S \to a \ S \ b \ S, S \to \varepsilon\} \rangle \]

\begin{center}

    \begin{tikzpicture}[sibling distance=4em,
    every node/.style = {shape=rectangle, rounded corners,
      draw, align=center,
      top color=white, bottom color=blue!20}]]
    \node {S}
      child { node {a} }
      child { node {S}
        child { node {$\varepsilon$}}
      }
      child { node {b} }
      child { node {S}
        child {node {a}}
        child { node {S}
          child { node {$\varepsilon$}}
        }
        child { node {b} }
        child { node {S}
          child {node {a}}
          child {node {S}
            child {node {$\varepsilon$}}
          }
          child {node {b}}
          child {node {S}
            child {node {$\varepsilon$}}
          }
        }
      };
  \end{tikzpicture}
\end{center}

\end{example}

\begin{theorem}
  Пусть $G = \langle \Sigma, N, P, S \rangle$ --- КС-грамматика.
  Вывод $S \derives \alpha$, где $\alpha \in (N \cup \Sigma)^*, \alpha \neq \varepsilon$ существует $\Leftrightarrow$ существует дерево вывода в грамматике $G$ с кроной $\alpha$.
\end{theorem}

\section{Пустота КС-языка}

\begin{theorem}
  Существует алгоритм, определяющий, является ли язык, порождаемый КС грамматикой, пустым.
\end{theorem}

\begin{proof}
  Следующая лемма утверждает, что если в КС языке есть выводимое слово, то существует другое выводимое слово с деревом вывода не глубже количества нетерминалов грамматики.
  Для доказательства теоремы достаточно привести алгоритм, последовательно строящий все деревья глубины не больше количества нетерминалов грамматики, и проверяющий, являются ли такие деревья деревьями вывода.
  Если в результате работы алгоритма не удалось построить ни одного дерева, то грамматика порождает пустой язык.
\end{proof}

\begin{lemma}
  Если в данной грамматике выводится некоторая цепочка, то существует цепочка, дерево вывода которой не содержит ветвей длиннее m, где m --- количество нетерминалов грамматики.
\end{lemma}

\begin{proof}
  Рассмотрим дерево вывода цепочки $\omega$. Если в нем есть 2 узла, соответствующих одному нетерминалу A, обозначим их $n_1$ и $n_2$.

  Предположим, $n_1$ расположен ближе к корню дерева, чем $n_2$.

  $S \derives \alpha A_{n_1} \beta \derives \alpha \omega_1 \beta; S \derives \alpha \gamma A_{n_2} \delta \beta \derives \alpha \gamma \omega_2 \delta \beta$, при этом $\omega_2$ является подцепочкой $\omega_1$.

  Заменим в изначальном дереве узел $n_1$ на $n_2$. Полученное дерево является деревом вывода $\alpha \omega_2 \delta$.

  Повторяем процесс замены одинаковых нетерминалов до тех пор, пока в дереве не останутся только уникальные нетерминалы.

  В полученном дереве не может быть ветвей длины большей, чем m.

  По построению оно является деревом вывода.
\end{proof}


\section{Нормальная форма Хомского}
\label{section:CNF}

\begin{definition}
Контекстно-свободная грамматика $\langle \Sigma, N, P, S\rangle$ находится в \textit{Нормальной Форме Хомского}, если она содержит только правила следующего вида:

\begin{itemize}
  \item $A \to B C \text{, где } A, B, C \in N \text{, S не содержится в правой части правила }$
  \item $A \to a \text{, где } A \in N, a \in \Sigma$
  \item $S \to \varepsilon$
\end{itemize}
\end{definition}

\begin{theorem}
Любую КС грамматику можно преобразовать в НФХ.
\end{theorem}

\begin{proof}
  Алгоритм преобразования в НФХ состоит из следующих шагов:

  \begin{itemize}
    \item Замена неодиночных терминалов
    \item Удаление длинных правил
    \item Удаление $\varepsilon$-правил
    \item Удаление цепных правил
    \item Удаление бесполезных нетерминалов
  \end{itemize}

  То, что каждый из этих шагов преобразует грамматику к эквивалентной, при этом является алгоритмом, доказано в следующих леммах.
\end{proof}

\begin{lemma}
  Для любой КС-грамматики можно построить эквивалентную, которая не содержит правила с неодиночными терминалами.
\end{lemma}

\begin{proof}
  Каждое правило $A \to B_0 B_1 \dots B_k, k \geq 1$ заменить на множество правил:
  \begin{itemize}
    \item $A \to C_0 C_1 \dots C_k$
    \item $\{ C_i \to B_i \mid B_i \in \Sigma, C_i \text{ --- новый нетерминал} \}$
  \end{itemize}
\end{proof}

\begin{lemma}
  Для любой КС-грамматики можно построить эквивалентную, которая не содержит правил длины больше 2.
\end{lemma}

\begin{proof}
  Каждое правило $A \to B_0 B_1 \dots B_k, k \geq 2$ заменить на множество правил:
  \begin{itemize}
    \item $A \to B_0 C_0$
    \item $C_0 \to B_1 C_1$
    \item $\dots$
    \item $C_{k-3} \to B_{k-2} C_{k-2}$
    \item $C_{k-2} \to B_{k-1} B_k$
  \end{itemize}
\end{proof}


\begin{lemma}
  Для любой КС-грамматики можно построить эквивалентную, не содержащую $\varepsilon$-правил.
\end{lemma}

\begin{proof}
  Определим $\varepsilon$-правила:
  \begin{itemize}
    \item $A \to \varepsilon$
    \item $A \to B_0 \dots B_k, \forall i: \ B_i$ --- $\varepsilon$-правило.
  \end{itemize}

  Каждое правило $A \to B_0 B_1 \dots B_k$ заменяем на множество правил, где каждое $\varepsilon$-правило удалено во всех возможных комбинациях.
\end{proof}

\begin{lemma}
  Можно удалить все цепные правила
\end{lemma}

\begin{proof}
  \textit{Цепное правило} --- правило вида $A \to B\text{, где } A, B \in N\\$.
  \textit{Цепная пара} --- упорядоченная пара $(A,B)$, в которой $A\derives B$, используя только цепные правила.
  
  Алгоритм:
  \begin{enumerate}
  \item Найти все цепные пары в грамматике $G$.
  Найти все цепные пары можно по индукции:
  Базис: $(A,A)$ --- цепная пара для любого нетерминала, так как $A\derives A$ за ноль шагов.
  Индукция: Если пара $(A,B_0)$ --- цепная, и есть правило $B_0 \to B_1$, то $(A,B_1)$ --- цепная пара.
  \item Для каждой цепной пары $(A,B)$ добавить в грамматику $G'$ все правила вида $A \to a$, где $B \to a$ --- нецепное правило из $G$.
  \item Удалить все цепные правила
\end{enumerate}
Пусть $G$ --- контекстно-свободная грамматика. $G'$ --- грамматика, полученная в результате применения алгоритма к $G$. Тогда $L(G)=L(G')$.
\end{proof}

\begin{definition}
Нетерминал $A$ называется \textit{порождающим}, если из него может быть выведена конечная терминальная цепочка. Иначе он называется \textit{непорождающим}.
\end{definition}

\begin{lemma}
  Можно удалить все бесполезные (непорождающие) нетерминалы
\end{lemma}

\begin{proof}
  После удаления из грамматики правил, содержащих непорождающие нетерминалы, язык не изменится, так как непорождающие нетерминалы по определению не могли участвовать в выводе какого-либо слова.
  
  Алгоритм нахождения порождающих нетерминалов:
  \begin{enumerate}
  \item Множество порождающих нетерминалов пустое.
  \item Найти правила, не содержащие нетерминалов в правых частях и добавить нетерминалы, встречающихся в левых частях таких правил, в множество.
  \item Если найдено такое правило, что все нетерминалы, стоящие в его правой части, уже входят в множество, то добавить в множество нетерминалы, стоящие в его левой части.
  \item Повторить предыдущий шаг, если множество порождающих нетерминалов изменилось.
\end{enumerate}
В результате получаем множество всех порождающих нетерминалов грамматики, а все нетерминалы, не попавшие в него, являются непорождающими. Их можно удалить.
\end{proof}

\begin{example}
  Приведем в Нормальную Форму Хомского однозначную грамматику правильных скобочных последовательностей: $S \to a S b S \mid \varepsilon$

  Первым шагом добавим новый нетерминал и сделаем его стартовым: 

  \begin{align*}
    S_0 &\to S  \\ 
    S   &\to a S b S \mid \varepsilon
  \end{align*}

  Заменим все терминалы на новые нетерминалы: 

  \begin{align*}
    S_0 &\to S \\ 
    S   &\to L S R S \mid \varepsilon \\ 
    L   &\to a \\ 
    R   &\to b
  \end{align*}

  Избавимся от длинных правил: 

  \begin{align*}
    S_0 &\to S \\ 
    S   &\to L S' \mid \varepsilon \\ 
    S'  &\to S S'' \\ 
    S'' &\to R S \\
    L   &\to a \\ 
    R   &\to b
  \end{align*}

  Избавимся от $\varepsilon$-продукций: 

  \begin{align*}
    S_0 &\to S \mid \varepsilon \\ 
    S   &\to L S' \\ 
    S'  &\to S'' \mid S S'' \\ 
    S'' &\to R   \mid R S \\
    L   &\to a \\ 
    R   &\to b
  \end{align*}

  Избавимся от цепных правил: 

  \begin{align*}
    S_0 &\to L S' \mid \varepsilon \\ 
    S   &\to L S' \\ 
    S'  &\to b \mid R S \mid S S'' \\ 
    S'' &\to b \mid R S \\
    L   &\to a \\ 
    R   &\to b
  \end{align*}
\end{example}

\begin{definition}\label{defn:wCNF}
Контекстно-свободная грамматика $\langle \Sigma, N, P, S\rangle$ находится в \textit{ослабленной Нормальной Форме Хомского}, если она содержит только правила следующего вида:

\begin{itemize}
  \item $A \to B C \text{, где } A, B, C \in N$
  \item $A \to a \text{, где } A \in N, a \in \Sigma$
  \item $A \to \varepsilon \text{, где } A \in N$
\end{itemize}

То есть ослабленная НФХ отличается от НФХ тем, что:
\begin{enumerate}
  \item $\varepsilon$ может выводиться из любого нетерминала
  \item $S$ может появляться в правых частях правил
\end{enumerate}
\end{definition}

\section{Лемма о накачке}

\begin{lemma}
Пусть $L$ --- контекстно-свободный язык над алфавитом $\Sigma$, тогда существует такое $n$, что для любого слова $\omega \in L$, $|\omega| \geq n$ найдутся слова $u,v,x,y,z\in \Sigma^*$, для которых верно: $uvxyz = \omega, vy\neq \varepsilon,|vxy|\leq n$ и для любого $k \geq 0$  $uv^kxy^kz \in L$.
\end{lemma}

Идея доказательства леммы о накачке.

\begin{enumerate}
    \item Для любого КС языка можно найти грамматику в нормальной форме Хомского.
    \item Очевидно, что если брать достаточно длинные цепочки, то в дереве вывода этих цепочек, на пути от корня к какому-то листу обязательно будет нетерминал, встречающийся минимум два раза. Если $m$ --- количество нетерминалов в НФХ, то длины $2^{m+1}$ должно хватить. Это и будет $n$ из леммы.
    \item Возьмём путь, на котором есть хотя бы дважды повторяется некоторый нетерминал. Скажем, это нетерминал  $N_1$. Пойдём от листа по этому пути. Найдём первое появление $N_1$. Цепочка, задаваемая поддеревом для этого узла --- это $x$ из леммы.
    \item Пойдём дальше и найдём второе появление $N_1$. Цепочка, задаваемая поддеревом для этого узла --- это $vxy$ из леммы.
    \item Теперь мы можем копировать кусок дерева между этими повторениями $N_1$ и таким образом накачивать исходную цепочку.
\end{enumerate}

Надо только проверить выполение ограничений на длины.

Нахождение разбиения и пример накачки продемонстрированы на рисунках~\ref{fig:pumping1} и~\ref{fig:pumping2}.

\begin{figure}
\centering
\includegraphics[width=0.5\textwidth]{pics/pumping_tree_1.pdf}
\caption{Разбиение цепочки для леммы о накачке}
\label{fig:pumping1}
\end{figure}

\begin{figure}
\centering
\includegraphics[width=0.5\textwidth]{pics/pumping_tree_2.pdf}
\caption{Пример накачки цепочки с рисунка~\ref{fig:pumping1}}
\label{fig:pumping2}
\end{figure}


Для примера предлагается проверить неконтекстно-свободность языка $L=\{a^nb^nc^n \mid n>0\}$.


\section{Замкнутость КС языков относительно операций}


\begin{theorem}
Контекстно-свободные языки замкнуты относительно следующих операций:
\begin{enumerate}
  \item Объединение: если $L_1$ и $L_2$ --- контекстно-свободные языки, то и $L_3 = L_1 \cup L_2$ --- контекстно-свободный.
  \item Конкатенация: если $L_1$ и $L_2$ --- контекстно-свободные языки, то и $L_3 = L_1 \cdot L_2$ --- контекстно-свободный.
  \item Замыкание Клини: если $L_1$ --- контекстно-свободный, то и $L_2 = \bigcup\limits_{i=0}^{\infty} L_1^i $ --- контекстно-свободный.
  \item Разворот: если $L_1$ --- контекстно-свободный, то и $L_2 = {L_1}^r$ --- контекстно-свободный.
  \item Пересечение с регулярными языками: если $L_1$ --- контекстно-свободный, а $L_2$ --- регулярный, то  $L_3 = L_1 \cap L_2$ --- контекстно-свободный.
  \item Разность с регулярными языками: если $L_1$ --- контекстно-свободный, а $L_2$ --- регулярный, то  $L_3 = L_1 \setminus L_2$ --- контекстно-свободный.
\end{enumerate}
\end{theorem}
Для доказательства пунктов 1--4 можно построить КС граммтику нового языка имея грамматики для исходных. 
Будем предполагать, что множества нетерминальных символов различных граммтик для исходных языков не пересекаются.
\begin{enumerate}
\item $G_1=\langle\Sigma_1,N_1,P_1,S_1\rangle$ --- граммтика для $L_1$, $G_1=\langle\Sigma_2,N_2,P_2,S_2\rangle$ --- граммтика для $L_2$, тогда $G_3=\langle\Sigma_1 \cup \Sigma_2, N_1 \cup N_2 \cup \{S_3\}, P_1 \cup P_2 \cup \{S_3 \to S_1 \mid S_2\} ,S_3\rangle$ --- граммтика для $L_3$. 

\item $G_1=\langle\Sigma_1,N_1,P_1,S_1\rangle$ --- граммтика для $L_1$, $G_1=\langle\Sigma_2,N_2,P_2,S_2\rangle$ --- граммтика для $L_2$, тогда $G_3=\langle\Sigma_1 \cup \Sigma_2, N_1 \cup N_2 \cup \{S_3\}, P_1 \cup P_2 \cup \{S_3 \to S_1 S_2\} ,S_3\rangle$ --- граммтика для $L_3$. 

\item $G_1=\langle\Sigma_1,N_1,P_1,S_1\rangle$ --- граммтика для $L_1$, тогда $G_2=\langle\Sigma_1, N_1 \cup \{S_2\}, P_1 \cup \{S_2 \to S_1 S_2\ \mid \varepsilon\}, S_2\rangle$ --- граммтика для $L_2$. 

\item $G_1=\langle\Sigma_1,N_1,P_1,S_1\rangle$ --- граммтика для $L_1$, тогда $G_2=\langle\Sigma_1, N_1, \{N^i \to \omega^R \mid N^i \to \omega \in P_1 \}, S_1\rangle$ --- граммтика для $L_2$. 
\end{enumerate}

Чтобы доказать замкнутость относительно пересечения с регулярными языками, построим по КС грамматике рекурсивный автомат $R_1$, по регулярному выражению --- детерминированный конечный автомат $R_2$, и построим их прямое произведение $R_3$.
Переходы по терминальным символам в новом автомате возможны тогда и только тогда, когда они возможны одновременно и в исходном рекурсивном автомате и в исходном конечном. 
За рекурсивные вызовы отвечает исходныа рекурсивный автомат. 
Значит цепочка принимается $R_3$ тогда и только тогда, когда она принимается одновременно $R_1$ и $R_2$: так как состояния $R_3$ --- это пары из состояния $R_1$ и $R_2$, то по трассе вычислений $R_3$ мы всегда можем построить трассу для $R_1$ и $R_2$ и наоборот.

Чтобы доказать замкнутость относительно разности с регулятным языком, достаточно вспомнить, что регулярные языки замкнуты относительно дополнения, и выразить разность через пересечение с дополнением: 
$$
L_1 \setminus L_2 = L_1 \cap \overline{L_2}
$$

\qed

\begin{theorem}
Контекстно-свободные языки не замкнуты относительно следующих операций:
\begin{enumerate}
  \item Пересечение: если $L_1$ и $L_2$ --- контекстно-свободные языки, то и $L_3 = L_1 \cap L_2$ --- не контекстно-свободный.
  \item Разность: если $L_1$ и $L_2$ --- контекстно-свободные языки, то и $L_3 = L_1 \setminus L_2$ --- не контекстно-свободный.
\end{enumerate}
\end{theorem}

Чтобы доказать незамкнутость относительно пресечения, рассмотрим языки $L_1 = \{a^n b^n c^k \mid n \geq 0, k \geq 0\}$ и $L_2 = \{a^k b^n c^n \mid n \geq 0, k \geq 0\}$.
Очевидно, что $L_1$ и $L_2$ --- контекстно-свободные языки.
Рассмотрим $L_3 = L_1 \cap L_2 = \{a^n b^n c^n \mid n \geq 0\}$. 
$L_3$ не является контекстно-свободным по лемме о накачке для контекстно-свободных языков.

Чтобы доказать незамкнутость относительно разности проделаем следующее.
\begin{enumerate}
\item Рассмотрим языки $L_4 = \{a^m b^n c^k \mid m \neq n, k \geq 0\}$ и $L_5 = \{a^m b^n c^k \mid n \neq k, m \geq 0\}$. 
Эти языки являются контекстно-свободными.
Это легко заметить, если знать, что язык $L'_4 = \{a^m b^n c^k \mid 0 \leq m < n, k \geq 0\}$ задаётся следующей граммтикой:
\begin{align*}
S \to & S c & T \to & a T b \\
S \to & T &   T \to & T b \\
      &   &   T \to & b. 
\end{align*} 

\item Рассмотрим язык $L_6 = \overline{L'_6} = \overline{\{a^n b^m c^k \mid n \geq 0, m \geq 0, k \geq 0\}}$. Данный язык является регулярным.

\item Рассмотрим язык $L_7 = L_4 \cup L_5 \cup L_6$ --- контектсно свободный, так как является объединением контекстно-свободных.

\item Рассмотрим $\overline{L_7} = \{a^n b^n c^n \mid n \geq 0\} = L_3$: $L_4$ и $L_5$ задают языки с правильным порядком символов, но неравным их количеством, $L_6$ задаёт язык с неправильным порядком символов. 
Из пердыдущего пункта мы знаем, что $L_3$  не является контекстно-свободным.

\end{enumerate}

\qed

\section{Вопросы и задачи}
\begin{enumerate}
  \item Постройте дерево вывода цепочки $w=aababb$ в грамматике $G=\langle\{a,b\},\{S\},\{S\rightarrow \varepsilon \ | \ a \ S \ b \ S \}, S \rangle$.
  \item Постройте все левосторонние выводы цепочки $w=ababab$ в грамматике $G=\langle\{a,b\},\{S\},\{S\rightarrow \varepsilon \ | \ a \ S \ b \ | S \ S\}, S \rangle$.
  \item Постройте все правосторонние выводы цепочки $w=ababab$ в грамматике $G=\langle\{a,b\},\{S\},\{S\rightarrow \varepsilon \ | \ a \ S \ b \ | S \ S\}, S \rangle$.
  \item \label{t1}Постройте все деревья вывода цепочки $w=ababab$ в грамматике $G=\langle\{a,b\},\{S\},\{S\rightarrow \varepsilon \ | \ a \ S \ b \ | S \ S\}, S \rangle$, соответствующие левосторонним выводам.
  \item \label{t2}Постройте все деревья вывода цепочки $w=ababab$ в грамматике $G=\langle\{a,b\},\{S\},\{S\rightarrow \varepsilon \ | \ a \ S \ b \ | S \ S\}, S \rangle$, соответствующие правосторонним выводам.
\end{enumerate}

\input{CFPQ}
\chapter{CYK для вычисления КС запросов}\label{chpt:CFPQ_CYK}

В данной главе мы рассмотрим алгоритм CYK, позволяющий установить принадлежность слова грамматике и предоставить его вывод, если таковой имеется.

Наш главный интерес заключается в возможности применения данного алгоритма для решения описанной в предыдущей главе задачи --- поиска путей с ограничениями в терминах формальных языков. Как уже было указано выше, будем рассматривать случай контекстно-свободных языков.

\section{Алгоритм CYK}\label{sect:lin_CYK}

Алгоритм CYK (Cocke-Younger-Kasami) --- один из классических алгоритмов синтаксического анализа. Его асимптотическая сложность в худшем случае --- $O(n^3 * |N|)$, где $n$ --- длина входной строки, а $N$ --- количество нетерминалов во входной граммтике~\cite{Hopcroft+Ullman/79/Introduction}. 

Для его применения необходимо, чтобы подаваемая на вход грамматика находилась в Нормальной Форме Хомского (НФХ)~\ref{section:CNF}. Других ограничений нет и, следовательно,данный алгоритм применим для работы с произвольными контекстно-своболными языками.

В основе алгоритма лежит принцип динамического программирования. Используются два соображения:

\begin{enumerate}
\item Для правила вида $A \to a$ очевидно, что из $A$ выводится $\omega$ (с применением этого правила) тогда и только тогда, когда $a = \omega$:

\[
  A \derives \omega \iff \omega = a
\]

\item Для правила вида $A \to B C$ понятно, что из $A$ выводится $\omega$ (с применением этого правила) тогда и только тогда, когда существуют две цепочки $\omega_1$ и $\omega_2$ такие, что $\omega_1$ выводима из $B$, $\omega_2$ выводима из $C$ и при этом $\omega = \omega_1 \omega_2$:

\[
A \derives[] B C \derives \omega \iff \exists \omega_1, \omega_2 : \omega = \omega_1 \omega_2, B \derives \omega_1, C \derives \omega_2
\]

Или в терминах позиций в строке:

\[
A \derives[] B C \derives \omega \iff \exists k \in [1 \dots |\omega|] : B \derives \omega[1 \dots k], C \derives \omega[k+1 \dots |\omega|]
\]
\end{enumerate}

В процессе работы алгоритма заполняется булева трехмерная матрица $M$ размера $n \times n \times  |N|$ таким образом, что $$M[i, j, A] = true \iff A \derives \omega[i \dots j]$$.

Первым шагом инициализируем матрицу, заполнив значения $M[i, i, A]$:

\begin{itemize}
  \item $M[i, i, A] = true \text{, если в грамматике есть правило } A \to \omega[i]$.
  \item $M[i, i, A] = false$, иначе.
\end{itemize}

Далее используем динамику: на шаге $m > 1$ предполагаем, что ячейки матрицы $M[i', j', A]$ заполнены для всех нетерминалов $A$ и пар $i', j': j' - i' < m$.
Тогда можно заполнить ячейки матрицы $M[i, j, A] \text{, где } j - i = m$ следующим образом:

\[ M[i, j, A] = \bigvee_{A \to B C}^{}{\bigvee_{k=i}^{j-1}{M[i, k, B] \wedge M[k, j, C]}} \]

По итогу работы алгоритма значение в ячейке $M[0, |\omega|, S]$, где $S$ --- стартовый нетерминал грамматики, отвечает на вопрос о выводимости цепочки $\omega$ в грамматике.

\begin{example}\label{exampl:CYK}
  Рассмотрим пример работы алгоритма CYK на грамматике правильных скобочных последовательностей в Нормальной Форме Хомского.


\begin{align*}
S &\to A S_2 \mid \varepsilon & S_2  &\to b \mid B S_1 \mid S_1 S_3   & A   &\to a \\
S_1   &\to A S_2              & S_3  &\to b \mid B S_1              & B   &\to b\\      
\end{align*}

Проверим выводимость цепочки $\omega = a a b b a b$.

Так как трехмерные матрицы рисовать на двумерной бумаге не очень удобно, мы будем иллюстрировать работу алгоритма двумерными матрицами размера $n \times n$, где в ячейках указано множество нетерминалов, выводящих соответствующую подстроку.

Шаг 1. Инициализируем матрицу элементами на главной диагонали:

\[
\begin{pmatrix}
\{A\}       & \varnothing & \varnothing    & \varnothing      & \varnothing & \varnothing    \\
\varnothing & \{A\}       & \varnothing    & \varnothing      & \varnothing & \varnothing    \\
\varnothing & \varnothing & \{B, S_2, S_3\} & \varnothing     & \varnothing & \varnothing    \\
\varnothing & \varnothing & \varnothing    & \{B, S_2, S_3\}   & \varnothing & \varnothing   \\
\varnothing & \varnothing & \varnothing    & \varnothing      & \{A\}       & \varnothing    \\
\varnothing & \varnothing & \varnothing    & \varnothing      & \varnothing & \{B, S_2, S_3\} \\
\end{pmatrix}
\]

Шаг 2. Заполняем диагональ, находящуюся над главной:

\[
\begin{pmatrix}
\{A\}       & \varnothing & \varnothing                             & \varnothing      & \varnothing & \varnothing    \\
\varnothing & \{A\}       & \cellcolor{lightgray}\{S_1\}            & \varnothing      & \varnothing & \varnothing    \\
\varnothing & \varnothing & \{B, S_2, S_3\} & \varnothing     & \varnothing & \varnothing    \\
\varnothing & \varnothing & \varnothing    & \{B, S_2, S_3\}   & \varnothing & \varnothing   \\
\varnothing & \varnothing & \varnothing    & \varnothing      & \{A\}       & \cellcolor{lightgray}\{S_1\}            \\
\varnothing & \varnothing & \varnothing    & \varnothing      & \varnothing & \{B, S_2, S_3\} \\
\end{pmatrix}
\]

В двух ячейках появилисб нетерминалы $S_1$ благодяря присутствиб в грамматике правила $S_1 \to A S_2$.

Шаг 3. Заполняем следующую диагональ:

\[
\begin{pmatrix}
\{A\}       & \varnothing & \varnothing    & \varnothing      & \varnothing & \varnothing    \\
\varnothing & \{A\}       & \{S_1\}        & \cellcolor{lightgray}\{S_2\}          & \varnothing & \varnothing    \\
\varnothing & \varnothing & \{B, S_2, S_3\} & \varnothing     & \varnothing & \varnothing    \\
\varnothing & \varnothing & \varnothing    & \{B, S_2, S_3\}   & \varnothing & \cellcolor{lightgray}\{S_2, S_3\}  \\
\varnothing & \varnothing & \varnothing    & \varnothing      & \{A\}       & \{S_1\}            \\
\varnothing & \varnothing & \varnothing    & \varnothing      & \varnothing & \{B, S_2, S_3\} \\
\end{pmatrix}
\]

Шаг 4. И следующую за ней:

\[
\begin{pmatrix}
\{A\}       & \varnothing & \varnothing    & \cellcolor{lightgray}\{S_1, S\}       & \varnothing & \varnothing    \\
\varnothing & \{A\}       & \{S_1\}            & \{S_2\}          & \varnothing & \varnothing    \\
\varnothing & \varnothing & \{B, S_2, S_3\} & \varnothing     & \varnothing & \varnothing    \\
\varnothing & \varnothing & \varnothing    & \{B, S_2, S_3\}   & \varnothing & \{S_2, S_3\}  \\
\varnothing & \varnothing & \varnothing    & \varnothing      & \{A\}       & \{S_1\}            \\
\varnothing & \varnothing & \varnothing    & \varnothing      & \varnothing & \{B, S_2, S_3\} \\
\end{pmatrix}
\]

Шаг 5 Заполняем предпоследнюю диагональ:

\[
\begin{pmatrix}
\{A\}       & \varnothing & \varnothing    & \{S_1, S\}       & \varnothing & \varnothing    \\
\varnothing & \{A\}       & \{S_1\}            & \{S_2\}          & \varnothing & \cellcolor{lightgray}\{S_2\}        \\
\varnothing & \varnothing & \{B, S_2, S_3\} & \varnothing     & \varnothing & \varnothing    \\
\varnothing & \varnothing & \varnothing    & \{B, S_2, S_3\}   & \varnothing & \{S_2, S_3\}  \\
\varnothing & \varnothing & \varnothing    & \varnothing      & \{A\}       & \{S_1\}            \\
\varnothing & \varnothing & \varnothing    & \varnothing      & \varnothing & \{B, S_2, S_3\} \\
\end{pmatrix}
\]

\bigbreak
Шаг 6. Завершаем выполнение алгоритма:

\[
\begin{pmatrix}
\{A\}       & \varnothing & \varnothing    & \{S_1, S\}       & \varnothing & \cellcolor{lightgray}\{S_1, S\}     \\
\varnothing & \{A\}       & \{S_1\}            & \{S_2\}          & \varnothing & \{S_2\}        \\
\varnothing & \varnothing & \{B, S_2, S_3\} & \varnothing     & \varnothing & \varnothing    \\
\varnothing & \varnothing & \varnothing    & \{B, S_2, S_3\}   & \varnothing & \{S_2, S_3\}  \\
\varnothing & \varnothing & \varnothing    & \varnothing      & \{A\}       & \{S_1\}            \\
\varnothing & \varnothing & \varnothing    & \varnothing      & \varnothing & \{B, S_2, S_3\} \\
\end{pmatrix}
\]


Стартовый нетерминал находится в верхней правой ячейке, а значит цепочка $a a b b a b$ выводима в нашей грамматике.
\end{example}

\begin{example}
Теперь выполним алгоритм на цепочке $\omega=abaa$.

Шаг 1. Инициализируем таблицу:

\[
\begin{pmatrix}
\{A\}       & \varnothing    & \varnothing & \varnothing    \\
\varnothing & \{B, S_2, S_3\} & \varnothing & \varnothing       \\
\varnothing & \varnothing    & \{A\}       & \varnothing    \\
\varnothing & \varnothing    & \varnothing & \{A\}          \\
\end{pmatrix}
\]

Шаг 2. Заполняем следующую диагональ:

\[
\begin{pmatrix}
\{A\}       & \cellcolor{lightgray}\{S_1, S\}     & \varnothing & \varnothing    \\
\varnothing & \{B, S_2, S_3\} & \varnothing & \varnothing       \\
\varnothing & \varnothing    & \{A\}       & \varnothing    \\
\varnothing & \varnothing    & \varnothing & \{A\}          \\
\end{pmatrix}
\]

Больше ни одну ячейку в таблице заполнить нельзя и при этом стартовый нетерминал отсутствует в правой верхней ячейке, а значит эта строка не выводится в грамматике правильных скобочных последовательностей.

\end{example}

\section{Алгоритм для графов на основе CYK}
\label{graph:CYK}
Первым шагом на пути к решению задачи достижимости с использованием CYK является модификация представления входа. Прежде мы сопоставляли каждому символу слова его позицию во входной цепочке, поэтому при инициализации заполняли главную диагональ матрицы. Теперь вместо этого обозначим числами позиции между символами. В результате слово можно представить в виде линейного графа следующим образом(в качестве примера рассмотрим слово $a a b b a b$ из предыдущей главы~\ref{sect:lin_CYK}):

\begin{center}
    \begin{tikzpicture}[shorten >=1pt,on grid,auto]
    \node[state] (q_0) at (0,0)  {$0$};
    \node[state] (q_1) at (2,0)  {$1$};
    \node[state] (q_2) at (4,0)  {$2$};
    \node[state] (q_3) at (6,0)  {$3$};
    \node[state] (q_4) at (8,0)  {$4$};
    \node[state] (q_5) at (10,0) {$5$};
    \node[state] (q_6) at (12,0) {$6$};
    \path[->]
    (q_0) edge  node {$a$} (q_1)
    (q_1) edge  node {$a$} (q_2)
    (q_2) edge  node {$b$} (q_3)
    (q_3) edge  node {$b$} (q_4)
    (q_4) edge  node {$a$} (q_5)
    (q_5) edge  node {$b$} (q_6);
    \end{tikzpicture}
\end{center}

Что нужно изменить в описании алгоритма, чтобы он продолжал работать при подобной нумерации? Каждая буква теперь идентифицируется не одним числом, а парой --- номера слева и справа от нее. При этом чисел стало на одно больше, чем при прежнем способе нумерации.

Возьмем матрицу  $(n + 1) \times (n + 1) \times  |N|$ и при инициализации будем заполнять не главную диагональ, а диагональ прямо над ней. Таким образом, мы начинаем наш алгоритм с определения значений $M[i, j, A] \text{, где } j = i + 1$. При этом наши дальнейшие действия в рамках алгоритма не изменятся.

Для примера~\ref{exampl:CYK} на шаге инициализации матрица выглядит следующим образом:

\[
\begin{pmatrix}
\varnothing & \{A\}       & \varnothing & \varnothing    & \varnothing    & \varnothing & \varnothing    \\
\varnothing & \varnothing & \{A\}     & \varnothing    & \varnothing      & \varnothing & \varnothing    \\
\varnothing & \varnothing & \varnothing & \{B, S_2, S_3\} & \varnothing       & \varnothing & \varnothing    \\
\varnothing & \varnothing & \varnothing & \varnothing    & \{B, S_2, S_3\} & \varnothing & \varnothing   \\
\varnothing & \varnothing & \varnothing & \varnothing    & \varnothing    & \{A\}       & \varnothing    \\
\varnothing & \varnothing & \varnothing & \varnothing    & \varnothing    & \varnothing & \{B, S_2, S_3\} \\
\varnothing & \varnothing & \varnothing & \varnothing    & \varnothing    & \varnothing & \varnothing    \\

\end{pmatrix}
\]

А в результате работы алгоритма имеем:

\[
\begin{pmatrix}
\varnothing & \{A\}       & \varnothing & \varnothing    & \{S_1, S\}     & \varnothing & \{S_1, S\}     \\
\varnothing & \varnothing & \{A\}       & \{S_1\}        & \{S_2\}            & \varnothing & \{S_2\}        \\
\varnothing & \varnothing & \varnothing & \{B, S_2, S_3\} & \varnothing       & \varnothing & \varnothing    \\
\varnothing & \varnothing & \varnothing & \varnothing    & \{B, S_2, S_3\} & \varnothing & \{S_2, S_3\}  \\
\varnothing & \varnothing & \varnothing & \varnothing    & \varnothing    & \{A\}       & \{S_1\}            \\
\varnothing & \varnothing & \varnothing & \varnothing    & \varnothing    & \varnothing & \{B, S_2, S_3\} \\
\varnothing & \varnothing & \varnothing & \varnothing    & \varnothing    & \varnothing & \varnothing    \\
\end{pmatrix}
\]

Мы представили входную строку в виде линейного графа, а на шаге инициализации получили его матрицу смежности. Добавление нового нетерминала в язейку матрицы можно рассматривать как нахождение нового пути между соответствующими вершинами, выводимого из добавленного нетерминала. Таким образом, шаги алгоритма напоминают построение транзитивного замыкания графа. Различие заключается в том, что мы добавляем новые ребра только для тех пар нетерминалов, для которых существует соответстующее правило в грамматике.

Алгоритм можно обобщить и на произвольные графы с метками, рассматриваемые в этом курсе. При этом можно ослабить ограничение на форму входной грамматики: она должна находиться в ослабленной Нормальной Форме Хомского~(\ref{defn:wCNF}).

Шаг инициализации в алгоритме теперь состоит из двух пунктов.
\begin{itemize}
\item Как и раньше, с помощью продукций вида \[A \to a \text{, где } A \in N, a \in \Sigma\]
заменяем терминалы на ребрах входного графа на множества нетерминалов, из которых они выводятся.
\item  Добавляем в каждую вершину петлю, помеченную множеством нетерминалов для которых в данной граммтике есть правила вида $$A \to \varepsilon\text{, где } A \in N.$$ 
\end{itemize}

 Затем используем матрицу смежности получившегося графа (обозначим ее $M$) в качестве начального значения. Дальнейший ход алгоритма можно описать псевдокодом, представленным в листинге~\ref{alg:graphParseCYK}.

\begin{algorithm}[H]
    \begin{algorithmic}[1]
        \caption{Алгоритм КС достижимости на основе CYK}
        \label{alg:graphParseCYK}
        \Function{contextFreePathQuerying}{G, $\mathcal{G}$}

        \State{$n \gets$ the number of nodes in $\mathcal{G}$}
        \State{$M \gets$ the modified adjacency matrix of $\mathcal{G}$}
        \State{$P \gets$ the set of production rules in $G$}
        \While{$M$ is changing}
        \For {$k \in 0..n$}
            \For {$i \in 0..n$}
                \For {$j \in 0..n$}
                    \ForAll {$N_1 \in M[i, k]$, $N_2 \in M[k, j]$}
                        \If {$N_3 \to N_1 N_2 \in P$ }
                            \State{$M[i, j] \mathrel{+}= \{N_3\}$}
                        \EndIf
                    \EndFor
                \EndFor
            \EndFor
        \EndFor
        \EndWhile
        \State \Return $M$
        \EndFunction
    \end{algorithmic}
\end{algorithm}

После завершения алгоритма, если в некоторой ячейке результируюшей матрицы с номером $(i, j)$ находятся стартовый нетерминал, то это означает, что существует путь из вершины $i$ в вершину $j$, удовлетворяющий данной грамматике. Таким образом, полученная матрица является ответом для задачи достижимости для заданных графа и граммтики.

\begin{example}
\label{CYK_algorithm_ex}
Рассмотрим работу алгоритма на графе

\begin{center}
    \begin{tikzpicture}[node distance=3cm,shorten >=1pt,on grid,auto]
    \node[state] (q_0)   {$0$};
    \node[state] (q_1) [above right=of q_0] {$1$};
    \node[state] (q_2) [right=of q_0] {$2$};
    \node[state] (q_3) [right=of q_2] {$3$};
    \path[->]
    (q_0) edge  node {$a$} (q_1)
    (q_1) edge  node {$a$} (q_2)
    (q_2) edge  node {$a$} (q_0)
    (q_2) edge[bend left, above]  node {$b$} (q_3)
    (q_3) edge[bend left, below]  node {$b$} (q_2);
    \end{tikzpicture}
\end{center}

и грамматике:

\begin{align*}
S   & \to A B    & A   & \to a     \\
S   & \to A S_1  & B   & \to b\\
S_1 & \to S B   &&\\
\end{align*}

Данный пример является классическим и еще не раз будет использоваться в рамках данного курса. \\

\textbf{Инициализация.}
Заменяем терминалы на ребрах графа на нетерминалы, из которых они выводятся, и строим матрицу смежности получившегося графа:

\begin{center}
    \begin{tikzpicture}[node distance=3cm,shorten >=1pt,on grid,auto]
    \node[state] (q_0)   {$0$};
    \node[state] (q_1) [above right=of q_0] {$1$};
    \node[state] (q_2) [right=of q_0] {$2$};
    \node[state] (q_3) [right=of q_2] {$3$};
    \path[->]
    (q_0) edge  node {$\{A\}$} (q_1)
    (q_1) edge  node {$\{A\}$} (q_2)
    (q_2) edge  node {$\{A\}$} (q_0)
    (q_2) edge[bend left, above]  node {$\{B\}$} (q_3)
    (q_3) edge[bend left, below]  node {$\{B\}$} (q_2);
    \end{tikzpicture}
\end{center}

\[
\begin{pmatrix}
\varnothing & \{A\}       & \varnothing & \varnothing \\
\varnothing & \varnothing & \{A\}       & \varnothing \\
\{A\}       & \varnothing & \varnothing & \{B\}       \\
\varnothing & \varnothing & \{B\}       & \varnothing \\
\end{pmatrix}
\]

\textbf{Итерация 1.}
Итерируемся по $k$, $i$ и $j$, пытаясь найти пары нетерминалов, для которых существуют правила вывода, их выводящие. Нам интересны следующие случаи:

\begin{itemize}
    \item $k = 2, i = 1, j = 3: A \in M[1, 2], B \in M[2, 3]$, так как в грамматике присутствует правило $S \to A B$, добавляем нетерминал $S$ в ячейку $M[1, 3]$.
    \item $k = 3, i = 1, j = 2: S \in M[1, 3], B \in M[3, 2]$, поскольку в грамматике есть правило $S_1 \to S B$, добавляем нетерминал $S_1$ в ячейку $M[1, 2]$.
\end{itemize}

В остальных случаях либо какая-то из клеток пуста, либо не существует продукции в грамматике, выводящей данные два нетерминала.

Матрица после данной итерации:

\[
\begin{pmatrix}
\varnothing & \{A\}       & \varnothing & \varnothing \\
\varnothing & \varnothing & \cellcolor{lightgray}\{A, \boldsymbol{S_1}\}  & \cellcolor{lightgray}\{S\}       \\
\{A\}       & \varnothing & \varnothing & \{B\}       \\
\varnothing & \varnothing & \{B\}       & \varnothing \\
\end{pmatrix}
\]

\textbf{Итерация 2.}
Снова итерируемся по $k$, $i$, $j$. Рассмотрим случаи:

\begin{itemize}
    \setlength\itemsep{1em}
    \item $k = 1, i = 0, j = 2: A \in M[0, 1], S_1 \in M[1, 2]$, так как в грамматике присутствует правило $S \to A S_1$, добавляем нетерминал $S$ в ячейку $M[0, 2]$.
    \item $k = 2, i = 0, j = 3: S \in M[0, 2], B \in M[2, 3]$, поскольку в грамматике есть правило $S_1 \to S B$, добавляем нетерминал $S_1$ в ячейку $M[0, 3]$.
\end{itemize}

Матрица на данном шаге:

\[
\begin{pmatrix}
\varnothing & \{A\}       & \cellcolor{lightgray}\{S\}       & \cellcolor{lightgray}\{S_1\}     \\
\varnothing & \varnothing & \{A, S_1\}  & \{S\}       \\
\{A\}       & \varnothing & \varnothing & \{B\}       \\
\varnothing & \varnothing & \{B\}       & \varnothing \\
\end{pmatrix}
\]

\textbf{Итерация 3.}
Рассматриваемые на данном шаге случаи:

\begin{itemize}
    \setlength\itemsep{1em}
    \item $k = 0, i = 2, j = 3: A \in M[2, 0], S_1 \in M[0, 3]$, так как в грамматике присутствует правило $S \to A S_1$, добавляем нетерминал $S$ в ячейку $M[2, 3]$.
    \item $k = 3, i = 2, j = 2: S \in M[2, 3], B \in M[3, 2]$, поскольку в грамматике есть правило $S_1 \to S B$, добавляем нетерминал $S_1$ в ячейку $M[2, 2]$.
\end{itemize}

Матрица после этой итерации:

\[
\begin{pmatrix}
\varnothing & \{A\}       & \{S\}      & \{S_1\}     \\
\varnothing & \varnothing & \{A, S_1\} & \{S\}       \\
\{A\}       & \varnothing & \cellcolor{lightgray}\{S_1\}    & \cellcolor{lightgray}\{B, \boldsymbol{S}\}    \\
\varnothing & \varnothing & \{B\}      & \varnothing \\
\end{pmatrix}
\]

\textbf{Итерация 4.}
Рассмариваемые случаи:

\begin{itemize}
    \setlength\itemsep{1em}
    \item $k = 2, i = 1, j = 2: A \in M[1, 2], S_1 \in M[2, 2]$, так как в грамматике присутствует правило $S \to A S_1$, добавляем нетерминал $S$ в ячейку $M[1, 2]$.
    \item $k = 2, i = 1, j = 3: S \in M[1, 2], B \in M[2, 3]$, поскольку в грамматике есть правило $S_1 \to S B$, добавляем нетерминал $S_1$ в ячейку $M[1, 3]$.
\end{itemize}

Матрица:

\[
\begin{pmatrix}
\varnothing & \{A\}       & \{S\}         & \{S_1\}     \\
\varnothing & \varnothing & \cellcolor{lightgray}\{A, \boldsymbol{S}, S_1\} & \cellcolor{lightgray}\{S, \boldsymbol{S_1}\}  \\
\{A\}       & \varnothing & \{S_1\}       & \{B, S\}    \\
\varnothing & \varnothing & \{B\}         & \varnothing \\
\end{pmatrix}
\]

\textbf{Итерация 5.}
Рассмотрим на это шаге:

\begin{itemize}
    \setlength\itemsep{1em}
    \item $k = 1, i = 0, j = 3: A \in M[0, 1], S_1 \in M[1, 3]$, поскольку в грамматике есть правило $S \to A S_1$, добавляем нетерминал $S$ в ячейку $M[0, 3]$.
    \item $k = 3, i = 0, j = 2: S \in M[0, 3], B \in M[3, 2]$, поскольку в грамматике есть правило $S_1 \to S B$, добавляем нетерминал $S_1$ в ячейку $M[0, 2]$.
\end{itemize}

Матрица на этой итерации:
\[
\begin{pmatrix}
\varnothing & \{A\}       & \cellcolor{lightgray}\{S, \boldsymbol{S_1}\}    & \cellcolor{lightgray}\{\boldsymbol{S}, S_1\}  \\
\varnothing & \varnothing & \{A, S, S_1\} & \{S, S_1\}  \\
\{A\}       & \varnothing & \{S_1\}       & \{B, S\}    \\
\varnothing & \varnothing & \{B\}         & \varnothing \\
\end{pmatrix}
\]

\textbf{Итерация 6.}
Интересующие нас на этом шаге случаи:

\begin{itemize}
    \setlength\itemsep{1em}
    \item $k = 0, i = 2, j = 2: A \in M[2, 0], S_1 \in M[0, 2]$, поскольку в грамматике есть правило $S \to A S_1$, добавляем нетерминал $S$ в ячейку $M[2, 2]$.
    \item $k = 2, i = 2, j = 3: S \in M[2, 2], B \in M[2, 3]$, поскольку в грамматике есть правило $S_1 \to S B$, добавляем нетерминал $S_1$ в ячейку $M[2, 3]$.
\end{itemize}

Матрица после данного шага:

\[
\begin{pmatrix}
\varnothing & \{A\}       & \{S, S_1\}    & \{S, S_1\}    \\
\varnothing & \varnothing & \{A, S, S_1\} & \{S, S_1\}    \\
\{A\}       & \varnothing & \cellcolor{lightgray}\{\boldsymbol{S}, S_1\}    & \cellcolor{lightgray}\{B, S, \boldsymbol{S_1}\} \\
\varnothing & \varnothing & \{B\}         & \varnothing   \\
\end{pmatrix}
\]

На следующей итерации матрица не изменяется, поэтому заканчиваем работу алгоритма. В результате, если ячейка $M[i, j]$ содержит стартовый нетерминал $S$, то существует путь из $i$ в $j$, удовлетворяющий ограничениям, заданным грамматикой.
\end{example}

Можно заметить, что мы делаем много лишних итераций.
Можно переписать алгоритм так, чтобы он не просматривал заведомо пустые ячейки.
Данную модификацию предложил Й.Хеллингс в работе~\cite{hellingsRelational}, также она реализована в работе~\cite{10.1007/978-3-319-46523-4_38}.

Псевдокод алгоритма Хеллингса представлен в листинге~\ref{alg:graphParseHellings}.

\begin{algorithm}[H]
    \begin{algorithmic}[1]
        \caption{Алгоритм Хеллингса}
        \label{alg:graphParseHellings}
        \Function{contextFreePathQuerying}{$G= \langle \Sigma, N, P, S \rangle$, $\mathcal{G} = \langle V,E,L \rangle$}

        \State{$r \gets \{(N_i,v,v) \mid v \in V \wedge N_i \to \varepsilon \in P \} \cup \{(N_i,v,u) \mid (v,t,u) \in E \wedge N_i \to t \in P \}$}
        \State{$m \gets r$}
        \While{$m \neq \varnothing$}
        \State{$(N_i,v,u) \gets$ m.pick()}
        \For {$(N_j,v',v) \in r$}
            \For {$N_k \to N_j N_i \in P$ таких что $((N_k, v',u) \notin r)$}
                \State{$m \gets  m \cup \{(N_k, v',u)\}$}
                \State{$r \gets  r \cup \{(N_k, v',u)\}$}                
            \EndFor
        \EndFor
        \For {$(N_j,u,v') \in r$}
            \For {$N_k \to N_i N_j \in P$ таких что $((N_k, v, v') \notin r)$}
                \State{$m \gets  m \cup \{(N_k, v, v')\}$}
                \State{$r \gets  r \cup \{(N_k, v, v')\}$}                
            \EndFor
        \EndFor

        \EndWhile
        \State \Return $r$
        \EndFunction
    \end{algorithmic}
\end{algorithm}


\begin{example}
  Запустим алгоритм Хеллингса на нашем примере.
  
  \textbf{Инициализация}
  $$
  m = r = \{(A,0,1),(A,1,2),(A,2,0),(B,2,3),(B,3,2)\}
  $$
  
  \textbf{Итерации внешнего цикла.} Будем считеть, что $r$ и $m$ --- упорядоченные списки и $pick$ возврпщает его голову, оставляя хвост.
  Новые элементы добавляются в конец.
  \begin{enumerate}
  \item Обрабатываем $(A,0,1)$. 
  Ни один из вложенных циклов не найдёт новых путей, так как для рассматриваемого ребра есть только две возможности достроить путь: $2 \xrightarrow{A} 0 \xrightarrow{A} 1$ и $0 \xrightarrow{A} 1 \xrightarrow{A} 2$
  и ни одна из соответствующих строк не выводтся в заданной граммтике.
  \item Перед началом итерации 
     $$
     m = \{(A,1,2),(A,2,0),(B,2,3),(B,3,2)\},
     $$ $r$ не изменилось.
     Обрабатываем $(A,1,2)$.
     В данной ситуации второй цикл найдёт тройку $(B,2,3)$ и соответсвующее правило $S \to A \ B$. 
     Это значит, что и в $m$ и в $r$ добавится тройка $(S, 1, 3)$.
  \item
   Перед началом итерации 
     $$
     m = \{(A,2,0),(B,2,3),(B,3,2),(S,1,3)\},
     $$ 
     $$
     r= \{(A,0,1),(A,1,2),(A,2,0),(B,2,3),(B,3,2),(S,1,3)\}.
     $$
     Обрабатываем $(A,2,0)$. 
     Внутринние циклы ничего не найдут, новых путей н появится.
   \item
   Перед началом итерации 
     $$
     m = \{(B,2,3),(B,3,2),(S,1,3)\},
     $$ 
     $$
     r= \{(A,0,1),(A,1,2),(A,2,0),(B,2,3),(B,3,2),(S,1,3)\}.
     $$
     Обрабатываем $(B,2,3)$. 
     Первый цикл мог бы найти $(A,1,2)$, однако при проверке во вложенном цикле выяснится, что $(S, 1, 3)$ уже найдена. 
     В итоге, на данной итерации новых путей н появится.
   \item
   Перед началом итерации 
     $$
     m = \{(B,3,2),(S,1,3)\},
     $$ 
     $$
     r= \{(A,0,1),(A,1,2),(A,2,0),(B,2,3),(B,3,2),(S,1,3)\}.
     $$
     Обрабатываем $(B,3,2)$. 
     Первый цикл найдёт $(S,1,3)$ и соответствующее правило $S_1 \to S \ B$. 
     Это значит, что и в $m$ и в $r$ добавится тройка $(S_1, 1, 2)$. 
   \item
   Перед началом итерации 
     $$
     m = \{(S,1,3),(S_1, 1, 2)\},
     $$ 
     $$
     r= \{(A,0,1),(A,1,2),(A,2,0),(B,2,3),(B,3,2),(S,1,3),(S_1, 1, 2)\}.
     $$
     Обрабатываем $(S,1,3)$. 
     Второй цикл мог бы найти $(B,3,2)$, однако при проверке во вложенном цикле выяснится, что $(S_1, 1, 2)$ уже найдена. 
     В итоге, на данной итерации новых путей н появится.
   \item
   Перед началом итерации 
     $$
     m = \{(S_1, 1, 2)\},
     $$ 
     $$
     r= \{(A,0,1),(A,1,2),(A,2,0),(B,2,3),(B,3,2),(S,1,3),(S_1, 1, 2)\}.
     $$
     Обрабатываем $(S_1,1,2)$. 
     Первый цикл найдёт $(A,0,1)$ и соответствующее правило $S \to A \ S_1$. 
     Это значит, что и в $m$ и в $r$ добавится тройка $(S, 0, 2)$. 

   \item
   Перед началом итерации 
     $$
     m = \{(S, 0, 2)\},
     $$ 
     $$
     r= \{(A,0,1),(A,1,2),(A,2,0),(B,2,3),(B,3,2),(S,1,3),(S_1, 1, 2),(S, 0, 2)\}.
     $$
     Обрабатываем $(S, 0, 2)$. 
     Найдено: $(B,2,3)$ и соответствующее правило $S_1 \to S \ B$. 
     B $m$ и в $r$ добавится тройка $(S_1, 0, 3)$. 

   \item
   Перед началом итерации 
     $$
     m = \{(S_1, 0, 3)\},
     $$ 
     \begin{align*}
     r= \{&(A,0,1),(A,1,2),(A,2,0),(B,2,3),(B,3,2),(S,1,3),(S_1, 1, 2),(S, 0, 2),\\
          &(S_1, 0, 3)\}.
     \end{align*}
     Обрабатываем $(S_1, 0, 3)$. 
     Найдено: $(A,2,0)$ и соответствующее правило $S \to A \ S_1$. 
     B $m$ и в $r$ добавится тройка $(S, 2, 3)$. 

   \item
   Перед началом итерации 
     $$
     m = \{(S, 2, 3)\},
     $$ 
     \begin{align*}
     r= \{&(A,0,1),(A,1,2),(A,2,0),(B,2,3),(B,3,2),(S,1,3),(S_1, 1, 2),(S, 0, 2),\\
          &(S_1, 0, 3),(S, 2, 3)\}.
     \end{align*}

     Обрабатываем $(S, 2, 3)$. 
     Найдено: $(B,3,2)$ и соответствующее правило $S_1 \to S \ B$. 
     B $m$ и в $r$ добавится тройка $(S_1, 2, 2)$. 

   \item
   Перед началом итерации 
     $$
     m = \{(S_1, 2, 2)\},
     $$ 
     \begin{align*}
     r= \{&(A,0,1),(A,1,2),(A,2,0),(B,2,3),(B,3,2),(S,1,3),(S_1, 1, 2),(S, 0, 2),\\
          &(S_1, 0, 3),(S, 2, 3),(S_1, 2, 2)\}.
     \end{align*}
     Обрабатываем $(S_1, 2, 2)$. 
     Найдено: $(A,1,2)$ и соответствующее правило $S \to A \ S_1$. 
     B $m$ и в $r$ добавится тройка $(S, 1, 2)$. 

   \item
   Перед началом итерации 
     $$
     m = \{(S, 1, 2)\},
     $$ 
     \begin{align*}
     r= \{&(A,0,1),(A,1,2),(A,2,0),(B,2,3),(B,3,2),(S,1,3),(S_1, 1, 2),(S, 0, 2),\\
          &(S_1, 0, 3),(S, 2, 3),(S_1, 2, 2),(S, 1, 2)\}.
     \end{align*}
     Обрабатываем $(S, 1, 2)$. 
     Найдено: $(B,2,3)$ и соответствующее правило $S_1 \to S \ B$. 
     B $m$ и в $r$ добавится тройка $(S_1, 1, 3)$. 

   \item
   Перед началом итерации 
     $$
     m = \{(S_1, 1, 3)\},
     $$ 
     \begin{align*}
     r= \{&(A,0,1),(A,1,2),(A,2,0),(B,2,3),(B,3,2),(S,1,3),(S_1, 1, 2),(S, 0, 2),\\
          &(S_1, 0, 3),(S, 2, 3),(S_1, 2, 2),(S, 1, 2),(S_1, 1, 3)\}.
     \end{align*}
     Обрабатываем $(S_1, 1, 3)$. 
     Найдено: $(A,0,1)$ и соответствующее правило $S \to A \ S_1$. 
     B $m$ и в $r$ добавится тройка $(S, 0, 3)$. 

   \item
   Перед началом итерации 
     $$
     m = \{(S, 0, 3)\},
     $$ 
     \begin{align*}
     r= \{&(A,0,1),(A,1,2),(A,2,0),(B,2,3),(B,3,2),(S,1,3),(S_1, 1, 2),(S, 0, 2),\\
          &(S_1, 0, 3),(S, 2, 3),(S_1, 2, 2),(S, 1, 2),(S_1, 1, 3),(S, 0, 3)\}.
     \end{align*}
     Обрабатываем $(S, 0, 3)$. 
     Найдено: $(B,3,2)$ и соответствующее правило $S_1 \to S \ B$. 
     B $m$ и в $r$ добавится тройка $(S_1, 0, 2)$. 

   \item
   Перед началом итерации 
     $$
     m = \{(S_1, 0, 2)\},
     $$ 
     \begin{align*}
     r= \{&(A,0,1),(A,1,2),(A,2,0),(B,2,3),(B,3,2),(S,1,3),(S_1, 1, 2),(S, 0, 2),\\
          &(S_1, 0, 3),(S, 2, 3),(S_1, 2, 2),(S, 1, 2),(S_1, 1, 3),(S, 0, 3),(S_1, 0, 2)\}.
     \end{align*}
     Обрабатываем $(S_1, 0, 2)$. 
     Найдено: $(A,2,0)$ и соответствующее правило $S \to A \ S_1$. 
     B $m$ и в $r$ добавится тройка $(S, 2, 2)$. 

   \item
   Перед началом итерации 
     $$
     m = \{(S, 2, 2)\},
     $$ 
     \begin{align*}
     r= \{&(A,0,1),(A,1,2),(A,2,0),(B,2,3),(B,3,2),(S,1,3),(S_1, 1, 2),(S, 0, 2),\\
          &(S_1, 0, 3),(S, 2, 3),(S_1, 2, 2),(S, 1, 2),(S_1, 1, 3),(S, 0, 3),(S_1, 0, 2),\\
          &(S, 2, 2)\}.
     \end{align*}
     Обрабатываем $(S, 2, 2)$. 
     Найдено: $(B,2,3)$ и соответствующее правило $S_1 \to S \ B$. 
     B $m$ и в $r$ добавится тройка $(S_1, 2, 3)$. 

   \item
   Перед началом итерации 
     $$
     m = \{(S_1, 2, 3)\},
     $$ 
     \begin{align*}
     r= \{&(A,0,1),(A,1,2),(A,2,0),(B,2,3),(B,3,2),(S,1,3),(S_1, 1, 2),(S, 0, 2),\\
          &(S_1, 0, 3),(S, 2, 3),(S_1, 2, 2),(S, 1, 2),(S_1, 1, 3),(S, 0, 3),(S_1, 0, 2),\\
          &(S, 2, 2),(S_1, 2, 3)\}.
     \end{align*}
     Обрабатываем $(S_1, 2, 3)$. 
     Могло бы быть найдено: $(A,1,2)$ и соответствующее правило $S \to A \ S_1$, однако тройка $(S, 1, 3)$ уже есть в $r$. 
     А значит никаких новых троек найдено не будет и $m$ становится пустым.
     Это была последняя итерация внешнего цикла, в $r$ на текущий момент уже содержится всё ршение. 

  \end{enumerate}

\end{example}

Как можно заметить, количество итераций внешнего цикла также получилось достаточно большим. 
Проверьте, зависит ли оно от порядка обработки элементов из $m$.
При этом внутренние циклы в нашем случае достаточно короткие, так как просматриваются только ``существенные'' элементы и избегается дублирование.

\section{Вопросы и задачи}
\begin{enumerate}
    \item Проверить работу алгоритма CYK для цепочек на грамматике
    \begin{flushleft}
    $E \to E + E$ \\
    $E \to E * E$ \\
    $E \to (E)$   \\
    $E \to n$     \\
    \end{flushleft}
    и словах (алфавит $\Sigma = \{n, +, *, (, )\}$)
    \begin{flushleft}
    $ (n + n) * n$    \\
    $ n + n * n$      \\
    $n + n + n + n$   \\
    $n + (n * n) + n$ \\
    \end{flushleft}

    \item Изучить вычислительную сложность алгоритма CYK для матриц в зависимости от размера входного графа (размер грамматики считать фиксированным).

    \item Проверить работу алгоритма CYK для графов на графе

    \begin{center}
        \begin{tikzpicture}[node distance=3cm,shorten >=1pt,on grid,auto]
        \node[state] (q_0)  {$0$};
        \node[state] (q_1) [right=of q_0]  {$1$};
        \node[state] (q_2) [right=of q_1]  {$2$};
        \node[state] (q_3) [right=of q_2]  {$3$};
        \node[state] (q_4) [right=of q_3]  {$4$};
        \path[->]
        (q_0) edge  node {$a$} (q_1)
        (q_1) edge  node {$b$} (q_2)
        (q_2) edge  node {$a$} (q_3)
        (q_3) edge  node {$b$} (q_4)
        (q_1) edge[bend left, above]  node {$b$} (q_3)
        (q_4) edge[bend left, below]  node {$a$} (q_1);
        \end{tikzpicture}
    \end{center}

    И грамматике

    \begin{flushleft}
        $S \to S S$ \\
        $S \to A B$ \\
        $A \to a$   \\
        $B \to b$     \\
    \end{flushleft}

    \item Оцените временную сложность алгоритма Хеллингса и сравните её с оценкой для наивного обобщения CYK.

\end{enumerate}
\chapter{КС и конъюнктивная достижимость через произведение матриц}\label{chpt:MatrixBasedAlgos}

В данном разделе мы рассмотрим алгоритм решения задачи контекстно-свободной и конъюнктивной достижимости, основанный на произведении матриц. Будет показано, что при использовании конъюнктивных граммтик, представленный алгоритм находит переапроксимацию истинного решения задачи.

\section{КС достижимость через произведение матриц}
\label{Matrix-CFPQ}
В главе~\ref{graph:CYK}~был изложен алгоритм для решения задачи КС достижимости на основе CYK. Заметим, что обход матрицы напоминает умножение матриц в ячейках которых множества нетерминалов:

\begin{align*}
M_3 = &M_1 \times M_2 \\
M_3[i,j] = &\sum_{k=1}^{n} M[i,k] * M[k,j]
\end{align*}
, где сумма --- это объединение множеств:

$$
\sum_{k=1}^{n} = \bigcup_{k=1}^{n}
$$
, а поэлементное умножение определено следующим образом:
$$
S_1 * S_2 = \{N_1^0 ... N_1^m\} * \{N_2^0 ... N_2^l\} = \{N_3 \mid (N_3 \rightarrow N_1^i N_2^j) \in P\}.
$$

Таким образом, алгоритм решения задачи КС достижимости может быть выражена в терминах перемножения матриц над полукольцом с соответствующими операциями.

Для частного случая этой задачи, синтаксического анализа линейного входа, существует алгоритм Валианта~\cite{Valiant:1975:GCR:1739932.1740048}, использующий эту идею.
Однако он не обобщается на графы из-за того, что существенно использует возможность упорядочить обход матрицы (см. разницу в CYK для линейного случая и для графа). 
Поэтому, хотя для линейного случая алгоритм Валианта является алгоритмом синтаксического анализа для произвольных КС граммтик за субкубическое время, его обобщение до задачи КС достижимости в произвольных графах с сохранением асимптотики является нетривиальной задачей~\cite{Yannakakis}. 
В настоящее время алгоритм с субкубической сложностью получен только для частного случая --- языка Дика с одним типом скобок --- Филипом Брэдфорlом~\cite{8249039}.

В случае с линейным входом, отдельного внимания заслуживает работа Лиллиан Ли (Lillian Lee)~\cite{Lee:2002:FCG:505241.505242}, где она показывает, что задача перемножения матриц сводима к синтаксическому анализу линейного входа. Аналогичных результатов для графов на текущий момент не известно.

Поэтому рассмотрим более простую идею, изложенную в статье Рустама Азимова~\cite{Azimov:2018:CPQ:3210259.3210264}: будем строить транзитивное замыкание графа через наивное (не через возведение в квадрат) умножение матриц.

Пусть $\mathcal{G} = (V, E)$ --- входной граф и $G = (N,\Sigma,P)$ --- входная грамматика. Тогда алгоритм может быть сформулирован как представлено в листинге~\ref{alg:graphParse}.

\begin{algorithm}[H]
\begin{algorithmic}[1]
\caption{Context-free recognizer for graphs}
\label{alg:graphParse}
\Function{contextFreePathQuerying}{$\mathcal{G}$, G}
    
    \State{$n \gets$ количество узлов в $\mathcal{G}$}
    \State{$E \gets$ направленные ребра в $\mathcal{G}$}
    \State{$P \gets$ набор продукций из $G$}
    \State{$T \gets$ матрица $n \times n$, в которой каждый элемент $\varnothing$}
    \ForAll{$(i,x,j) \in E$}
    \Comment{Инициализация матрицы}
        \State{$T_{i,j} \gets T_{i,j} \cup \{A~|~(A \rightarrow x) \in P \}$}
    \EndFor
    \ForAll{$i \in 0\ldots n-1$}
    \Comment{Добавление петель для нетерминалов, порождающих пустую строку}
        \State{$T_{i,i} \gets T_{i,i} \cup \{ A \in N \mid A \to \varepsilon \}$}
    \EndFor    
    \While{матрица $T$ меняется}
       
        \State{$T \gets T \cup (T \times T)$}
        \Comment{Вычисление транзитивного замыкания} 
    \EndWhile
\State \Return $T$
\EndFunction
\end{algorithmic}
\end{algorithm}


\begin{example}[Пример работы]

Пусть есть граф $\mathcal{G}$:
\begin{center}
    \begin{tikzpicture}[node distance=2.5cm,shorten >=1pt,on grid,auto]
    \node[state] (q_0)   {$1$};
    \node[state] (q_1) [above right=of q_0] {$2$};
    \node[state] (q_2) [right=of q_0] {$0$};
    \node[state] (q_3) [right=of q_2] {$3$};
    \path[->]
    (q_0) edge  node {a} (q_1)
    (q_1) edge  node {a} (q_2)
    (q_2) edge  node {a} (q_0)
    (q_2) edge[bend left, above]  node {b} (q_3)
    (q_3) edge[bend left, below]  node {b} (q_2);
    \end{tikzpicture}
    
\end{center}

и грамматика $G$:
\begin{align*}
S   &\to A B    &A  \to a \\ 
S  &\to A S_1   &B  \to b\\ 
S_1 &\to S B 
\end{align*}


Пусть $T_i$ --- матрица, полученная из $T$ после применения цикла, описанного в строках \textbf{8-9} алгоритма~\ref{alg:graphParse}, $i$ раз.
Тогда $T_0$, полученная напрямую из графа, выглядит следующим образом:

\[
T_0 = \begin{pmatrix}
    \varnothing & \{A\}       & \varnothing & \{B\}       \\
    \varnothing & \varnothing & \{A\}       & \varnothing \\
    \{A\}       & \varnothing & \varnothing & \varnothing \\
    \{B\}       & \varnothing & \varnothing & \varnothing \\
\end{pmatrix}
\]

Далее показано получение матрицы $T_1$.

\[
T_0 \times T_0 = \begin{pmatrix}
    \varnothing & \varnothing & \varnothing & \varnothing \\
    \varnothing & \varnothing & \varnothing & \varnothing \\
    \varnothing & \varnothing & \varnothing & \{S\}       \\
    \varnothing & \varnothing & \varnothing & \varnothing \\
\end{pmatrix}
\]

\[
T_1 = T_0 \cup (T_0 \times T_0) = \begin{pmatrix}
    \varnothing & \{A\}       & \varnothing & \{B\}       \\
    \varnothing & \varnothing & \{A\}       & \varnothing \\
    \{A\}       & \varnothing & \varnothing & \cellcolor{lightgray} \{\pmb{S}\}       \\
    \{B\}       & \varnothing & \varnothing & \varnothing \\
\end{pmatrix}
\]

После первой итерации цикла нетерминал в ячейку $T[2,3]$ добавился нетерминал $S$. 
Это означает, что существует такой путь $\pi$ из вершины 2 в вершину 3 в графе $\mathcal{G}$, что $S \xrightarrow{*} \omega(\pi)$. В данном примере путь состоит из двух ребер $2 \xrightarrow{a} 0$ и $ 0 \xrightarrow{b} 3$, так что $S \xrightarrow{*} ab$.

Вычисление транзитивного замыкания заканчивается через $k$ итераций, когда достигается неподвижная точка процесса: $T_{k-1} = T_k$. Для данного примера $k = 13$, так как $T_{13} = T_{12}$. Весь процесс рабты алгоритма (все матрицы $T_i$) показан ниже (на каждой итерации новые элементы выделены жирным).

{\footnotesize
\begin{alignat*}{7}
& &&T_2 &&= \begin{pmatrix}
\varnothing & \{A\}       & \varnothing & \{B\}       \\
\varnothing & \varnothing & \{A\}       & \varnothing \\
\cellcolor{lightgray} \{A, \pmb{S_1}\}  & \varnothing & \varnothing & \{S\}       \\
\{B\}       & \varnothing & \varnothing & \varnothing \\
\end{pmatrix} \ \ \ \ &&T_3 &&= \begin{pmatrix}
\varnothing & \{A\}       & \varnothing & \{B\}       \\
\cellcolor{lightgray} \{\pmb{S}\}       & \varnothing & \{A\}       & \varnothing \\
\{A, S_1\}  & \varnothing & \varnothing & \{S\}       \\
\{B\}       & \varnothing & \varnothing & \varnothing \\
\end{pmatrix} \\ & &&T_4 &&= \begin{pmatrix}
\varnothing & \{A\}       & \varnothing & \{B\}       \\
\{S\}       & \varnothing & \{A\}       & \cellcolor{lightgray} \{\pmb{S_1}\}     \\
\{A, S_1\}  & \varnothing & \varnothing & \{S\}       \\
\{B\}       & \varnothing & \varnothing & \varnothing \\
\end{pmatrix}  \ \ \ \ &&T_5 &&= \begin{pmatrix}
\varnothing & \{A\}       & \varnothing & \cellcolor{lightgray} \{B, \pmb{S}\}    \\
\{S\}       & \varnothing & \{A\}       & \{S_1\}     \\
\{A, S_1\}  & \varnothing & \varnothing & \{S\}       \\
\{B\}       & \varnothing & \varnothing & \varnothing \\
\end{pmatrix} \\ & &&T_6 &&= \begin{pmatrix}
\cellcolor{lightgray} \{\pmb{S_1}\}     & \{A\}       & \varnothing & \{B, S\}    \\
\{S\}       & \varnothing & \{A\}       & \{S_1\}     \\
\{A, S_1\}  & \varnothing & \varnothing & \{S\}       \\
\{B\}       & \varnothing & \varnothing & \varnothing \\
\end{pmatrix} \ \ \ \ &&T_7 &&= \begin{pmatrix}
\{S_1\}     & \{A\}       & \varnothing & \{B, S\}    \\
\{S\}       & \varnothing & \{A\}       & \{S_1\}     \\
\cellcolor{lightgray} \{A, S_1, \pmb{S}\}  & \varnothing & \varnothing & \{S\}    \\
\{B\}       & \varnothing & \varnothing & \varnothing \\
\end{pmatrix}  \\
& &&T_8 &&= \begin{pmatrix}
\{S_1\}     & \{A\}       & \varnothing & \{B, S\}    \\
\{S\}       & \varnothing & \{A\}       & \{S_1\}     \\
\{A, S_1, S\}  & \varnothing & \varnothing & \cellcolor{lightgray} \{S, \pmb{S_1}\} \\
\{B\}       & \varnothing & \varnothing & \varnothing \\
\end{pmatrix} \ \ \ \ &&T_9 &&= \begin{pmatrix}
\{S_1\}     & \{A\}       & \varnothing & \{B, S\}    \\
\{S\}       & \varnothing & \{A\}       & \cellcolor{lightgray} \{S_1, \pmb{S}\}     \\
\{A, S_1, S\}  & \varnothing & \varnothing & \{S, S_1\} \\
\{B\}       & \varnothing & \varnothing & \varnothing \\
\end{pmatrix} \\ & &&T_{10} &&= \begin{pmatrix}
\{S_1\}     & \{A\}       & \varnothing & \{B, S\}    \\
\cellcolor{lightgray} \{S, \pmb{S_1}\}       & \varnothing & \{A\}       & \{S_1, S\}     \\
\{A, S_1, S\}  & \varnothing & \varnothing & \{S, S_1\} \\
\{B\}       & \varnothing & \varnothing & \varnothing \\
\end{pmatrix}  \ \ \ \  &&T_{11} &&= \begin{pmatrix}
\cellcolor{lightgray} \{S_1, \pmb{S}\}     & \{A\}       & \varnothing & \{B, S\}    \\
\{S, S_1\}       & \varnothing & \{A\}       & \{S_1, S\}     \\
\{A, S_1, S\}  & \varnothing & \varnothing & \{S, S_1\} \\
\{B\}       & \varnothing & \varnothing & \varnothing \\
\end{pmatrix} \\ & &&T_{12} &&= \begin{pmatrix}
\{S_1, S\}     & \{A\}       & \varnothing & \cellcolor{lightgray} \{B, S, \pmb{S_1}\}    \\
\{S, S_1\}       & \varnothing & \{A\}       & \{S_1, S\}     \\
\{A, S_1, S\}  & \varnothing & \varnothing & \{S, S_1\} \\
\{B\}       & \varnothing & \varnothing & \varnothing \\
\end{pmatrix} \ \ \ \ &&T_{13} &&= \begin{pmatrix}
\{S_1, S\}     & \{A\}       & \varnothing & \{B, S, S_1\}    \\
\{S, S_1\}       & \varnothing & \{A\}       & \{S_1, S\}     \\
\{A, S_1, S\}  & \varnothing & \varnothing & \{S, S_1\} \\
\{B\}       & \varnothing & \varnothing & \varnothing \\
\end{pmatrix}
\end{alignat*}
}

Таким образом, результат алгоритма~\ref{alg:graphParse} для нашего примера --- это матрица $T_{13} = T_{12}$. Заметим, что для данного алгоритма приведённый пример также является худшим случаем: на каждой итерации в матрицу добавляется ровно один нетерминал, при том, что необходимо заполнить порядка $O(n^2)$ ячеек.

\end{example}


\subsection{Расширение алгоритма для конъюнктивных грамматик}

Матричный алгоритм для конъюнктивных грамматик отличается от алгоритма~\ref{alg:graphParse} для контекстно-свободных грамматик только операцией умножения матриц, в остальном алгоритм остается без изменений. Определим операцию умножения матриц.
\begin{definition}
    Пусть $M_1$ и $M_2$ матрицы размера $n$. Определим операцию $\circ$  сдедующим образом:
     $$M_1 \circ M_2 = M_3,$$ $$M_3 [i,j] = \{A \mid \exists (A \rightarrow B_1 C_1 \& \ldots \& B_m C_m) \in P, (B_k , C_k) \in d[i,j] \forall k = 1,\ldots,m\}$$, где $$d[i,j] = \bigcup_{k = 1}^{n} M_1 [i,k] \times M_2 [k,j].$$
\end{definition}

Важно заметить, что алгоритм для конъюнктивных грамматик, в отличие от алгоритма для контекстно-свободных грамматик, дает лишь верхнюю оценку ответа. То есть все нетерминалы, которые должны быть в ячейках матрицы результата, содержатся там, но вместе с ними содержатся и лишние нетерминалы. Рассмотрим пример, иллюстрирующий появление лишних нетерминалов.

\begin{example}
    Грамматика $G$:
    \begin{align*}
    S &\to AB \& DC & C &\to c \\ 
    A &\to a        & D &\to DC \mid b\\
    B &\to BC \mid b
    \end{align*}
    Очевидно, что грамматика $G$ задает язык из одного слова $L(G) = \{abc\} = \{abc^*\} \cap \{a^* bc\}$.
    
    Пусть есть граф $\mathcal{G}$:
    \begin{center}
        \begin{tikzpicture}[node distance=2.5cm, shorten >=1pt,on grid,auto]
        \node[state] (q_0)   {$0$};
        \node[state] (q_1) [right=of q_0] {$1$};
        \node[state] (q_2) [right=of q_1] {$2$};
        \node[state] (q_6) [below=of q_2] {$6$};
        \node[state] (q_3) [right=of q_2] {$3$};
        \node[state] (q_5) [right=of q_6] {$5$};
        \node[state] (q_4) [right=of q_3] {$4$};
        \path[->]
        (q_0) edge  node {a} (q_1)
        (q_1) edge  node {b} (q_2)
        (q_1) edge  node {a} (q_6)
        (q_2) edge  node {c} (q_3)
        (q_3) edge  node {c} (q_4)
        (q_6) edge  node {b} (q_5)
        (q_5) edge  node {c} (q_4);
        \end{tikzpicture}
    \end{center}
    Применяя алгоритм, получим, что существует путь из вершины 0 в вершину 4, выводимый из нетерминала $S$. Однако очевидно, что в графе такого пути нет. 
    Такое поведение алгоритма наблюдается из-за того, что существует путь ``abcc'', соответствующий $L(AB) = \{abc^*\}$ и путь ``aabc'', соответствующий $L(DC) = \{a^{*}bc\}$, но они различны. Однако алгоритм не может это проверить, так как оперирует понятием достижимости между вершинами, а не наличием различных путей. Более того, в общем случае для конъюнктивных граммтик такую проверку реалиховать нельзя. Поэтому для классической семантики достидимости с ограничениями в терминах конъюнктивных граммтик результат работы алгоритма является оценкой сверху.
    
    Существует альтернативная семантика, когда мы трактуем конъюнкцию в праой части правил как крнъюнкцию в Datalog (подробнее о Datalog в параграфе~\ref{Subsection Datalog}). Т.е если есть правило $S \to AB \& DC$, то должен быть путь принадлежащий языку $L(AB)$ и путь принадлежащий языку $L(DC)$. В такой семантике алгоритм дает точный ответ.
\end{example}

Подробнее алгоритм описан в статье Рустама Азимова и Семёна Григорьева~\cite{565CECD7E8F5C6063935B41DB41797AA37D53B04}. Стоит также отметить, что обобщения данного алогритма для булевых грамматик не существует, хотя и сущетсвует частное решение для случая, когда граф не содержит циклов (является DAG-ом), предложенное Екатериной Шеметовой~\cite{Shemetova2019}.

\section{Особенности реализации}

Алгоритмы, описанные выще, удобны с точки зрения реализации тем, что могут быть эффективно реализованы с использованием высокопроизводительных библиотек линейной алгебры, которые эксплуатируют возможности параллельных вычислений на современных CPU и  GPGPU~\cite{Mishin:2019:ECP:3327964.3328503}. 
Это позволяет с минимальными затратими получить эффективную параллельную реализацию алгоритма для решения задачи КС достижимости в графах. 
Благодаря этому, хотя асимптотически приведенные алгоритмы имеют большую сложность чем, скажем, алгоритм Хеллингса, в результате эффективного распараллеливания на практике они работают быстрее однопоточных алгоритов с лучшей сложностью.

Далее рассмотрим некоторые детали, упрощающие реализацию с использованием современных библиотек и аппаратного обеспечения.

Так как множество нетерминалов и правил конечно, то мы можем свести представленный выше алгоритм к булевым матрицам: для каждого нетерминала заведём матрицу, такую что в ячейке стоит 1 тогда и только тогда, когда в исходной матрице в соответствующей ячейке сожержится этот нетерминал.
Тогда перемножение пары таких матриц, соответсвующих нетерминалам $A$ и $B$, соответствует построению путей, выводимых из нетерминалов, для которых есть правила с правой частью вида $A B$. 

\begin{example}
Представим в виде набора булевых матриц следующую матрицу:
\[
T_0 = \begin{pmatrix}
\varnothing & \{A\}       & \varnothing & \{B\}       \\
\varnothing & \varnothing & \{A\}       & \varnothing \\
\{A\}       & \varnothing & \varnothing & \varnothing \\
\{B\}       & \varnothing & \varnothing & \varnothing \\
\end{pmatrix}
\]

Тогда:
\begin{alignat*}{7}
& &&T_{0\_A} &&= \begin{pmatrix}
0 & 1       & 0 & 0       \\
0 & 0 & 1       & 0 \\
1  & 0 & 0 & 0       \\
0       & 0 & 0 & 0 \\
\end{pmatrix} \ \ \ \ &&T_{0\_B} &&= \begin{pmatrix}
0 & 0       & 0 & 1       \\
0       & 0 & 0       & 0 \\
0  & 0 & 0 & 0       \\
1       & 0 & 0 & 0 \\
\end{pmatrix}
\end{alignat*}
Тогда при наличии правила $S \to A B$ в граммтике получим:
\[
T_{1\_S} =T_{0\_A} \times T_{0\_B} = \begin{pmatrix}
0 & 0       & 0 & 0       \\
0       & 0 & 0       & 0 \\
0  & 0 & 0 & 1       \\
0       & 0 & 0 & 0 \\
\end{pmatrix}
\]
\end{example}

Алгоритм же может быть переформулирован так, как показано в листинге~\ref{lst:algo1}. Такой взгляд на алгоритм позволяет использдвать для его реализации существующие высокорпоизводительные библиотеки для работы с булевыми матрицами (например M4RI\footnote{M4RI --- одна из высокопроизводительных библиотек для работы с логическими матрицами на CPU. Реализует Метод Четырёх Русских. Исходный код библиотеки: \url{https://bitbucket.org/malb/m4ri/src/master/}. Дата посещения: 30.03.2020.}~\cite{DBLP:journals/corr/abs-0811-1714}) или библиотеки для линейной алгебры (например CUSP~\cite{Cusp}).

\begin{algorithm}
  \floatname{algorithm}{Listing}
\begin{algorithmic}[1]
\caption{Context-free path quering algorithm. Boolean matrix version}
\label{lst:algo1}
\Function{evalCFPQ}{$D=(V,E), G=(N,\Sigma,P)$}
    \State{$n \gets$ |V|}
    \State{$T \gets \{T^{A_i} \mid A_i \in N, T^{A_i}$ is a matrix $n \times n$, $T^{A_i}_{k,l} \gets$ \texttt{false}\} }
    \ForAll{$(i,x,j) \in E$, $A_k \mid A_k \to x \in P$}
        %\Comment{Matrices initialization}
        %\For{$A_k \mid A_k \to x \in P$}
          {$T^{A_k}_{i,j} \gets \texttt{true}$}
        %\EndFor
    \EndFor
    \For{$A_k \mid A_k \to \varepsilon \in P$}
       {$T^{A_k}_{i,i} \gets \texttt{true}$}
    \EndFor

    \While{any matrix in $T$ is changing}
        %\Comment{Transitive closure calculation}
        \For{$A_i \to A_j A_k \in P$}
          { $T^{A_i} \gets T^{A_i} + (T^{A_j} \times T^{A_k})$ } 
        \EndFor
    \EndWhile
\State \Return $T$
\EndFunction
\end{algorithmic}
\end{algorithm}

С другой стороны, для запросов, выразимых в терминах граммтик с небольшим количеством нетерминалов, практически может быть выгодно представлять множества нетерминалов в ячейке матрицы в виде битового вектора следующим образом.
Нумеруем все нетерминалы с нуля, в векторе стоит 1 на позиции $i$, если в множестве есть нетерминал с номером $i$.
Таким образом, в каждой ячейке хранится битовый вектор длины $|N|$.
Тогда операция умножения определяется следующим образом:
$$v_1 \times v_2 = \{v \mid \exists (v,v_3) \in P, \textit{append}(v_1, v_2) \& v_3 = v_3\},$$ где $\&$ --- побитовое \texttt{``и''}.

Правила надо кодировать соответственно: продукция это пара, где первый элемент --- битовый вектор длины $|N|$ с единственной единицей в позиции, соответствующей нетерминалу в правой части, а второй элемент --- вектор длины $2|N|$, с двумя единицами кодирующими первый и второй нетерминалы, соответственно.

\begin{example}
Пусть $N = \{S, A, B\}$ и в грамматике есть продукция $S \to A B$. Тогда занумеруем нетерминалы $ (S, 0), (A, 1), (B, 2)$. Продукция примет вид $[1, 0, 0] \to [0, 1, 0, 0, 0, 1]$. Матрица $T_0$ примет вид (здесь ``$.$'' означает, что в ячейке стоит $[0,0,0]$):
\[
T_0 = \begin{pmatrix}
. & [0,1,0]       & . & [0,0,1]       \\
. & . & [0,1,0]       & . \\
[0,1,0]       & . & . & . \\
[0,0,1]      & . & . & . \\
\end{pmatrix}
\]

После выполнения умножения получим:
\[
T_1 = T_0 + T_0 \times T_0 =
\begin{pmatrix}
. & [0,1,0]       & . & [0,0,1]       \\
. & . & [0,1,0]       & . \\
[0,1,0]       & . & . & \cellcolor{lightgray}[1,0,0] \\
[0,0,1]      & . & . & . \\
\end{pmatrix}
\]
\end{example}


На практике в роли векторов могут выступать беззнаковые целые числа. 
Например, 32 бита под ячейки в матрице и 64 бита под правила (или 8 и 16, если позволяет количество нетерминалов в граммтике, или 16 и 32).
Тогда умножение выражается через битовые операции и сравнение, что довольно эффективно с точки зрения вычислений.

Для небольших запросов такой подход к реализации может оказаться быстрее --- в данном случае скорость зависит от деталей. Минус подхода в том, что нет возможности использовать готовые библиотеку ленейной алгебры без предварительной модификации. Только если они не являются параметризуемыми и не позволяют задать собственный тип и собственную операцию умножения и сложения (иными словами, собственное полукольцо). Такую возможность предусматривает, например, стандарт GraphBLAS\footnote{GraphBLAS --- открытый стандарт, описывающий набор примитивов и операций, необходимый для реализации графовых алгоритмов в терминах лнейной алгебры. Web-страница проекта: \url{https://github.com/gunrock/graphblast}. Дата доступа: 30.03.2020.} и, соответственно, его реализации, такие как SuiteSparce\footnote{SuteSparse --- это специализированное программное обеспечения для работы с разреженными матрицами, которое включает в себя реализацию GraphBLAS API. Web-страница проекта: \url{http://faculty.cse.tamu.edu/davis/suitesparse.html}. Дата доступа: 30.03.2020.}~\cite{Davis2018Algorithm9S}.

Также стоит замеить, что при работе с реальными графами матрицы как правило оказываются разреженными, а значит необходимо использовать соответствующие представления матриц (CRS, покоординатное, Quad Tree~\cite{quadtree}) и библиотеки, работающие с таким представлениями. 

Однако даже при использовании разреженных матриц, могут возникнуть проблемы с размером реальных данных и объёмом памяти. Напрмиер, для вычислений на GPGPU лучше всего, когда все нужные для вычисления матрицы помещаются на одну карту. Тогда можно свести обмен данными сежду хостом и графическим сопроцессором к мимнимуму. Если не помещаются все, то нужно, чтобы помещалясь хотя бы тройка непостредственно обрабатываемых матриц (два операнда и результат). В самом тяжёлом случае в памяти не удаётся раместить даже операнды целиком и тогда приходится прибегать к поблочному умнодению матриц.

Отдельной инженерной проблемой является масштабирование алгоритмов на несколько вычислительных узлов, как на несколько CPU, так и на несколько GPGPU.

Важным свойством рассмотренного алгоритма является то, что описанные проблемы с объёмом памяти и масштабированием могут эффективно решаться в рамках библиотек линейной алгебры общего назначения, что избавляет от необходимости создавать специализированные решения для конкетных задач. 


\section{Вопросы и задачи}
\begin{enumerate}
    \item Находить кратчайшие пути в графах, используя идеи из секции~\ref{Matrix-CFPQ}.
    \item Превратить граф, использующийся для CFPQ, в дерево.
    \item Реализовать предложенные идеи на различных архитектурах.
    \item Замерить производительность и расходы памяти по сравнению с существующими реализациями.
\end{enumerate}

\input{TensorProduct}
\input{SPPF}
\input{GLL-based_CFPQ}
\input{GLR-based_CFPQ}
\chapter{Комбинаторы для КС запросов}

\section{Парсер комбинаторы}

Что это, с чем едят, плюсы, минусы. Про семантику, безопасность, левую рекурсию и т.д.
Набор примитивных парсеров и функций, которые умеют из существующих арсеров строить более сложные (собственно, комбинаторы парсеров).

Разобрать символ, разобрать последовательность, разобрать альтернативу. впринципе, этого достаточно, но это не очень удобно.

Проблемы с левой рекурсией.
Существуют решения. Одно из них --- Meerkat.
Подробно про него?

\section{Комбинаторы для КС запросов}

Вообще говоря, идея использовать комбинаторы для навигации по графам достаточно очевидно и не нова.
немного про Trails~\cite{Kroni:2013:PGA:2489837.2489844}.

Комбинаторы для запросов к графам на основе Meerkat~\cite{Verbitskaia:2018:PCC:3241653.3241655}

Обобщённые запросы, типобезопасность и всё такое.
Примеры запросов.

\section{Вопросы и задачи}
\begin{enumerate}
  \item Реализовать библиотеку парсер комбинаторов.
  \item Что-нибудь полезное с ними сделать.
\end{enumerate}

\chapter{Производные для КС запросов}\label{chpt:CFPQ_Derivatives}

В данной главе мы рассмотрим производные Бжозовского и их возможное применение.

\section{Производные}

Впервые производные формальных языков были определены в 1964 году учёным Янушем Бжозовским (в честь него они и были названы). Он определил это понятие для регулярных языков, предложил алгоритм для вычисления производной обобщенного регулярного выражения, также Бжозовский занимался исследованием свойств производных {\cite{Brzozowski1964}}. 

Изначально всё начиналось с производных регулярных языков, и в главе мы в основном будем говорить именно о них.

Рассмотрим формальное определение производной произвольного языка:

\begin{definition}
    \textit{Производной языка $\mathcal{L}$ по символу а} называется язык $\mathcal{L'} = \partial_{a}(\mathcal{L}) = \{w \mid aw \in \mathcal{L}\}$, где:
    \begin{itemize}
        \item $\mathcal{L} \subseteq \{w \mid w \in \Sigma^*\}$ --- произвольный язык над алфавитом $\Sigma$
        \item $a \in \Sigma$ --- символ, по которому берётся производная
    \end{itemize}
    То есть фактически производная языка --- это суффиксы слов, начинающихся на символ, по которому язык дифференцируется.
\end{definition}

\begin{example}[Производные языка]
Рассмотрим язык $\mathcal{L} = \{pen, plain, day, pray\}$ и несколько языков, порождённых от него с помощью дифференцирования: 
\begin{enumerate}
    \item $\mathcal{L'} = \partial_{p}(\mathcal{L}) = \{en, lain, ray\}$
    \item $\mathcal{L''} = \partial_{e}(\mathcal{L'}) = \{n\}$
    \item $\mathcal{L'''} =\partial_{n}(\mathcal{L''}) = \{\varepsilon\}$
\end{enumerate}
\end{example}

\section{Принадлежность языку}

С помощью производных ествественным образом можно проверять принадлежность слова регулярному языку.
Есть регулярное выражение $R$ и язык, порожденный этим регулярным выражением $L(R)$. Возьмём произвольное слово $w$ и зададимся вопросом: $w \in L(R)$? 

Пусть $w = a_{0}a_{1}... a_{n-1}$, а новый язык $\mathcal{L'}$ получен последовательным дифференцирование исходного языка $L(R)$ по каждому из символов слова $w$, то есть $\mathcal{L'} = \partial_{a_{n-1}}(...\partial_{a_{1}}(\partial_{a_{0}}(L(R)))...)$. В таком случае принадлежность $w$ языку $L(R)$ определяется наличием пустого слова в итоговом языке: $\varepsilon \in \mathcal{L'} \Rightarrow w \in L(R)$. 

\begin{example}
Дан язык $\mathcal{L} = \{pen, plain, day, pray\}$. Принадлежат ли данному языку слова  $pen$, $pet$?
\begin{itemize}
    \item $\partial_{p}(\mathcal{L}) = \{en, lain, ray\} \Rightarrow \partial_{e}(\partial_{p}(\mathcal{L})) = \{n\} \Rightarrow \partial_{n}(\partial_{e}(\partial_{p}(\mathcal{L}))) = \{\varepsilon\}$. А значит, $pen \in \mathcal{L}$
    \item $\partial_{p}(\mathcal{L}) = \{en, lain, ray\} \Rightarrow \partial_{e}(\partial_{p}(\mathcal{L})) = \{n\} \Rightarrow \partial_{t}(\partial_{e}(\partial_{p}(\mathcal{L}))) = \emptyset$. Значит, $pet \notin \mathcal{L}$
\end{itemize}

\end{example}

Описанный выше механизм является основной идеей всех алгоритмом, которые стоятся на производных. 

\section{Построение производных}

Концептуально понятно, как выглядят производные, но хотелось бы уметь считать их алгоритмически, причём сохраняя конструктивное представление языка. Например, для КС грамматики, вычисляя её производную, строится другая КС грамматика.

Рассмотрим алгоритм вычисления производной для регулярных языков, которые представлены как регулярные выражения. Пусть $r_{1}$ и $r_{2}$ ---  два регулярных выражения, $a$ --- произвольный терминальный символ. Определим вспомогательную функцию N --- Nullable, которая проверяет язык на содержание в нём пустого слова $\varepsilon$:

\begin{align*}
  N(\varepsilon) &= true \\
  N(a) &= false \\ 
  N(r_{1} \cdot r_{2}) &= N(r_{1}) \land N(r_{2}) \\
  N(r_{1} \mid r_{2}) &= N(r_{1}) \lor N(r_{2}) \\
  N(r^*) &= true
\end{align*}

Теперь мы готовы перейти непосредственно к вычислению производной по символу $a$: 

\begin{align*}
  \partial_{a}(\emptyset) &= \emptyset \\
  \partial_{a}(\varepsilon) &= \emptyset \\ 
  \partial_{a}(b) &= \begin{cases} 
  \emptyset, & \mbox{if } a \neq b \\ 
  \varepsilon, & \mbox{otherwise} \end{cases} \\
  \partial_{a}(r_{1} \cdot r_{2}) &= \begin{cases} 
  \partial_{a}(r_{1}) \cdot r_{2} \mid \partial_{a}(r_{2}) & \mbox{if } N(r_{1}) = \mbox{true} \\ 
  \partial_{a}(r_{1}) \cdot r_{2}, & \mbox{otherwise} \end{cases} \\
  \partial_{a}(r_{1} \mid r_{2}) &= \partial_{a}(r_{1}) \mid \partial_{a}(r_{2}) \\
  \partial_{a}(r^*) &= \partial_{a}(r) \cdot r^*
\end{align*}

Все правила, за исключением, может быть, последнего, достаточно тривиальны. Последнее доказывается по индукции, но особо любопытные могут руками вычислить для первой пары слагаемых и рассмотреть закономерность.

Приводить подобный алгоритм для КС языков мы не будем, но важно понимать, что он есть, и идея не сильно отличается от алгоритма для регулярных языков.

\section{Задача достижимости}
Используя производные, можно решать задачу достижимости. Даны регулярный запрос и граф, для простоты зафиксируем стартовую вершину. Для решения задачи достижимости при помощи производных рекурсивно выполняем следующее:

\begin{enumerate}
    \item При переходе по ребру дифференцируем запрос по метке на нём
    \item Передаём эту производную вдоль ребра в следующую вершину
    \item В этой следующей вершине производная запоминается, если ранее не встречалось регулярное выражение, которое порождается тот же язык (таким образом формируется набор регулярных выражений, порождающих языки, которые уже передовались на вершину). В противном случае -- терминируемся.
\end{enumerate}

После того, как вышеописанный алгоритм завершился для всех вершин, проходим по ним и ищем в их наборах Nullable регулярные выражения, которые и сигнализируют о том, что путь из стартовой веришны в данную вершину существует.

Некоторые замечания:

\begin{itemize}
    \item Если алгоритм запускается не на одной вершине, а сразу на нескольких, передаются вдоль рёбер не просто производные, а пары: (стартовая вершина, производная).
    \item Что мы в общем случае делаем с циклами, чтобы задача считалась алгоритмически? Разбиваем граф на наибольшие по включению компоненты сильной связности с связями между ними, стягиваем компоненты. Тогда граф превращается в дек, в котором проблем с нетерминируемостью нет, а с компонентами сильной связности разбираемся отдельно. Для этого достаточно рассмотреть одну такую компоненту: если есть цикл, то по нему можно ходить, только если в запросе есть применение звезды Клини, и, как нам уже известно, производная запроса данной операции --- конечная конструкция. В итоге мы сможем, переходя по вершинами, прийти к моменту, когда внутри компоненты в наборах производных вершин новых элементов не будет прибавляться, а значит, мы можем закончить алгоритм.
\end{itemize}

Рассмотрим решение задачи достижимости с помощью производных на конкретном примере:

\begin{example}
Даны регулярный запрос $R = a^* \mid a^* \cdot b$ и граф $\mathcal{G}$:
    
    \begin{center}
        \begin{tikzpicture}[shorten >=1pt,on grid,auto]
        \node[state] (q_0)   {$0$};
        \node[state] (q_1) [above right=of q_0] {$1$};
        \node[state] (q_2) [right=of q_0] {$2$};
        \node[state] (q_3) [right=of q_2] {$3$};
        \path[->]
        (q_0) edge  node {a} (q_1)
        (q_1) edge  node {a} (q_2)
        (q_2) edge  node {a} (q_0)
        (q_2) edge[bend left, above]  node {b} (q_3)
        (q_3) edge[bend left, below]  node {b} (q_2);
        \end{tikzpicture}
    \end{center}

Стартовая веришина --- вершина графа $\mathcal{G}$ с индексом 0. Хотим проверить существование путей из вершины 0 до других вершин графа по $R$. Заведем 4 множества, в которых будем хранить регулярные выражения, передаваемые в конкретную вершину: $M_{0}$ --- для вершины с индексом 0, $M_{1}$ --- для вершины с индексом 1 и так далее. 

\begin{enumerate}
    \item Начинаем идти вдоль ребра с меткой $a$ из вершины 0 в 1. Дифференцируем наш запрос по данной метке: $\partial_{a}(R) = \partial_{a}(a^* \mid a^* \cdot b) = \partial_{a}(a^*) \mid \partial_{a}(a^* \cdot b) = \partial_{a}(a) \cdot a^* \mid \partial_{a}(a^*) \cdot b \mid \partial_{a}(b) = a^* \mid a^* \cdot b = R$. Запоминаем эту производную в $M_{1}$:  $M_{1} = \{R\}$ 
    \item Аналогично идём из вершины 1 в вершину 2 по $a$. В итоге в $M_{2}$ также будет находиться R.
    \item В вершине 2 параллельно идём по метке $a$ в 0 и по метке $b$ в вершину с индексом 3. 
    \begin{enumerate}
        \item В первом потоке при переходе по $a$ мы дифференцируем $R$ по a, получаем также $R$, кладем его в $M_{0}$. Пытаясь продолжить эти вычисления в данном потоке, мы обнаружим, что при переходе по $a$ из вершины 0 в вершину 1 наша производная будет снова R, но в множестве $M_{1}$ $R$ уже находится, а значит, данная ветка вычислений терминируется.
        \item При переходе по $b$ диффиринцируем переданную производную $R$: $\partial_{b}(R) = \partial_{b}(a^* \mid a^* \cdot b) = \partial_{b}(a^*) \mid \partial_{b}(a^* \cdot b) = \partial_{b}(a) \cdot a^* \mid \partial_{b}(a^*) \cdot b \mid \partial_{b}(b) = \partial_{b}(b) = \varepsilon$. Запоминаем производную в $M_{3}$:  $M_{3} = \{\varepsilon\}$. Дифференцируем её по $b$ при переходе из 3 в 2. Кладём в $M_{2}$ $\partial_{b}(\varepsilon) = \emptyset$. Tаким oбразом, $M_{2} = \{R, \emptyset\}$. На данном этапе можно терминирироваться, т.к. для достижимости нам интересны регулярные выражения, порождающий $\varepsilon$, но дифференцируя $\emptyset$ по каким-либо меткам, мы снова получаем $\emptyset$. При "честном" вычислении, в нашем случае пустое множество будет добавлено во все множества $M_{i}$, и уже после этого вычисление завершится.    
    \end{enumerate}
    \item Теперь у нас есть сформированные множества производных для всех вершин и мы можем говорить о достижимости до данных вершин из стартовой. Рассмотрим $M_{0}$, $M_{0} = \{R\}$. Содержит ли $R$ $\varepsilon$? Посчитаем для этого нашу функцию Nullable: $N(R) = N(a^* \mid a^* \cdot b) = N(a^*) \mid N(a^* \cdot b) = true \mid N(a^* \cdot b) = true$. Из чего следует, что существует путь из вершины 0 в вершину 0. Аналогичным способом заключаем, что веришны 2 и 3 достижимы из вершины 0 (т.к. в их множества также содержится $R$). Рассмотрим $M_{3} = \{\varepsilon\}$. $N(\varepsilon) = true$, значит, вершина  3 тоже достижима из 0. Алгоритм завершён.
\end{enumerate}

Таким образом, выполнив указанный выше алгоритм для данного примера, мы определили, что из стартовой вершины с индексом 0 достижимы все вершины графа $\mathcal{G}$.
\end{example}

\section{Парсинг на производных}

Статьи~\cite{DBLP:journals/corr/abs-1010-5023,Adams:2016:CPP:2908080.2908128,Might:2011:PDF:2034574.2034801,andersenparsing}
Реализации.
На Scala~\footnote{\url{https://github.com/djspiewak/parseback}}, на Racket~\footnote{\url{https://bitbucket.org/ucombinator/derp-3/src/86bca8a720231e010a3ad6aefd1aa1c0f35cbf6b/src/derp.rkt?at=master&fileviewer=file-view-default}}.

\section{Адаптация для КС запросов}

Для регулярных запросов над графами~\cite{Nole:2016:RPQ:2949689.2949711}.
Хорошо работают в распределённых системах, в которых реализовван параллелизм уровня вершин. 
Например Google Pregel.



\section{Вопросы и задачи}
\begin{enumerate}
  \item Предъявить несколько выводов для одной цепочки.
  \item Построить выводы
  \item Построить деревья вывода !!! Перенести из раздела про SPPF
\end{enumerate}


\input{CFPQ_to_Datalog}
\chapter{Многокомпонентные контекстно-свободные языки}

Общая теория. Лпределение, свойства, классы.

Про MIX и $O_n$

\section{Поиск путей с ограничениями в терминах многокомпонентных контекстно-свободных языков}

В статанализе --- ещё одна аппроксимация.

Алгоритм на матрицах
\chapter{Конъюнктивные и булевы грамматики}

Впервые конъюнктивные и булевы грамматики были предложены Александром Охотиным~\cite{DBLP:journals/jalc/Okhotin01,Okhotin:2003:BG:1758089.1758123}. Дадим определение конъюнктивной грамматики.

\begin{definition}
    \textit{Конъюнктивной грамматикой} называется $G = (\Sigma,N,P,S)$, где:
    \begin{itemize}
        \item $\Sigma$ и $N$ --- дизъюнктивные конечные непустые множества терминалов и нетерминалов.
        \item $P$ --- конечное множество продукций, каждая вида
        \[
        A\rightarrow \alpha_1\&...\&\alpha_n
        \]
        ,где $A \in N,n \geq 1$ и $\alpha_1,...,\alpha_n \in (\Sigma \cup N)^*$.
        \item $S \in N$  --- стартовый нетерминал.
    \end{itemize}
\end{definition}

Конъюнктивная грамматика генерирует строки, выводя их из начального символа, так же, как это происходит в контекстно-свободных грамматиках в параграфе~\ref{CFG}. Промежуточные строки, используемые в процессе вывода, являются формулами следующего вида:

\begin{definition}\label{Definition of conjunctive formula}
    Пусть $G = (\Sigma,N,P,S)$ --- конъюнктивная грамматика. Множество конъюнктивных формул $ \mathcal{F}$ определяется индуктивно:
    \begin{itemize}
        \item Пустая строка $\varepsilon$ --- конъюнктивная формула. 
        \item Любой символ из $(\Sigma \cup N)$ --- формула.
        \item Если $\mathcal{A}$ и $\mathcal{B}$ непустые формулы, тогда $\mathcal{AB}$ --- формула.
        \item Если $\mathcal{A}_1,\ldots,\mathcal{A}_n$ $(n \geqslant 1)$ --- формула, тогда $(\mathcal{A}_1\&\ldots\&\mathcal{A}_n)$ --- формула.
    \end{itemize}
\end{definition}

\begin{definition}
    Пусть $G = (\Sigma,N,P,S)$ --- конъюнктивная грамматика. Аналогично определению отношения непосредственной выводимости в контекстно-свободной грамматике~\ref{def derivability in CFG} определим $\xRightarrow[G]{}$ как отношение непосредственной выводимости на множестве конъюнктивных формул.
    \begin{itemize}
        \item Любой нетерминал в любой формуле может быть перезаписан телом любого правила для этого терминала заключенным в скобки. То есть для любых $s^{'},s^{''} \in (\Sigma \cup N \cup \{(, \&, )\})^*$ и $A\in N$, таких что $s^{'}As^{''}$ --- формула, и для всех правил вида $A \rightarrow \alpha_1\&\ldots\&\alpha_n \in P$, имеем $s^{'}As^{''}\xRightarrow[G]{}s^{'}(\alpha_1\&\ldots\&\alpha_n)s^{''}$. 
        \item Если формула содержит подформулу в виде конъюнкции одной или нескольких одинаковых терминальных строк, заключенных в скобки, тогда подформула может быть перезаписана терминальной строкой без скобок. То есть для любых $s^{'},s^{''} \in (\Sigma \cup N \cup \{(, \&, )\})^*$, $(n \geqslant 1)$ и $w \in \Sigma^*$, таких что $s^{'}(w\&\ldots\&w)s^{''}$ --- формула, имеем $s^{'}(w\&\ldots\&w)s^{''}\xRightarrow[G]{}s^{'}ws^{''}$.
    \end{itemize}
    Как и в случае контекстно-свободной грамматики обозначим $\xRightarrow[G]{}^*$ рефлексивное транзитивное замыкание отношения $\xRightarrow[G]{}$.
\end{definition}

\begin{definition}
    Пусть $G = (\Sigma,N,P,S)$ --- конъюнктивная грамматика. Язык, порождаемый формулой, это множество всех терминальных строк выводимых из этой формулы: $L_{G}(\mathcal{A}) = \{w\in\Sigma^* \mid \mathcal{A} \xRightarrow[G]{}^*w\}$. Очевидно, что язык порождаемый грамматикой, это язык порождаемый стартовым нетерминалом $S$ : $L(G) = L_{G}(S) = L(S)$.
\end{definition}

\begin{theorem}\label{Theorem language generated by a formula}
    Пусть $G = (\Sigma,N,P,S)$ --- конъюнктивная грамматика. Пусть $\mathcal{A}_1,\ldots,\mathcal{A}_n,\mathcal{B}$ --- формулы, $A \in N$, $a \in \Sigma$. Тогда,
    \begin{enumerate}
        \item $L(\varepsilon) = \{\varepsilon\}$.
        \item $L(a) = \{a\}$.
        \item $L(A) = \bigcup_{A \rightarrow \alpha_1\&\ldots\&\alpha_n \in P} L((\alpha_1\&\ldots\&\alpha_m))$.
        \item $L(\mathcal{AB}) = L(\mathcal{A})*L(\mathcal{B})$
        \item $L((\mathcal{A}_1\&\ldots\&\mathcal{A}_n)) = \bigcap_{i = 1}^{n}L(\mathcal{A}_i)$.
    \end{enumerate}
\end{theorem}

Теорема~\ref{Theorem language generated by a formula} уже подразумевает интерпретацию грамматики как системы уравнений. Используем математический подход, чтобы лучше охарактеризовать конъюнктивные языки с помощью систем уравнений.

\begin{definition}[Выражение]
    Пусть $\Sigma$ конечный непустой алфавит. Пусть $X = \{X_1,\ldots,X_N\}$ вектор переменных. Выражение над алфавитом $\Sigma$, зависящее от переменных $X$, определяется индуктивно:
    \begin{itemize}
       \item $\varepsilon$ --- выражение.
       \item Любой символ $a\in\Sigma$ --- выражение.
       \item Любая переменная $X_i\in X$ --- выражение.
       \item Если $\phi_1$ и $\phi_2$ выражения, то $\phi_1\phi_2, (\phi_1\mid\phi_2), (\phi_1\&\phi_2)$ также выражения.
    \end{itemize}
    Заметим, что любая формула, в терминах определения~\ref{Definition of conjunctive formula}, является выражением, где нетерминалы формулы это переменные выражения. С другой стороны, любое выражение, не содержащее дизъюнкции, формула.
\end{definition}

Предположим, что переменные $X_i$ приняли в качестве значений слова из языка над алфавитом $\Sigma$. Определим значение всего выражения.

\begin{definition}[Значение выражения]\label{Value of conjunctive expression}
    Пусть $L = (L_1,\ldots,L_n) (L_i \subseteq \Sigma^*)$ вектор из $n$ языков над $\Sigma$, где $n \geqslant 1$. Пусть $\phi$ выражение над $\Sigma$, зависящее от переменных $X_1,\ldots,X_n$. Значение выражения $\phi$ на векторе $L$ --- это язык над тем же алфавитом $\Sigma$. Обозначим его $\phi(L)$ и определим индуктивно на структуре выражения:
    \begin{itemize}
       \item $\varepsilon(L) = \{\varepsilon\}$.
       \item $a(L) = \{a\}$ для любого $a\in\Sigma$.
       \item $X_i(L) = L_i$ для любого $X_i \in X$.
       \item $\phi_1\phi_2 = \phi_1(L) * \phi_2(L), (\phi_1\mid\phi_2)(L) = \phi_1(L) \cup \phi_2(L), (\phi_1\&\phi_2)(L) = \phi_1(L) \cap \phi_2(L)$ для любых выражений $\phi_1$ и $\phi_2$.
    \end{itemize}
\end{definition}

Обобщим определение~\ref{Value of conjunctive expression} на случай вектора выражений.

\begin{definition}[Значение вектора выражений]
    Пусть $L = (L_1,\ldots,L_n) (L_i \subseteq \Sigma^*)$ вектор из $n$ языков над $\Sigma$, где $n \geqslant 1$. Пусть $\phi_1,\ldots,\phi_m$ выражения над $\Sigma$, зависящее от переменных $X_1,\ldots,X_n$. Значение вектора выражений $P = (\phi_1,\ldots,\phi_m)$ на векторе $L$ --- это вектор языков $P(L) = (\phi_1(L),\ldots,\phi_m(L))$ над тем же алфавитом $\Sigma$. 
\end{definition}

Зададим частичный порядок относительно включения $``\preccurlyeq"$ на множестве языков и расширим его на вектора языков длины $n$: $(L_1^{'},\ldots,L_n^{'})\preccurlyeq(L_1^{''},\ldots,L_n^{''})$ если и только если $L_i^{'} \subseteq L_i^{''}$ для любого $1\leqslant i \leqslant n$

\begin{definition}\label{Definition a conjuctive system of equations}
   $X = P(X)$ система уравнений над алфавитом $\Sigma$ и $X = \{X_1,\ldots,X_n\}$, где $P = (\phi_1,\ldots,\phi_n)$ вектор выражений над алфавитом $\Sigma$, зависящий от $X$.
   
   Вектор языков $L = (L_1,\ldots,L_n)$ является решением системы уравнений если $L = P(L)$.
   
   Наименьшее решение $L$ это вектор языков, такой что для любого другого сравнимого вектора языков $L^{'}$ выполняется $L \preccurlyeq L^{'}$.
\end{definition}

Заметим, что оператор $P$ на множестве $2^{\Sigma}\times\ldots\times2^{\Sigma}$, что решение системы~\ref{Definition a conjuctive system of equations} это неподвижная точка $P$ и что наименьшее решение системы это наименьшая неподвижная точка оператора $P$.

\begin{theorem}\label{Theorem of a least fixed point solution}
    Для любой системы из определения~\ref{Definition a conjuctive system of equations} с переменными $X_1,\ldots,X_n$, оператор $P = (\phi_1,\ldots,\phi_n)$ имеет наименьшую неподвижную точку $L = (L_1,\ldots,L_n) = \lim_{i\to\infty}P^{i}\underbrace{(\varnothing,\ldots,\varnothing)}_n$.
\end{theorem}

Приведем пример конъюнктивной грамматики.

\begin{example}[Пример конъюнктивной грамматики]
    Следующая конъюнктивная грамматика $G$ порождает язык $\{a^nb^nc^n\mid n \geq 0\}$:
    
    \begin{align*}
    1.\ S   &\to A B \& D C \\
    2.\ A  &\to a A \mid \varepsilon \\ 
    3.\ B &\to b B c \mid \varepsilon \\
    4.\ C   &\to c C \mid \varepsilon \\ 
    5.\ D   &\to aDb \mid \varepsilon
    \end{align*}
    
    Легко видеть, что $L(AB) = \{a^ib^jc^k\mid j = k\}$ и $L(DC) = \{a^ib^jc^k\mid i = j\}$. Тогда $L(S) = L(AB) \cap L(DC) = \{a^nb^nc^n\mid n \geq 0\}$. 
    
    В этой грамматике строка $abc$ может быть получена следующим образом. Для начала представим грамматику в виде системы уравнений:
    \begin{align*}
    S &= A B \cap D C \\ 
    A &= \{a\}A \cup \varepsilon \\ 
    B &= \{b\}B\{c\} \cup \varepsilon \\
    C &= \{c\}C \cup \varepsilon \\ 
    D &= \{a\}D\{b\} \cup \varepsilon
    \end{align*}
    Используя теорему~\ref{Theorem of a least fixed point solution}, будем итеративно вычислять $P^{i}\underbrace{(\varnothing,\ldots,\varnothing)}_5$. На каждом шаге будем подставлять все терминальные строки из языков, порожденных нетерминалами на предыдущем шаге, в соответствующие нетерминалы правой части каждого уравнения и записывать получившиеся терминальные строки в языки нетерминалов текущего шага. Продолжаем до тех пор пока язык, порождаемый нетерминалом $S$, не будет содержать терминальную строку $``abc''$.
    \begin{enumerate}
        \item На начальном этапе имеем $P^{0}(\varnothing,\ldots,\varnothing) = (S: \varnothing, A: \varnothing, B: \varnothing, C: \varnothing, D: \varnothing)$ 
        \item Подставляем в первое уравнение терминальные строки из шага 1 в соответствующие нетерминалы, т.е. 
        \begin{align*}
            S:  \varnothing &= \varnothing\varnothing \cap \varnothing\varnothing \\ 
            A: \{\varepsilon\} &= \{a\}\varnothing \cup \{\varepsilon\} \\ 
            B: \{\varepsilon\} &= \{b\}\varnothing\{c\} \cup \{\varepsilon\} \\
            C: \{\varepsilon\} &= \{c\}\varnothing \cup \{\varepsilon\} \\ 
            D: \{\varepsilon\} &= \{a\}\varnothing\{b\} \cup \{\varepsilon\}
        \end{align*}
        В конце итерации получаем $P^{1}(\varnothing,\ldots,\varnothing) = (S: \varnothing, A: \{\varepsilon\}, B: \{\varepsilon\}, C: \{\varepsilon\}, D: \{\varepsilon\})$
        \item Делаем еще одну итерацию,
        \begin{align*}
            S:  \{\varepsilon\} &= \{\varepsilon\}\{\varepsilon\} \cap \{\varepsilon\}\{\varepsilon\} \\ 
            A: \{a, \varepsilon\} &= \{a\}\{\varepsilon\} \cup \{\varepsilon\} \\ 
            B: \{bc, \varepsilon\} &= \{b\}\{\varepsilon\}\{c\} \cup \{\varepsilon\} \\
            C: \{c, \varepsilon\} &= \{c\}\{\varepsilon\} \cup \{\varepsilon\} \\ 
            D: \{ab, \varepsilon\} &= \{a\}\{\varepsilon\}\{b\} \cup \{\varepsilon\}
        \end{align*}
        В конце итерации получаем $P^{2}(\varnothing,\ldots,\varnothing) = (S: \{\varepsilon\}, A: \{a, \varepsilon\}, B: \{bc, \varepsilon\}, C: \{c, \varepsilon\}, D: \{ab, \varepsilon\})$
        \item Еще одна итерация,
        \begin{align*}
            S:  \{\fbox{abc}, \varepsilon\} &= \{a, \varepsilon\}\{bc, \varepsilon\} \cap \{ab, \varepsilon\}\{c, \varepsilon\} \\ 
            A: \{a, aa, \varepsilon\} &= \{a\}\{a, \varepsilon\} \cup \{\varepsilon\} \\ 
            B: \{bc, bbcc, \varepsilon\} &= \{b\}\{bc, \varepsilon\}\{c\} \cup \{\varepsilon\} \\
            C: \{c, cc, \varepsilon\} &= \{c\}\{c, \varepsilon\} \cup \{\varepsilon\} \\ 
            D: \{ab, aabb, \varepsilon\} &= \{a\}\{ab, \varepsilon\}\{b\} \cup \{\varepsilon\}
        \end{align*}
        В конце итерации получили $P^{3}(\varnothing,\ldots,\varnothing) = (S: \{\fbox{abc}, \varepsilon\}, A: \{a, aa, \varepsilon\}, B: \{bc, bbcc, \varepsilon\}, C: \{c, cc, \varepsilon\}, D: \{ab, aabb, \varepsilon\})$. Заметим, что терминальная строка $``abc"$ появилась в языке, который порождает стартовый нетерминал $S$. Т.е. терминальная строка $``abc"$ выводима в грамматике $G$, что и требовалось показать.
    \end{enumerate}
    
    Заметим, что строку $``abc"$ также можно получить применением правил вывода. здесь цифра над стрелкой соответствует номеру примененного правила. 
    \begin{align*}
        S &\xRightarrow{1}(AB\&DC) \\
        &\xRightarrow{2}(aAB\&DC) \xRightarrow{2} (a\varepsilon B\&DC) \\
        &\xRightarrow{3}(abBc\&DC) \xRightarrow{3}(ab\varepsilon c\&DC) \\
        &\xRightarrow{5}(abc\&aDbC) \xRightarrow{5}(abc\&a\varepsilon bC) \\
        &\xRightarrow{4}(abc\&abcC) \xRightarrow{4}(abc\&abc\varepsilon) \\
        &\Rightarrow(abc\&abc) \Rightarrow abc
    \end{align*}
\end{example}

\begin{example}
    Конъюнктивная грамматика $G$ для языка $L = \{wcw \mid w \in \{a, b\}^*\}$:
    \begin{align*}
    S &\to C \& D \\ 
    C &\to aCa \mid aCb \mid bCa \mid bCb \mid c \\ 
    D &\to aA\&aD \mid bB\&bD \mid cE \\
    A &\to aAa \mid aAb \mid bAa \mid bAb \mid cEa \\
    B &\to aBa \mid aBb \mid bBa \mid bBb \mid cEb \\
    E &\to aE \mid bE \mid \varepsilon
    \end{align*}
\end{example}

Подробнее о конъюнктивных грамматиках можно прочитать в статьях~\cite{DBLP:journals/jalc/Okhotin01, Okhotin2002, DBLP:journals/tcs/Okhotin03a, f60a33d409364914be560cac0e54b12c}.

Дадим определение булевой грамматики.

\begin{definition}
    \textit{Булевой грамматикой} называется $G = (\Sigma,N,P,S)$, где:
    \begin{itemize}
        \item $\Sigma$ и $N$ --- дизъюнктивные конечные непустые множества терминалов и нетерминалов.
        \item $P$ --- конечное множество продукций, каждая вида
        \[
        A\rightarrow \alpha_1\&...\&\alpha_m\&\neg\beta_1\&...\&\neg\beta_n
        \]
        ,где $A \in N, m, n >=0, m+n \geq 1$ и $\alpha_i,\beta_j \in (\Sigma \cup N)^*$.
        \item $S \in N$  --- стартовый нетерминал.
    \end{itemize}
\end{definition}

Приведем пример булевой грамматики.

\begin{example}
    Следующая булева грамматика порождает язык  $\{a^mb^nc^n\mid m,n \geq 0, m \neq n \}$:
    
    \begin{align*}
    S   &\to A B \& \neg D C \\ 
    A  &\to a A \mid \varepsilon \\ 
    B &\to b B c \mid \varepsilon \\
    C   &\to c C \mid \varepsilon \\ 
    D   &\to aDb \mid \varepsilon
    \end{align*}
    
    Очевидно, что $L(AB) = \{a^mb^nc^n\mid m,n \in \mathbb{N}\}$ и $L(DC) = \{a^nb^nc^m\mid m,n \in \mathbb{N}\}$. Тогда $L(AB)\cap\overline{L(DC)} = \{a^mb^nc^n\mid m,n \geq 0, m \neq n \}$.
\end{example}

Подробнее о булевых грамматиках можно прочитать в статьях~\cite{Okhotin:2003:BG:1758089.1758123,Okhotin:2014:PMM:2565359.2565379}.

Определим бинарную нормальную форму конъюнктивной грамматики.
\begin{definition}[Бинарная нормальная форма]
    Конъюнктивная грамматика $G = (\Sigma, N, P, S)$ находится в бинарной нормальной форме, если каждое правило из P имеет вид,
    \begin{itemize}
        \item $A \rightarrow B_1 C_1 \& \ldots\& B_m C_m$, где $m \geqslant 1; A,B_i,C_i \in N$.
        \item $A \rightarrow a$, где $A \in N, a \in \Sigma$.
        \item $S \rightarrow \varepsilon$, если только $S$ не содержится в правой части всех правил.
    \end{itemize}
\end{definition}

\begin{theorem}\label{Binary normal form conjunctive grammar theorem}
    Для каждой конъюнктивной грамматики $G$ можно построить конъюнктивную грамматику в бинарной нормальной форме $G^{'}$, такую что $L(G) = L(G^{'})$.
\end{theorem}
Доказательство теоремы~\ref{Binary normal form conjunctive grammar theorem} описано в статье~\cite{DBLP:journals/jalc/Okhotin01}.



\section{Вопросы и задачи}
\begin{enumerate}
  \item !!! 
  \item !!!
\end{enumerate}

%\section{Conclusion}
We describe the modifications of the proposed approach~\cite{grigorevcomposition} for biological sequences analysis using the combination of formal grammars and neural networks.
We show that it is possible to improve the quality of the solution by representing parsing result as an image and handling it by using convolutional layers while processing it with a neural network.
Also, we provide a technique that removes the parsing step from the trained model use and allows to run models on the original RNA sequences.
As a result, the performance of the solution is significantly improved.
We demonstrate the applicability of the proposed modifications for real-world problems\footnote{
Project description is available at the project page: \url{https://research.jetbrains.org/groups/plt\_lab/projects?project\_id=43}.
Source code and documentation are published at GitHub: \url{https://github.com/LuninaPolina/SecondaryStructureAnalyzer}. Access date: 07.03.2020}.

We can provide several directions for future research.
First of all, it is necessary to investigate the applicability of the proposed approach for other sequences processing tasks such as 16s rRNA processing and chimeric sequences filtration.

Another possible application is a secondary structure prediction.
We plan to investigate the possibility of creating a network which generates the most possible contact map for the given sequence.
It is necessary to compare this approach with both classical approaches and tools for secondary structure prediction~\cite{jabbari2014fast,jabbari2007hfold,sato2009centroidfold,10.1093/bioinformatics/btr215} and artificial neural network-based ones~\cite{Lu2019,Singh2019}.

The image-based model demonstrates a higher quality.
We believe that it is caused by a better locality of features.
If so, it should be possible to create a deep convolutional network for secondary structure analysis: further investigation is needed.

Finally, it is important to find a theoretical base for grammar tuning.
It is important to adopt the theoretical results on secondary structure description by using formal grammar, such as~\cite{MQbioinformatics19} to find the optimal grammar for our approach.



\bibliographystyle{abbrv}
\bibliography{Formal_lang_CFPQ_course_notes}


\end{document}
