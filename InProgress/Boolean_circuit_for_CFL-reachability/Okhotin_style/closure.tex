\section{Closure properties of languages with polynomial rational indices}
\label{sec:closure}
Given a context-free language $L$ with the polynomial rational index, it is interesting to find which language operations preserve this property.  Boasson et al. \cite{RatBasic} give the following useful relations for polynomial indices of two languages $L$ and $L'$.
\begin{lemma}[\cite{RatBasic}]
\label{lem:closure}
Context-free languages with polynomial rational indices are closed under intersection with a regular language, union, concatenation, the Kleene star, homomorphism and inverse homomorphism. More precisely,
\begin{itemize}
\item $\rho_{L \cup L'}(n) \le  \max{(\rho_L(n), \rho_{L'}(n))} $
\item $\rho_{LL'}(n) \le \rho_L(n) + \rho_{L'}(n)$
\item $\rho_{L^{*}}(n) \le n(\rho_L(n))$
\item $\rho_{L \cap R}(n) \le \rho_L(nm)$, where $R$ is a regular language recognised by an $m$-state automaton
\item $\rho_{h(L)}(n) \le \rho_L(n)$ and $\rho_{h^{-1}(L)}(n) < n(\rho_L(n) +1)$, where $h: \Sigma^* \rightarrow \Delta^*$ is a homomorphism
\item $\rho_{\tau(L)}(n) \le (mn + 1)\rho_L(mn)$, where $\tau$ is a rational transduction and $m$ is some integer.
\end{itemize}
\end{lemma}
 From the relations above it is easy to see that the family of context-free languages with polynomial rational indices is a full trio. Every full trio is closed under left and right quotient with regular languages, prefix, suffix, infix, and outfix \cite{GinsburgAlgebraic}. Obviously, CFLs with the polynomial rational indices languages are closed under reversal.  Next we show that context-free languages with the polynomial rational indices are closed under insertion of a regular language.
\begin{theorem}
Context-free languages with the polynomial rational indices are closed under the insertion of a regular language. 
\\Particularly, $\rho_{L_{INSERT(K)}}(n) \le (mn + 1)\rho_L(mn)$, where $m$ is the number of states in the NFA accepting $K$.
\end{theorem}
\begin{proof}
 Let $L$ be a language with the polynomial rational index over an alphabet $\Sigma$ and $K$ be a regular language over an alphabet $\Delta$, where an NFA $M(K)$ with $m$ states is an NFA accepting $K$.  Define a homomorphism $h: \Delta^*  \rightarrow \bar{\Delta}^{*}$, such that $h(a)=\bar{a}$,  $\forall a \in \Delta$. In simple words, $h$ makes all symbols from $\Delta$ ``marked''. Then, by defining a homomorphism $g$, such that $g(a) = \varepsilon$, $\forall a \in \bar{\Delta}$ ($g$ erases the symbols of $\bar{\Delta}$), one can insert an arbitrary number of symbols from $\bar{\Delta}$ into strings in $L$ using an inverse homomorphism $g^{-1}$. To obtain a string from $L_{INSERT(K)}$ it is left to intersect $g^{-1}(L)$ with a regular set $K'$ containing strings in the form $xyz$, where $x, z \subseteq \Sigma^{*}$ and $y \in h(K)$. Then ``marked'' symbols from  $\bar{\Delta}$ is unmarked by a homomorphism $\phi:  \bar{\Delta^*}  \rightarrow \Delta^{*}$, where $\phi(\bar{a}) = a$, $\forall \bar{a} \in \bar{\Delta}$. Finally, every word $w' \in L_{INSERT(K)}$ can be written as $\phi(g^{-1}(w) \cap K') = \tau(w)$, where $w \in L$ and $\tau$ is a rational transduction. By Lemma~\ref{lem:closure}, languages with the polynomial rational indices are closed under rational trunsductions, so $L_{INSERT(K)}$ has the polynomial rational index. An NFA $M(K')$ can be easily constructed from $M(K)$ and has $O(m)$ states. Then the value of the rational index $\rho_{L_{INSERT(K)}}(n) \le (mn + 1)\rho_L(mn)$.
\end{proof}



Using closure properties, it is easier to find new subclasses of context-free languages for which the CFL-reachability problem is in NC.
\begin{example}[Metalinear languages \cite{metalinear}.]
Let $G = (\Sigma, N, P, S)$ be a context-free grammar. $G$ is \textit{metalinear} if all productions of $P$ are of the following forms:
\begin{enumerate}
\item $S \rightarrow A_1A_2...A_k$, where $A_i \in N \setminus \{S\}$
\item $A \rightarrow u$, where $A \in N \setminus \{S\}$ and $u \in (\Sigma^*((N \setminus \{S\}) \cup {\varepsilon})\Sigma^*)$
\end{enumerate}


The width of a metalinear grammar is $max\{k\vert S \rightarrow A_1A_2...A_k \}$. Metalinear languages of width 1 are obviously linear languages. It is easy to see that every metalinear language is a union of concatenations of $k$ linear languages. Linear languages have polynomial rational index,  CFLs with the polynomial rational index are closed under concatenation and union, so metalinear languages have the polynomial rational index and, hence, is in NC.
\end{example}

