\section{Context-free path querying by Kronecker product}


In this section, we introduce the algorithm for context-free path querying which is based on the Kronecker product of Boolean matrices.
The algorithm solves all-pairs CFPQ problem with all-path semantics (according to~\cite{hellingsPathQuerying}).
The algorithm works in the following two steps.
\begin{enumerate}
\item \emph{Index creation}.
 During this step, the algorithm computes an index that contains the information necessary to restore paths for the given pairs of vertices.
 This index can be used to solve the reachability problem without extracting paths.
 Note that the index is finite even when the set of paths is infinite.
\item \emph{Paths extraction}.
All paths for the given pair of vertices can be enumerated by using the index.
Since the set of paths can be infinite, all paths cannot be enumerated explicitly, thus advanced techniques such as lazy evaluation are required for the implementation.
Nevertheless, a single path can always be extracted with standard techniques.
\end{enumerate}

In the following subsections, we describe these steps, prove the correctness of the algorithm, and provide time complexity estimations.
For the first step, we start by introducing a naive algorithm.
After that, we show how to achieve cubic time complexity by using a dynamic transitive closure algorithm and shave off a logarithmic factor to achieve the best known time complexity for the CFPQ problem.
We finish by providing a step-by-step example of query evaluation with the proposed algorithm.

\subsection{Index Creation Algorithm}

The \textit{index creation} algorithm outputs the final adjacency matrix for the input graph with all pairs of vertices which are reachable through some nonterminal in the input grammar $G$, as well as the index matrix, which is to be used to extract paths in the \textit{path extraction} algorithm.

The algorithm is based on the generalization of the FSM intersection for an RSM,  and the edge-labeled directed input graph.
Since the RSM is composed as a set of FSMs, it could be easily presented as an adjacency matrix for some graph over labels set .
As shown in the Def.~\ref{def:bool:product}, we can apply Kronecker product from Boolean matrices to \textit{intersect} the RSM and the input graph to some extent.
But the RSM contains the nonterminal symbols with additional \textit{recursive calls} logic, which requires \textit{transitive closure} step to extract such symbols.

The core idea of the algorithm comes from Kronecker product and transitive closure.
The algorithm boils down to the iterative Kronecker product evaluation for the RSM $R$ adjacency matrix $\mathcal{M}_1$ and the input graph $\mathcal{G}$ adjacency matrix $\mathcal{M}_2$, followed by transitive closure, extraction of nonterminals and updating the graph adjacency matrix $\mathcal{M}_2$.
Listing~\ref{tensor:cfpq:cubic} shows main steps of the algorithm.
\begin{algorithm}[h]
\floatname{algorithm}{Listing}
\begin{algorithmic}[1]
\footnotesize
\caption{Kronecker product based CFPQ using dynamic transitive closure}
\label{tensor:cfpq:cubic}
\Function{contextFreePathQuerying}{G, $\mathcal{G}$}
    % Input data preparation
   \State{$n \gets$ The number of nodes in $\mathcal{G}$}
    \State{$R \gets$ Recursive automata for $G$}
    \State{$\mathcal{M}_1 \gets$ Boolean adjacency matrix for $R$}
    \State{$\mathcal{M}_2, \mathcal{A}_2 \gets$ Boolean adjacency matrix for $\mathcal{G}$}
    %\State{$M_3 \gets$ The empty matrix}
    \State{$C_3 \gets$ The empty matrix of size $|R|n \times |R|n$}
    % Eps-transition handling for graph
    \For{$s \in 0..dim(\mathcal{M}_1)-1$}
        \For{$S \in \textit{getNonterminals}(R,s,s)$}
            \For{$i \in 0..dim(\mathcal{M}_2)-1$}
                \State{$M^S_2[i,i] \gets 1 $}
            \EndFor
        \EndFor
    \EndFor
    \While{$\mathcal{M}_2$ is changing}
        \State{$M_3' \gets \bigvee_{M^S \in \mathcal{M}_1 \otimes \mathcal{A}_2} M^S$}
        %\State{$\mathcal{A}_2 \gets$ The empty matrix of size $n \times n$}
        \State{$\mathcal{A}_2 \gets$ The empty matrix}
        \State{$C_3' \gets$ The empty matrix of size $|R|n \times |R|n$}
        %\Comment{Evaluate Kroncker product}
        \For{$(i,j) \mid M_3'[i,j] \neq 0$}
            %\State{$C_3'[i,j] \gets 1$}
            \State{$C_3' \gets \textit{add}(C_3, C_3', i, j)$}
            \Comment{Updating the transitive closure}
            \State{$C_3 \gets C_3 + C_3'$}
        \EndFor
        %\State{$n \gets$ dim($M_3)$}
        %\Comment{Matrix $M_3$ size = $n \times n$}
        % Add non-terminals (possibly new)
        \For{$(i,j)\ |\ C_3'[i,j] \neq 0$}
                \State{$s, f \gets \textit{getStates}(C_3',i,j)$}
                \State{$x, y \gets \textit{getCoordinates}(C_3',i,j)$}
                \For{$S \in \textit{getNonterminals}(R,s,f)$}
                    \State{$M^S_2[x,y] \gets 1$}
                    \State{$A^S_2[x,y] \gets 1$}
                \EndFor
        \EndFor
    \EndWhile
\State \Return $\mathcal{M}_2, C_3$
\EndFunction
% \Function{add}{$C, C', i, j$}
%     \State{$C'[i,j] \gets {1}$}
%     \For{$(u,v) \mid C[u,i] = C[j,v] = 1, C[u,j] = C[u,v] = 0$}
%         \State{$C'[u,v] \gets {1}$}
%     \EndFor
%     \State \Return{$C'$}
% \EndFunction
\Function{getStates}{$C, i, j$}
    \State{$r \gets dim(\mathcal{M}_1)$}
    \Comment{$\mathcal{M}_1$ is adjacency matrix for $R$}
    \State \Return{$\left\lfloor{i / r}\right\rfloor, \left\lfloor{j / r}\right\rfloor$}
\EndFunction
\Function{getCoordinates}{$C, i, j$}
    \State{$n \gets dim(\mathcal{M}_2)$}
    \Comment{$\mathcal{M}_2$ is adjacency matrix for $\mathcal{G}$}
    \State \Return{$i \bmod n, j \bmod n$}
\EndFunction
\end{algorithmic}
\end{algorithm}
\subsubsection{Application of Dynamic Transitive Closure}
It is easy to see that the most time-consuming steps of the algorithm are the Kronecker product and transitive closure computations.
Note that the adjacency matrix $\mathcal{M}_2$ is always changed incrementally i. e. elements (edges) are added to $\mathcal{M}_2$ (and are never deleted from it) at each iteration of the algorithm.
So it is not necessary to recompute the whole product or transitive closure if an appropriate data structure is maintained.

To compute the Kronecker product, we employ the fact that it is left-distributive.
Let $\mathcal{A}_2$ be a matrix with newly added elements and $\mathcal{B}_2$ be a matrix with the all previously found elements, such that $\mathcal{M}_2 = \mathcal{A}_2 + \mathcal{B}_2$.
Then by the left-distributivity of the Kronecker product we have $\mathcal{M}_1 \otimes \mathcal{M}_2 = \mathcal{M}_1 \otimes (\mathcal{A}_2 + \mathcal{B}_2) = \mathcal{M}_1\otimes \mathcal{A}_2 + \mathcal{M}_1 \otimes \mathcal{B}_2$.
Note that $\mathcal{M}_1 \otimes \mathcal{B}_2$ is known and is already in the matrix $\mathcal{M}_3$ and its transitive closure also is already in the matrix $C_3$, because it has been calculated at the previous iterations, so it is left to update some elements of $\mathcal{M}_3$ by computing $\mathcal{M}_1\otimes \mathcal{A}_2$.


The fast computation of transitive closure can be obtained by using incremental dynamic transitive closure technique. Now we describe the function $add$ from Listing \ref{tensor:cfpq:cubic}. Let $C_3$ be a transitive closure matrix of the graph $G$ with $n$ vertices. We use an approach by~\cite{IBARAKI198395} to maintain dynamic transitive closure. The key idea of their algorithm is to recalculate reachability information only for those vertices, which become reachable after insertion of the certain edge. We have modified it to achieve a logarithmic speed-up.


For each newly inserted edge $(i, j)$ and every node $u \neq j$ of $G$ such that $C_3[u, i] = 1$ and $C_3[u, j]=0$, one needs to perform operation $C_3[u,v] = C_3[u, v] \wedge C_3[j, v]$ for every node $v$, where $1 \wedge 1 = 0 \wedge 0 = 1 \wedge 0 = 0$ and $0 \wedge 1 = 1$.
Notice that these operations are equivalent to the element-wise (Hadamard) product of two vectors of size $n$, where multiplication operation is denoted as $\wedge$. To check whether $C_3[u, i] = 1$ and $C_3[u, j]=0$ one needs to multiply two vectors: the first vector represents reachability of the given vertex $i$ from other vertices $\{u_1, u_2, ..., u_n\}$ of the graph and the second vector represents the same for the given vertex $j$. The operation $C_3[u, v] \wedge C_3[j, v]$ also can be reduced to the computation of the Hadamard product of two vectors of size $n$ for the given $u_k$. The first vector contains the information whether vertices  $\{v_1, v_2, ..., v_n\}$ of the graph are reachable from the given vertex $u_k$ and the second vector represents the same for the given vertex $j$. The element-wise product of two vectors can be calculated naively in time $O(n)$. Thus, the time complexity of the transitive closure can be reduced by speeding up element-wise product of two vectors of size $n$.


To achieve logarithmic speed-up, we use the Four Russians' trick.
First we split each vector into $n/\log n$ parts of size $\log n$.
Then we create a table $T$ such that $T(a, b)$ = $a \wedge b$ where $a, b \ \in {\{0,1\}}^{\log n}$.
This takes time $O(n^2 \log n)$, since there are $2^{\log n} = n$ variants of Boolean vectors of size $\log n$ and hence $n^2$ possible pairs of vectors $(a, b)$ in total, and each component takes $O(\log n)$ time.
With table $T$, we can calculate product of two parts of size $\log n$ in constant time.
There are $n/\log n$ such parts, so the element-wise product of two vectors of size $n$ can be calculated in time $O(n/\log n)$ with $O(n^2 \log n)$ preprocessing.

\begin{theorem}
    Let $\mathcal{G} =  \langle V,E,L\rangle$ be a graph and $G = \langle\Sigma, N, S, P\rangle$ be a grammar.
    Let $\mathcal{M}_{2}$ be a resulting adjacency matrix after the execution of the algorithm in Listing~\ref{tensor:cfpq:cubic}. Then for any valid indices $i, j$ and for each nonterminal $N_i \in N$ the following statement holds: the cell $M_{2,(k)}^{N_i}[i,j]$ contains $\{1\}$, iff there is a path $i\pi j$ in the graph $\mathcal{G}$ such that $ N_i \xrightarrow{*} l(\pi)$.
\end{theorem}{}
\begin{proof}
    The main idea of the proof is to use induction on the height of the derivation tree obtained on each iteration.
\end{proof}{}

\begin{theorem}{}
\label{theorem: subcubic}
    Let $\mathcal{G} = \langle V,E,L \rangle$ be a graph and $G = \langle\Sigma, N, S, P\rangle$ be a grammar.
    The algorithm from Listing~\ref{tensor:cfpq:cubic} calculates resulting matrices $\mathcal{M}_2$ and $C_3$ in $O({|P|}^3n^3/\log (|P|n))$ time where $n = |V|$. Moreover, the maintaining of the dynamic transitive closure dominates the cost of the algorithm.
\end{theorem}

\begin{proof}
 Let $|\mathcal{A}|$ be a number of non-zero elements in a matrix $\mathcal{A}$. Consider the total time which is needed for computing the Kronecker products. The elements of the matrices $\mathcal{A}_2^{(i)}$ are pairwise distinct on every $i$-th iteration of the algorithm therefore the total number of operations is $\sum\limits_i{T(\mathcal{M}_1 \otimes \mathcal{A}_2^{(i)})} = |\mathcal{M}_1| \sum\limits_i {|\mathcal{A}_2^{(i)}|} = (|N| + |\Sigma|){|P|}^2 \sum\limits_i {|\mathcal{A}_2^{(i)}|} = O({(|N| + |\Sigma|)}^2{|P|}^2 n^2).$


Now we derive the time complexity of maintaining the dynamic transitive closure.
Notice that $C_3$ has size of the Kronecker product of $\mathcal{M}_1 \otimes \mathcal{M}_2$, which is equal to $|R|n \times |R|n = |P|n \times |P|n$ so no more than ${|P|}^2n^2$ edges will be added during all iterations of the Algorithm.
Checking condition whether $C_3[u, i] = 1$ and $C_3[u, j]=0$ for every node $u \in V$ for each newly inserted edge $(i, j)$ requires one multiplication of vectors per insertion, thus total time is $O({|P|}^3n^3/\log (|P|n))$.
Note that after checking the condition, at least one element $C[u', j]$ changes value from 0 to 1 and then never becomes 0 for some $u'$ and $j$.
Therefore the operation $C_3[u',v] = C_3[u', v] \wedge C_3[j, v]$ for all $v \in V$ is executed at most once for every pair of vertices $(u',j)$ during the entire computation implying that the total time is equal to $O({|P|}^2n^2|P|n/\log (|P|n))=O({|P|}^3n^3/\log (|P|n))$ (using the  multiplication of vectors).


The matrix $C_3'$ contains only new elements, therefore $C_3$ can be updated directly using only $|C_3'|$ operations and hence ${|P|}^2n^2$ operations in total.
The same holds for cycle in line 17 of the algorithm from Listing~\ref{tensor:cfpq:cubic}, because operations are performed only for non-zero elements of the matrix $|C_3'|$.
Finally, the time complexity of the algorithm is $O({(|N| + |\Sigma|)}^2{|P|}^2 n^2) + O({|P|}^2n^2) + O({|P|}^2n^2 \log (|P|n)) + O({|P|}^3n^3/\log (|P|n)) + O({|P|}^2n^2)= O({|P|}^3n^3/\log (|P|n))$. \qed
\end{proof}

The complexity analysis of the Algorithm~\ref{tensor:cfpq:cubic} shows that the maintaining of the incremental transitive closure dominates the cost of the algorithm. Thus, CFPQ can be solved in truly subcubic $O(n^{3-\varepsilon})$ time if there is an incremental dynamic algorithm for the transitive closure for a graph with $n$ vertices with preprocessing time $O(n^{3-\varepsilon})$ and total update time $O(n^{3-\varepsilon})$. Unfortunately, such an algorithm is unlikely to exist: it was proven that there is no incremental dynamic transitive closure algorithm for a graph with $n$ vertices and at most $m$ edges with preprocessing time $poly(m)$, total update time $mn^{1-\varepsilon}$, and query time $m^{\delta-\varepsilon}$ for any $\delta \in (0, 1/2]$ per query that has an error probability of at most 1/3 assuming the widely believed Online Boolean Matrix-Vector Multiplication (OMv) Conjecture. OMv Conjecture introduced by~\cite{10.1145/2746539.2746609} states that for any constant $ \varepsilon>0$, there is no $O(n^{3-\varepsilon})$-time algorithm that solves OMv with an error probability of at most 1/3.

% In this section, we introduce the algorithm for context-free path querying.
%The algorithm determines the existence of a path, which forms a sentence of the language defined by
%the input RSM $R$, between each pair of vertices in the directed edge-labeled graph $\mathcal{G}$.
%The algorithm is based on the generalization of the FSM intersection for an RSM,
%and an input graph. Since a graph can be interpreted as a FSM, in which
%transitions correspond to the labeled edges between vertices of the graph,
%and an RSM is composed of a set of FSMs, the intersection of such machines
%can be computed using the classic algorithm for FSM intersection, presented
%in~\cite{automata:theory:10.5555/1177300}.
%Such a way of generalization leads to zero-overhead algorithm for RPQ, contrary to other algorithms which require regular expression to context-free grammar transformation.

%The intersection can be computed as a Kronecker product of the corresponding
%adjacency matrices for an RSM and a graph. Since we are only determining the
%reachability of vertices, it is enough to represent intersection result as
%a Boolean matrix. It simplifies the algorithm implementation and allows
%one to express it in terms of basic matrix operations.
% TODO: more accurate upper bound for the algorithm complexity

\subsubsection{Index creation for RPQ}
In case of the RPQ, the main \textbf{while} loop takes only one iteration to actually append data.
Since the input query is provided in the form of the regular expression, one can construct the corresponding RSM, which consists of the single \textit{component state machine}.
This CSM is built from the regular expression and is labeled as $S$, for example, which has no \textit{recursive calls}.
The adjacency matrix of the machine is build over $\Sigma$ only.
Therefore, calculating the Kronecker product, all relevant information is taken into account at the first iteration of the loop.
\subsection{Paths Extraction Algorithm}
After the index has been created, one can enumerate all paths between specified vertices.
The index stores information about all reachable pairs for all nonterminals.
Thus, the most natural way to use this index is to query paths between the specified vertices derivable from the specified nonterminal.
\begin{algorithm}[h]
\floatname{algorithm}{Listing}
\begin{algorithmic}[1]
\footnotesize
\caption{Paths extraction algorithm}
\label{tensor:pathsExtraction}
\State{$C_3 \gets $ result of index creation algorithm: final transitive closure}
\State{$\mathcal{M}_1 \gets $  the set of adjacency matrices of the input RSM}
\State{$\mathcal{M}_2 \gets $ the set of adjacency matrices of the final graph}
\State{$R \gets$ Recursive automata for the input RSM}

\Function{getPaths}{$v_s, v_f, N$}
\State{$q_N^0 \gets$ Start state of automata for $N$}
\State{$F_N \gets$ Final states of automata for $N$}
\State{$paths\gets \bigcup\limits_{f \in F_N} \Call{genPaths}{(q_N^0,v_s),(f,v_f)}$}
\State{$resultPaths \gets \emptyset$}
\For{$path \in paths$}
	\State{$currentPaths \gets \emptyset$}
	\For{$((s_i, v_i), (s_j, v_j)) \in path$}
	    \State{\begin{minipage}[t]{0.2\textwidth}
	    		\vspace{-13pt}
	    		\begin{align*}
	    			currentSubPaths \gets & \{(v_i,t,v_j) \mid M_2^t[s_i, s_j] \wedge M_1^t[v_i, v_j]\}\\
	    			& \cup \ \bigcup_{\{N \mid M_2^N[s_i, s_j]\}}\Call{getPaths}{v_i, v_j, N}
	    		\end{align*}
	    	\end{minipage}
	    }
		\State{$currentPaths \gets currentPaths \cdot currentSubPaths$}
		\Comment{Concatenation of paths}
	\EndFor
	\State{$resultPaths \gets resultPaths \cup currentPaths$}
\EndFor

\State \Return $resultPaths$
\EndFunction

% note: the first index in the pair is the state of the RSM
% note: the second index in the pair is the vertex of the graph

\Function{genPaths}{$(s_i,v_i), (s_j,v_j)$}
    \State{$q \gets \text{vector of zeros with size } dim(\mathcal{M}_2)$}
    \Comment{Vector for indicating the current vertex}
    \State{$q[s_i |R| + v_i] \gets 1$}
    \State{$resultPaths \gets \emptyset$}
    \State{$supposedPaths \gets \{([~], q)\}$}
    \Comment{Set of pairs: path and the vector for current vertex}
    \For{$i \in 1..dim(\mathcal{M}_2)$}
     	\For{$(path, q) \in supposedPaths$}
     		\If{$q[s_j |R| + v_j] = 1$}
     			\State{$resultPaths \gets resultPaths \cup path$}
     		\EndIf
     		
     		\State{$q \gets q \cdot (\mathcal{M}_2)^T$}
     		\Comment{Boolean vector-matrix multiplication}
     		
     		\State{$\text{Remove } (path, q)\text{ from } supposedPaths$}
     		\For{$j \text{ such that } q[j] = 1$}
     			\State{$pathNew \gets path$}
     			\If{$path \text{ is empty path}$}
     				\State{$pathNew \gets pathNew \cdot [((s_i, v_i), (j / |R|, j \bmod |R|))]$}
     				\Comment{Edge adding}
     			\Else
     				\State{$(s_k, v_k) \gets \text{ the last vertex of } path$}
     				\State{$pathNew \gets pathNew \cdot [((s_k, v_k), (j / |R|, j \bmod |R|))]$}
     				\Comment{Edge adding}
     			\EndIf
     			\State{$qNew \gets \text{vector of zeros with size } dim(\mathcal{M}_2)$}
     			\State{$qNew[j] \gets 1$}
     			\State{$\text{Add } (pathNew, qNew)\text{ to } supposedPaths$}
     		\EndFor
     	\EndFor   
    \EndFor
    \State \Return $resultPaths$
\EndFunction
\end{algorithmic}
\end{algorithm}

To do so, we provide a function \textsc{getPaths}($v_s, v_f, N$), where $v_s$ is a start vertex of the graph, $v_f$ --- the final vertex, and $N$ is a nonterminal.
Implementation of this function is presented in Listing~\ref{tensor:pathsExtraction}.

Paths extraction is implemented as three mutually recursive functions.
The entry point is \textsc{getPaths}($v_s, v_f, N$).
This function returns a set of the paths between $v_s$ and $v_f$ such that the word formed by a path is derivable from the nonterminal $N$.

To compute such paths, it is necessary to compute paths from vertices of the form $(q_N^s,v_s)$ to vertices of the form $(q_N^f, v_f)$ in the result of transitive closure, where $q_N^s$ is an initial state of RSM for $N$ and $q_N^f$ is a final state.
The function \textsc{getPathsInner}$((s_i,v_i),(s_j,v_j))$ is used to do it.
This function finds all possible vertices $(s_k,v_k)$  which split a path from $(s_i,v_i)$ to $(s_j,v_j)$ into two subpaths.
After that, function \textsc{getSubpaths}$((s_i,v_i),(s_j,v_j),(s_k,v_k))$ computes the corresponding subpaths.
Each subpath may be at least a single edge.
If single-edge subpath is labeled by terminal then corresponding edge should be added to the result else (label is nonterminal) \textsc{getPaths} should be used to restore paths.
If subpath is longer then one edge, \textsc{getPaths} should be used to restore paths. 

It is assumed that the sets are computed lazily, so as to ensure termination in the case of an infinite number of paths.
We also do not check paths for duplication manually, since they are assumed to be represented as sets.
\subsection{An example}
\label{example:section}
In this section, we introduce a detailed example to demonstrate the steps taken by the proposed algorithms.
Namely, consider the graph $\mathcal{G}$ presented in Figure~\ref{fig:example_input_graph} and the RSM $R$ presented in Figure~\ref{example:automata}.

In the first step, we represent both the graph and RSM as a set of Boolean matrices.
Notice that we should add a new empty matrix $M_2^{S}$ to $\mathcal{M}_2$,
where edges labeled by $S$ will be added at the time of the computation.

After the initialization, the algorithm handles the $\varepsilon$-case.
The input RSM does not have any $\varepsilon$-transitions and does not have any states that are both start and final, therefore, no edges are added at this stage.
After that, we should compute $\mathcal{M}_2$ and $C_3$ iteratively.
We denote the iteration number of the loop of matrices evaluation as the number in parentheses in the subscript.

\textbf{The first iteration.} First of all, we compute the Kronecker product of the
$\mathcal{M}_1$ and $\mathcal{M}_{2,(0)}$ matrices and collapse the result to the single Boolean matrix
$M_{3,(1)}$. For the sake of simplicity, we provide only
$M_{3,(1)}$, which is evaluated as follows.
{
    \renewcommand{\arraystretch}{0.5}
    \setlength\arraycolsep{0.1pt}
\begin{align*}
  \centering
& M_{3,(1)} = M_1^a \otimes M_{2,(0)}^a +  M_1^b \otimes M_{2,(0)}^b + M_1^S \otimes M_{2,(0)}^S =\\
& \kbordermatrix{
          & (0,0) & (0,1) & \vrule & (1,0) & (1,1) & \vrule &  (2,0) & (2,1) & \vrule &  (3,0) & (3,1) &\\
    (0,0) & . & .  & \vrule & . & 1  & \vrule & . & .  &  \vrule & . & .  \\
    (0,1) & . & .  & \vrule & 1 & .   & \vrule & . & .  &  \vrule & . & .  \\
    \hline
    (1,0) & . & .   & \vrule & . & .  & \vrule & . & .  & \vrule & . & . \\
    (1,1) & . & .   & \vrule & . & .  & \vrule & . & .  & \vrule & .  &1   \\
    \hline
    (2,0) & . & .   & \vrule & . & .  & \vrule & . & .  & \vrule & . & .  \\
    (2,1) & . & .   & \vrule & . & .  & \vrule & . & .  & \vrule & . &1  \\
    \hline
    (3,0) & . & .   & \vrule & . & .  & \vrule & . & .  & \vrule & . & .  \\
    (3,1) & . & .   & \vrule & . & .  & \vrule & . & .  & \vrule & . & .  \\
}
\end{align*}
}

As far as the input graph has no edges with label $S$, the correspondent block of the Kronecker product will be empty. The Kronecker product graph of the input graph $\mathcal{G}$ and RSM $R$ is shown in Figure~\ref{fig:example_1_product}. Then, the transitive closure evaluation result, stored in the matrix $C_{3,(1)}$, introduces one new path of length 2 (the thick edges in Figure~\ref{fig:example_1_product}).

\begin{figure}[h]
    \centering
   \begin{tikzpicture}[->,auto,node distance=0.5cm]
       \node[state] (q_0)                      {$(0, 0)$};
       \node[state] (q_1) [right=of q_0] {$(1, 0)$};
       \node[state] (q_2)  [right=of q_1] {$(2, 0)$};
       \node[state] (q_3) [right=of q_2] {$(3, 0)$};
       \node[state] (q_4)  [below=of q_0] {$(0, 1)$};
       \node[state] (q_5)  [right=of q_4] {$(1, 1)$};
      \node[state] (q_6)  [right=of q_5] {$(2, 1)$};
       \node[state] (q_7)  [right=of q_6] {$(3, 1)$};
       \path[->]
        (q_0) edge[very thick]  node {} (q_5)
        (q_4) edge  node {} (q_1)
        (q_5) edge [bend left, out=40, in=140, below, very thick]  node {} (q_7)
        (q_6) edge   node {} (q_7)
        ;
    \end{tikzpicture}
    \caption{The Kronecker product graph of RSM $R$ and the input graph $\mathcal{G}$ (edges which form new paths are thick)}
    \label{fig:example_1_product}
\end{figure}

This path starts in the vertex $(0,0)$ and finishes in the vertex $(3,1)$.
We can see, that 0 and 3 are the start and final states of some component
state machine for label $S$ in $R$ respectively. Thus we can conclude that
there exists a path between vertices 0 and 1 in the graph, such that the
respective word is derivable from $S$ in the $R$ execution flow.

As a result, we can add the edge $(0,S,1)$ to the resulting graph, what is done by updating the matrix $M_2^S$.

\textbf{The second iteration.} The modified graph Boolean adjacency matrices contain
an edge with label $S$. Therefore, this label contributes to the non-empty
corresponding matrix block in the evaluated matrix $M_{3,{2}}$. The transitive closure
evaluation introduces three new paths $(0, 1) \rightarrow (2,1), (1, 0) \rightarrow (3,1)$ and $(0, 1) \rightarrow (3,1)$ (see Figure~\ref{fig:example_2_product}). Since only the path between vertices $(0,1)$ and
$(3,1)$ connects the start and final states in the automaton, the edge $(1,S,1)$ is added to the resulting graph.
{
    \renewcommand{\arraystretch}{0.5}
    \setlength\arraycolsep{0.1pt}
\begin{align*}
  \centering
& M_{3,(2)} = M_1^a \otimes M_{2,(2)}^a +  M_1^b \otimes M_{2,(2)}^b + M_1^S \otimes M_{2,(2)}^S = \\
& \kbordermatrix{
          & (0,0) & (0,1) & \vrule & (1,0) & (1,1) & \vrule &  (2,0) & (2,1) & \vrule &  (3,0) & (3,1) &\\
    (0,0) & . & .  & \vrule & . & 1  & \vrule & . & .  &  \vrule & . & .  \\
    (0,1) & . & .  & \vrule & 1 & .   & \vrule & . & .  &  \vrule & . & .  \\
    \hline
    (1,0) & . & .   & \vrule & . & .  & \vrule & . & \mc  & \vrule & . & . \\
    (1,1) & . & .   & \vrule & . & .  & \vrule & . & .  & \vrule & .  &1   \\
    \hline
    (2,0) & . & .   & \vrule & . & .  & \vrule & . & .  & \vrule & . & .  \\
    (2,1) & . & .   & \vrule & . & .  & \vrule & . & .  & \vrule & . &1  \\
    \hline
    (3,0) & . & .   & \vrule & . & .  & \vrule & . & .  & \vrule & . & .  \\
    (3,1) & . & .   & \vrule & . & .  & \vrule & . & .  & \vrule & . & .  \\
}
\end{align*}
}
\begin{figure}[h]
    \centering
   \begin{tikzpicture}[->,auto,node distance=0.5cm]
       \node[state] (q_0)                      {$(0, 0)$};
       \node[state] (q_1) [right=of q_0] {$(1, 0)$};
       \node[state] (q_2)  [right=of q_1] {$(2, 0)$};
       \node[state] (q_3) [right=of q_2] {$(3, 0)$};
       \node[state] (q_4)  [below=of q_0] {$(0, 1)$};
       \node[state] (q_5)  [right=of q_4] {$(1, 1)$};
      \node[state] (q_6)  [right=of q_5] {$(2, 1)$};
       \node[state] (q_7)  [right=of q_6] {$(3, 1)$};
       \path[->]
        (q_1) edge[very thick] node {} (q_6)
        (q_0) edge node {} (q_5)
        (q_4) edge[very thick]  node {} (q_1)
        (q_5) edge [bend left, in=140, out=40, below]  node {} (q_7)
        (q_6) edge[very thick]   node {} (q_7)
        ;
    \end{tikzpicture}
    \caption{The Kronecker product graph of RSM $R$ and the updated graph $\mathcal{G}$ (edges which form new paths are thick)}
    \label{fig:example_2_product}
\end{figure}
The result graph is presented in Figure~\ref{fig:example_result}.
\begin{figure}[h]
    \centering
    \begin{tikzpicture}[shorten >=1pt,auto]
       \node[state] (q_0)                      {$0$};
       \node[state] (q_1) [right=of q_0]       {$1$};
        \path[->]
        (q_0) edge[bend left, above]   node {a} (q_1)
         (q_0) edge[in=190, out=-10, red, very thick]   node {S} (q_1)
        (q_1) edge [bend left, below] node {a} (q_0)
         (q_1) edge[loop above, red, thick] node {S} (q_1)
        (q_1) edge[loop right] node {b} (q_1);

    \end{tikzpicture}
    \caption{The result graph $\mathcal{G}$}
    \label{fig:example_result}
\end{figure}


\begin{figure}[h]
	\centering
	\begin{tikzpicture}[->,auto,node distance=0.5cm]
		\node[state] (q_0)                      {$(0, 0)$};
		\node[state] (q_1) [right=of q_0] {$(1, 0)$};
		\node[state] (q_2)  [right=of q_1] {$(2, 0)$};
		\node[state] (q_3) [right=of q_2] {$(3, 0)$};
		\node[state] (q_4)  [below=of q_0] {$(0, 1)$};
		\node[state] (q_5)  [right=of q_4] {$(1, 1)$};
		\node[state] (q_6)  [right=of q_5] {$(2, 1)$};
		\node[state] (q_7)  [right=of q_6] {$(3, 1)$};
		\path[->]
		(q_1) edge node {} (q_6)
		(q_0) edge[very thick] node {} (q_5)
		(q_4) edge  node {} (q_1)
		(q_5) edge [bend left, in=140, out=40, below]  node {} (q_7)
		(q_5) edge[very thick]   node {} (q_6)
		(q_6) edge[very thick]   node {} (q_7)
		;
	\end{tikzpicture}
	\caption{The Kronecker product graph of RSM $R$ and the final graph $\mathcal{G}$ (edges which form new paths are thick)}
	\label{fig:example_3_product}
\end{figure}


No more edges will be added to the graph $\mathcal{G}$ at the last iteration.
However, the new edge $(1, 1) \rightarrow (2,1)$ will be added to the resulting Kronecker product, and the transitive closure
evaluation introduces one new path $(0, 0) \rightarrow (3,1)$ that connects the start and final states in the automaton (see Figure~\ref{fig:example_3_product}).
At this point, the index creation is finished.
One can use it to answer reachability queries, but it also can be used
to restore paths for some reachable vertices. The resulting Kronecker product matrix
$M_3$, or so-called \textit{index}, can be used for it.
For example, we can restore paths from vertex 1 to vertex 1 derived from $S$ in the resulting graph.

To get these paths we should call \verb|getPaths(1, 1, S)| function.
A partial trace of this call is presented in Figure~\ref{trc:example}.
First, we query paths for all possible start and final states of the
machine for the provided graph vertices.
Since the component state machine with label $S$ in the example RSM has the single final state, the function \verb|genPaths| is called with the arguments $(0,1)$ and $(3,1)$.
Note, that the values passed to the functions in the path extraction algorithm are the
pairs of the machine state and graph vertex, which uniquely identify a cell of
the index matrix $M_3$. As a result,
we get the set of all possible paths in the graph from $1$ to $1$ derived from $S$.

\begin{figure}
\begin{minipage}[t]{0.48\textwidth}
{
\scriptsize
\setlength{\DTbaselineskip}{8pt}
\DTsetlength{0.2em}{0.5em}{0.2em}{0.4pt}{1.6pt}
\dirtree{%
.1 getPaths($1,1,S$).
.2 genPaths($(0,1),(3,1)$).
.3 return $\{[((0,1),(1,0)), ((1,0),(2,1)), ((2,1),(3,1))]\}$.
.2 currentSubPaths = $\{[1 \xrightarrow{a} 0]\}$.
.2 currentPaths = $\{[(1 \xrightarrow{a} 0)]\}$.
.2 getPaths($0,1,S$).
.3 genPaths($(0,0),(3,1)$).
.4 return $\{[((0,0),(1,1)), ((1,1),(3,1))], [((0,0),(1,1)), ((1,1),(2,1)), ((2,1),(3,1))]\}$.
.3 path = $[((0,0),(1,1)), ((1,1),(3,1))]$.
.3 currentSubPaths = $\{[0 \xrightarrow{a} 1]\}$.
.3 currentPaths = $\{[0 \xrightarrow{a} 1]\}$.
.3 currentSubPaths = $\{[1 \xrightarrow{b} 1]\}$.
.3 $\cdots$.
.3 resultPaths = $\{[0 \xrightarrow{a} 1 \xrightarrow{b} 1]\}$.
.3 path = $[((0,0),(1,1)), ((1,1),(2,1)), ((2,1),(3,1))]$.
.3 currentSubPaths = $\{[0 \xrightarrow{a} 1]\}$.
.3 currentPaths = $\{[0 \xrightarrow{a} 1]\}$.
.3 getPaths(1, 1, S) // \begin{minipage}[t]{14cm} An alternative way to get paths from 1 to 1 (leads to infinite set of paths) \end{minipage}.
.3 $\cdots$.
.8 return $r_\infty^{1 \leadsto 1}$ // \begin{minipage}[t]{5cm} An infinite set of path from 1 to 1 \end{minipage}.
.3 currentPaths = $\{[0 \xrightarrow{a} 1] \} \cdot r_\infty^{1\leadsto 1}$.
.3 currentSubPaths = $\{[1 \xrightarrow{b} 1]\}$.
.3 $\cdots$.
.3 return $\{[0 \xrightarrow{a} 1 \xrightarrow{b} 1]\} \cup (\{[0 \xrightarrow{a} 1] \} \cdot r_\infty^{1\leadsto 1} \cdot \{[1 \xrightarrow{b} 1]\})$.
.2 currentPaths = $\{[1 \xrightarrow{a} 0 \xrightarrow{a} 1 \xrightarrow{b} 1]\} \cup (\{[1 \xrightarrow{a} 0 \xrightarrow{a} 1] \} \cdot r_\infty^{1\leadsto 1} \cdot \{[1 \xrightarrow{b} 1]\})$.
.2 currentSubPaths = $\{[1 \xrightarrow{b} 1]\}$.
.2 $\cdots$.
.2 return = $\{[1 \xrightarrow{a} 0 \xrightarrow{a} 1 \xrightarrow{b} 1 \xrightarrow{b} 1]\} \cup (\{[1 \xrightarrow{a} 0 \xrightarrow{a} 1] \} \cdot r_\infty^{1\leadsto 1} \cdot \{[1 \xrightarrow{b} 1 \xrightarrow{b} 1]\})$.
}
}
\caption{Example of call stack trace}
\label{trc:example}
\end{minipage}
\end{figure}