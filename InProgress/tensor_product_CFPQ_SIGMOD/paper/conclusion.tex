\section{Conclusion and Future Work}

In this work, we present an improved version of the tensor-based algorithm for CFPQ: we reduce the algorithm to operations over Boolean matrices, and we provide the ability to extract all paths which satisfy the query.
Moreover, the provided algorithm can handle grammars in EBNF, thus it does not require CNF transformation of the grammar and avoids grammar size blow-up.
As a result, the algorithm demonstrates practical performance not only on CFPQ queries but also on RPQ ones, which is shown by our evaluation.
Thus, we provide a universal linear algebra based algorithm for RPQ and CFPQ evaluation with all-paths semantics.
Moreover, our algorithm opens a way to tackle the long-standing problem of determining whether a subcubic CFPQ exists.
This is done by reducing the algorithm to incremental transitive closure: incremental transitive closure with $O(n^{3-\varepsilon})$ total update time for $n^2$ updates, such that each update returns all of the new reachable pairs, implies $O(n^{3-\varepsilon})$ CFPQ algorithm.
We prove $O({|P|}^3n^3/\log (|P|n))$ time complexity by providing $O(n^3/\log{n})$ incremental transitive closure algorithm.


Recent hardness results for dynamic graph problems demonstrates that any further improvement for incremental transitive closure (and, hence, CFPQ) will imply a major breakthrough for other long-standing dynamic graph problems. An algorithm for incremental dynamic transitive closure with total update time $O(mn^{1-\varepsilon})$ ($n$ denotes the number of graph vertices, $m$ is the number of graph edges) even with polynomial $poly(n)$ time preprocessing of the input graph and $m^{\delta - \varepsilon}$ query time per query for any $\delta \in (0, 1/2]$ will refute the Online Boolean Matrix-Vector Multiplication (OMv) Conjecture, which is used to prove conditional lower bounds for many dynamic problems ~\citep{8948597, 10.1145/2746539.2746609}.


Thus, the first task for the future is to improve the logarithmic factor in the obtained bound.
We also plan to improve bounds in partial cases for which dynamic transitive closure can be computed faster than in the general case.
Examples of such cases are querying over planar graphs~\citep{10.1007/3-540-57273-2_72}, undirected graph, and others.
%Is it possible to use \cho{these facts} to provide a better CFPQ algorithm for respective partial cases?
In the case of planarity, it is interesting to investigate properties of the input graph and grammar which allow us to preserve planarity during query evaluation.

Note that both our algorithm and the state-of-the-art solutions have subcubic time complexity in terms of the grammar size \citep{Azimov:2018:CPQ:3210259.3210264, hellingsRelational, hellingsPathQuerying, 10.1145/258994.259006, 10.1145/3398682.3399163}.
However, our approach does not require the input grammar to be translated to the Chomsky Normal Form and thus avoids the quadratic blow-up in its size which leads to a better performance in practice.


An important task for future research is a detailed investigation of the paths extraction algorithm.~\cite{HellSinglePath} provides a theoretical investigation of the single-path extraction and shows that the problem is related to the formal language theory.
Extraction of all paths is more complicated and should be investigated carefully in order to provide an optimal algorithm.

%On the other hand, the reduction presented in the paper opens a way to investigate streaming graph querying.
%This way we can formulate the following questions. Is it possible to provide more detailed analysis of dynamic CFPQ queries than in~\cite{10.1007/978-3-662-54458-7_16}? Is it possible to create a practical solution for CFPQ querying of streaming graphs? Is it possible to improve the existing solutions for RPQ of streaming graphs?

From a practical perspective, it is necessary to analyze the usability of advanced algorithms for dynamic transitive closure.
In the current work, we evaluate the naive implementation in which transitive closure is recalculated from scratch on each iteration.
It is shown by~\cite{cs6345} that some advanced algorithms for dynamic transitive closure can be efficiently implemented.
Can one of these algorithms be efficiently parallelized and utilized in the proposed algorithm?


Also, it is necessary to evaluate GPGPU-based implementation.
Evaluation of Azimov's algorithm shows that it is possible to improve performance by using GPGPU because operations of linear algebra can be efficiently implemented on GPGPU~\citep{Mishin:2019:ECP:3327964.3328503,10.1145/3398682.3399163}.
Moreover, for practical reasons, it is interesting to provide a multi-GPU version of the algorithm and to utilize unified memory, which is suitable for linear algebra based processing of out-of-GPGPU-memory data and traversing on large graphs~\citep{8946118,10.14778/3384345.3384358}.

In order to simplify the distributed processing of huge graphs, it may be necessary to investigate different formats for sparse matrices, such as HiCOO format~\citep{10.5555/3291656.3291682}.
Another interesting question in this direction is about the utilization of virtualization techniques: should we implement a distributed version of the algorithm manually or it can be better to use CPU and RAM virtualization to get a virtual machine with a huge amount of RAM and a big number of computational cores.
The experience of the Trinity project team shows that it can make sense~\citep{10.1145/2463676.2467799}.

Finally, it is necessary to provide a multiple-source version of the algorithm and integrate it with a graph database.
RedisGraph\footnote{RedisGraph is a graph database that is based on the Property Graph
Model. Project web page: \url{https://oss.redislabs.com/redisgraph/}. Access date:
07.07.2020.}~\citep{8778293} is a suitable candidate for this purpose.
This database uses SuiteSparse---an implementation of GraphBLAS standard---as a base for graph processing.
This fact allowed to~\cite{10.1145/3398682.3399163} to integrate Azimov's algorithm to RedisGraph with minimal effort.
