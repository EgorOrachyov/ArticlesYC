
{\actuality} 
В современном мире становится всё больше данных, которые требуют анализа. При этом графы являются одной из самых распространённых структур данных. С их помощью большие объемы информации компактно и удобно представляются для анализа и обработки. Графы используются в программной инженерии и биоинформатике, в телекоммуникациях, социальных сетях и сетевом анализе и т.д. Также, в настоящее время активно развиваются графовые базы данных, используемые для хранения и анализа больших объёмов данных в виде графов. Следует упомянуть самые популярные графовые базы данных: RedisGraph, Neo4j, HyperGraphDB и OrientDB. При этом важной является задача поиска различных путей в графах. Чтобы описать свойства искомых путей, необходимо задать определенные ограничения на них. Данные ограничения формулируются в виде запроса к графу, а ответом на запрос является информация о существовании соответствующих путей, удовлетворяющих данным ограничениям. Говорят, что такие запросы вычеслены в relational семантике. Кроме того, в некоторых областях, в качестве доказательства существования таких путей необходимо предъявить все или хотя бы один из них, тогда говорят, что такие запросы вычислены в all-path и single-path семантиках соответственно.

Для описания ограничений на пути в помеченном графе естественно использовать формальные грамматики над некоторым алфавитом. Заданная грамматика ограничивает множество слов, полученных конкатенацией меток на рёбрах рассматриваемых путей. В настоящее время активно исследуются ограничения в виде контекстно-свободных (КС) грамматик, так как они позволяют описывать широкий класс запросов (КС-запросов), более широкий, чем, например, регулярные выражения. Однако большинство существующих алгоритмов вычисления КС-запросов имеют низкую производительность на больших графах, что затрудняет их применение на практике.

Одним из распространённых способов улучшения производительности алгоритмов на графах является их переформулирование в терминах линейной алгебры. Для тех алгоритмов, которые позволяют найти такую формулировку, имеется возможность применить разреженное представление матриц и параллельные вычисления, в частности, на GPU. Так, например, Лейуань Ван (Leyuan Wang) сформулировал алгоритм подсчёта количества треугольников в графе на языке линейной алгебры и реализовал полученный алгоритм на GPU. Полученная реализация показала лучшую производительность по сравнению с реализациями на CPU. Кроме того, такого рода алгоритмы зачастую просты в реализации, так как позволяют использовать существующие библиотеки линейной алгебры (GraphBLAS::SuiteSparse, cuSPARSE, cuBLAS, m4ri, Scipy и др.). Таким образом, перспективным является направление формулирования алгоритмов вычисления КС-запросов к графам на языке линейной алгебры.

Однако возможность использования линейной алгебры в задачах поиска путей с КС-ограничениями в графах в настоящее время не исследована. Таким образом, актуальной задачей является разработка алгоритмов вычисления КС-запросов к графам, использующих различные операции линейной алгебры, и исследование их свойств.

%Кроме того, в данной области КС-грамматика приводится в нормальную форму, что способствует её разрастанию и увеличивает время анализа. Таким образом, перспективным направлением является исследование возможности использования такой матричной операции, как произведение Кронекера. Использование данной операции позволит работать с КС-грамматиками произвольного вида без необходимости их преобразования. Таким образом, алгоритм поиска путей с КС-ограничениями в графах, использующий произведение Кронекера может позволить получить еще больший прирост производительности за счёт отсутсвия разрастания размеров грамматики.


%Обзор, введение в тему, обозначение места данной работы в
%мировых исследованиях и~т.\:п., можно использовать ссылки на~другие
%работы~\autocite{Gosele1999161}
%(если их~нет, то~в~автореферате
%автоматически пропадёт раздел <<Список литературы>>). Внимание! Ссылки
%на~другие работы в~разделе общей характеристики работы можно
%использовать только при использовании \verb!biblatex! (из-за технических
%ограничений \verb!bibtex8!. Это связано с тем, что одна
%и~та~же~характеристика используются и~в~тексте диссертации, и в
%автореферате. В~последнем, согласно ГОСТ, должен присутствовать список
%работ автора по~теме диссертации, а~\verb!bibtex8! не~умеет выводить в~одном
%файле два списка литературы).
%При использовании \verb!biblatex! возможно использование исключительно
%в~автореферате подстрочных ссылок
%для других работ командой \verb!\autocite!, а~также цитирование
%собственных работ командой \verb!\cite!. Для этого в~файле
%\verb!common/setup.tex! необходимо присвоить положительное значение
%счётчику \verb!\setcounter{usefootcite}{1}!.


{\progress}
Множество работ посвящены переформулированию классических алгоритмов на графах в терминах операций линейной алгебры. Например, Айдын Булук (Aydin Bulu\c{c}), 
Упасана Шридхар (Upasana Sridhar), Питер Чжан (Peter Zhang) и Арифул Азад (Ariful Azad) в своих работах показывают как можно свести к линейной алгебре такие алгортимы, как поиск в ширину, алгоритм Дейкстры, алгоритм Беллмана-Форда и поиск наибольшего паросочетания в двудольном графе. Кроме того, существует стандарт GraphBLAS, который определяет базовые строительные блоки на языке линейной алгебры для алгоритмов на графах. GraphBLAS основан на том, что графы могут быть представлены в виде матрицы смежности или матрицы инцидентности. Также, ввиду того, что данные на практике разрежены, целесообразно использовать разреженный формат матриц. Однако, не каждый алгоритм на графах можно переформулировать на языке линейной алгебры. Так, например, до сих пор не смогли это сделать для алгоритма поиска в глубину. Также, в настоящее время, такая формулировка не найдена и для алгоритмов вычисления КС-запросов на графах.

Лесли Вэлиант (Leslie Valiant) провёл исследование, посвященное синтаксическому анализу КС-языков с использованием матричных операций. Однако предложенный им алгоритм позволяет проводить анализ только над строками, что эквивалентно анализу лишь специальных, линейно помеченных, графов. 
Кроме того, Филип Брэдфорд (Philip Bradford) исследовал задачу поиска кратчайших путей в графе с КС-ограничениями. Но предложенные им алгоритмы хоть и сформулированы на языке линейной алгебры, но предназначены только для частного случая КС-грамматик и/или специализированных графов. Таким образом, данные алгоритмы не применимы к поиску путей с произвольными КС-ограничениями в произвольных графах.

Задаче поиска путей с КС-ограничениями в произвольных графах посвящены работы таких учёных, как Семён Григорьев, Джелле Хеллингс (Jelle Hellings), Сяованг Чжан (Xiaowang Zhang) и Мартин Мюзиканте (Martin Musicante). В данных исследованиях используются подходы, основанные на различных алгоритмах синтаксического анализа (LR, LL, GLL, CYK). Однако, данные алгоритмы не представленны в терминах линейной алгебры. Впервые вопрос о возможности нахождения матричного алгоритма вычисления КС-запросов к графам исследовал Михалис Яннакакис (Mihalis Yannakakis). Он указывал, что алгоритм Вэлианта может быть расширен для обработки графов без циклов, однако сомневался в возможности расширения алгоритма Вэлианта до произвольных графов.

Таким образом, на текущий момент не существует алгоритма вычисления произвольных КС-запросов к произвольным графам, выраженного на языке линейной алгебры. Поэтому необходимо дальнейшее исследование возможности разработки таких алгоритмов.
%Этот раздел должен быть отдельным структурным элементом по
% ГОСТ, но он, как правило, включается в описание актуальности
% темы. Нужен он отдельным структурынм элемементом или нет ---
% смотрите другие диссертации вашего совета, скорее всего не нужен.

{\aim} данной работы является исследование применимости линейной алгебры в задаче вычисления КС-запросов к графам.

{\aim} данной работы является исследование применимости линейной алгебры в задаче вычисления КС-запросов к графам и получение реализаций данных алгоритмов с улучшенной производительностью.

{\aim} данной работы является исследование применимости линейной алгебры в задаче вычисления КС-запросов к графам и получение более высокопроизводительных реализаций для данных алгоритмов с использованием параллельных вычислительных систем.

Достижение поставленной цели обеспечивается решением следующих {\tasks}:
\begin{enumerate}[beginpenalty=10000] % https://tex.stackexchange.com/a/476052/104425
  \item Разработать подход к вычислению КС-запросов к графам в терминах линейной алгебры.
  \item Разработать семейство алгоритмов, использующих предложенный подход и вычисляющих КС-запросы с relational, single-path и all-path семантиками.
  \item Разработать семейство алгоритмов вычисления КС-запросов к графам, использующих предложенный подход, вычисляющих КС-запросы с relational, single-path и all-path семантиками, и не требующих преобразований входной КС-грамматики.
  \item Реализовать предложенные алгоритмы на CPU и GPU, провести их экспериментальное исследование на синтетических и реальных данных, сравнить их между собой и с существующими реализациями.
\end{enumerate}


{\novelty}
\begin{enumerate}[beginpenalty=10000] % https://tex.stackexchange.com/a/476052/104425
	
	\item Алгоритмы вычисления КС-запросов к графам, которые используют подход, предложенный в диссертации, отличаются от алгоритмов, предложенных в работах Семёна Григорьева, Сяованга Чжана, Джелле Хеллингса, Филиппа Брэдфорда и Мартина Мюзиканте тем, что первые сформулированы в терминах линейной алгебры и вычисляют произвольные КС-запросы к произвольным графам. С практической точки зрения, предложенный подход, позволяет применять широкий класс оптимизаций для вычисления матричных операций и дает возможность автоматически распараллеливать вычисления за счёт существующих решений.
	
	\item Впервые получены алгоритмы вычисления произвольных КС-запросов к произвольным графам, сформулированные в терминах линейной алгебры.
	
	\item Алгоритмы, предложенный в диссертации, отличается от алгоритмов, предложенных в работах Семёна Григорьева, Сяованга Чжана, Джелле Хеллингса, Филиппа Брэдфорда и Мартина Мюзиканте возможностью работать с произвольными КС-грамматиками без необходимости их преобразования. Таким образом, удаётся избежать значительного увеличения размеров входной грамматики, от которого напрямую зависит временная сложность данных алгоритмов.
	
	\item Экспериментальное исследование алгоритмов вычисления КС-запросов с различной семантикой к произвольным графам, использующих операции линейной алгебры проводится впервые и позволяет судить о применимости на практике разработанных алгоритмов.
	
\end{enumerate}

{\influence} 
Теоретическая значимость диссертационного исследования заключается в разработке подхода к вычислению КС-запросов к графам, использующего операции линейной алгебры, в разработке формальных алгоритмов, использующих данный подход, а также в формальном доказательстве завершаемости, корректности и оценок временной сложности разработанных алгоритмов.

Полученные в ходе исследования реализации позволили оценить применимость на практике алгоритмов вычисления КС-запросов, сформулированных в терминах линейной алгебры. Кроме того, данные реализации могут быть использованы для интеграции с такими графовыми базами данных, как RedisGraph. Это добавит возможность вычислять КС-запросы к этим базам данных.

{\methods} Методология исследования основана на линейной алгебре и теории графов. Подход, предлагающий использовать операции линейной алгебры при анализе графов, начал активно развиваться со второй половины 20-го века. В 2013 году был создан стандарта GraphBLAS, определяющий для алгоритмов на графах базовые строительные блоки на языке линейной алгебры. На текущий момент множество классических алгоритмов из теории графов были сформулированы в терминах линейной алгебры.

Кроме того, в исследовании использовалась теория формальных языков. Одной из задач, для которых до сих пор не найдена формулировка в терминах линейной алгебры, является поиск путей в графе с ограничениями в виде КС-грамматик. Данная задача использует подход к анализу строк, который начал активно развиваться в 50-х годах 20-го века в связи с изучением естественных языков (работы Н.~Хомского). В последствии этот подход получил широкое распространение в различных областях, в том числе и связанных с анализом графов.
%При этом основными элементами данного подхода являются алфавит и грамматика исследуемого языка, выступающего в качестве ограничения на искомые пути в графе. Решаемые в связи с этим задачи связаны с поиском эффективных алгоритмов нахождения путей, удовлетворяющих заданным ограничениям.

Доказательство завершаемости, корректности и оценок временной сложности предложенных алгоритмов проводится с применением линейной алгебры, теории формальных языков, теории графов и теории сложности алгоритмов.

{\defpositions}
\begin{enumerate}[beginpenalty=10000] % https://tex.stackexchange.com/a/476052/104425
	\item Разработан подход к вычислению КС-запросов к графам в терминах линейной алгебры.
	\item Разработано семейство алгоритмов, использующих предложенный подход и вычисляющих КС-запросы с relational, single-path и all-path семантиками. Доказана завершаемость и корректность предложенных алгоритмов. Получена теоритическая оценка сверху временной сложности алгоритмов.
	\item Разработано семейство алгоритмов, использующих предложенный подход и вычисляющих КС-запросы с relational, single-path и all-path семантиками, а также не требующих преобразований входной КС-грамматики. Доказана завершаемость и корректность предложенных алгоритмов. Получена теоритическая оценка сверху временной сложности алгоритмов.
	\item Проведено экспериментальное исследование разработанных алгоритмов. Предложенные алгоритмы реализованы на CPU и GPU. Дальше детали.
\end{enumerate}

{\reliability} 
Достоверность и обоснованность результатов исследования опирается на использование формальных методов исследуемой области, выполнение формальных доказательств и инженерные эксперименты.

Основные результаты работы были представлены на ряде международных научных конференций: GRADES'18, GRADES'20, ADBIS’20, SIGMOD'21(еще не приняли). А также разработка предложенных алгоритмов была поддержана грантом РНФ \textnumero 18-11-00100 и грантом РФФИ \textnumero 19-37-90101.

%{\contribution} Автор принимал активное участие \ldots

{\publications} Все результаты диссертации изложены в 7~\cite{1,2,3,4,5,6,7} научных работах. Из них 2 работы~\cite{6,7} индексируются ВАК и 6 работ~\cite{1,2,3,4,5,6} индексируются Scopus. Работы~\cite{1,2,3,4,6,7} написаны в соавторстве. В~\cite{1,6,7} Р.~Азимову принадлежит разработка алгоритма, доказательство его корректности и завершаемости, реализация алгоритма, работа над текстом. В~\cite{2} Р.~Азимову принадлежит разработка алгоритма, доказательство его корректности и завершаемости, работа над текстом. В~\cite{3,4} Р.~Азимову принадлежит работа над доказательствами корректности и завершаемости алгоритма, работа над текстом.

\newcounter{firstbib}

\begin{thebibliography}{99}
	\bibitem{1} Azimov R. Context-Free Path Querying by
	Matrix Multiplication / Azimov R., Grigorev S. // In Proceedings of the
	1st Joint International Workshop on Graph Data Management Experiences \&
	Systems (GRADES) and Network Data Analytics (NDA) (GRADES-NDA’18)
	\bibitem{2} Azimov R. Context-Free Path Querying with Single-Path Semantics by
	Matrix Multiplication / Terekhov A., Khoroshev A., Azimov R., Grigorev S. // In Proceedings of the
	3rd Joint International Workshop on Graph Data Management Experiences \&
	Systems (GRADES) and Network Data Analytics (NDA) (GRADES-NDA’20)
	\bibitem{3} Azimov R. Context-Free Path Querying by Kronecker
	Product / Orachev E., Epelbaum I., Azimov R., Grigorev S. // In Proceedings of the
	24th European Conference on Advances in Databases and Information Systems (ADBIS’20)
	\bibitem{4} Azimov R. Context-Free Path Querying by Kronecker
	Product большая версия / Orachev E., Epelbaum I., Azimov R., Grigorev S. // In Proceedings of the (SIGMOD’21)
	
	\bibitem{5} Azimov R. Ненаписанная работа матричный алгоритм по всем путям
	
	\bibitem{6} Azimov R. Path Querying with Conjunctive Grammars by Matrix Multiplication / Azimov R., Grigorev S. //Programming and Computer Software. – 2019. – Vol. 45. – №. 7. – pp. 357-364.
	\setcounter{firstbib}{\value{enumiv}}
	
	\bibitem{7} Азимов Р. Ш. Синтаксический анализ графов с использованием конъюнктивных грамматик / Азимов Р., Григорьев С. // Труды ИСП РАН, 2018, том 1 вып. 2, с. 3-4.
	
\end{thebibliography}