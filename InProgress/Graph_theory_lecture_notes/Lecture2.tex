\section{Спектральная теория графов. Введение}

Матрица смежности.

Матрица Кирхгофа. Оператор Лапласа. Для неориентированного графа без кратных рёбер и петель.
\begin{definition}
Пусть неориентированный граф без кратных рёбер и петель (простой граф) $G=\langle V, E \rangle, |V| = n$. Тогда матрица Кирхгофа $ K=(k_{i,j})_{n \times n}$. 
$$ k_{i,j}:={\begin{cases}\deg(v_{i})&{\text{при}}\ i=j,\\-1&{\text{при}}\ (v_{i},v_{j})\in E(G),\\0&{\text{в противном случае}}.\end{cases}}
$$
\end{definition}

$K = D - A$, где $A$ --- матрица смежности графа, а $D$ --- матрица, на диагонали которой строят степени вершин, а остальные элементы равны нулю.


\begin{example}[Пример графа и его матрицы Кирхгофа]
Пусть дан граф:
  \begin{center}
  \begin{tikzpicture}[on grid, auto]
     \node[state] (q_0)   {$0$};
     \node[state] (q_1) [above right=1.4cm and 1.0cm of q_0] {$1$};
     \node[state] (q_2) [right=2.0cm of q_0] {$2$};
     \node[state] (q_3) [right=2.0cm of q_2] {$3$};
      \path[-]
      (q_0) edge (q_1)
      (q_1) edge (q_2)
      (q_2) edge (q_0)
      (q_2) edge (q_3);
  \end{tikzpicture}
  \end{center}
  $$ D =
  \left({
  \begin{array}{rrrrrr}
  2 & 0 & 0 & 0 \\
  0 & 2 & 0 & 0 \\
  0 & 0 & 3 & 0 \\
  0 & 0 & 0 & 1 \\
  \end{array}
  }\right)
$$
$$ A =
  \left({
  \begin{array}{rrrrrr}
  0 & 1 & 1 & 0 \\
  1 & 0 & 1 & 0 \\
  1 & 1 & 0 & 1 \\
  0 & 0 & 1 & 0 \\
  \end{array}
  }\right)
$$

$$ K = D - A =
  \left({
  \begin{array}{rrrrrr}
  2  & -1 & -1 & 0  \\
  -1 & 2  & -1 & 0  \\
  -1 & -1 & 3  & -1 \\
  0  & 0  & -1 & 1  \\
  \end{array}
  }\right)
$$
\end{example}

Определитель матрицы.


\begin{definition}[Дополнительный минор]
$M_{i,j}$ --- дополнительный минор, определитель матрицы, получающейся из исходной матрицы $A$ путём вычёркивания $i$-й строки и $j$-го столбца.
\end{definition}


\begin{definition}[Определитель матрицы $2 \times 2$]
Для матрицы $2\times 2$ определитель вычисляется как:

$$\Delta ={\begin{vmatrix}a&c\\b&d\end{vmatrix}} = ad - bc$$

\end{definition}

\begin{definition}[Определитель матрицы $N \times N$]

$$\Delta =\sum _{j=0}^{n-1}(-1)^{j}a_{0,j}{M}_{0,j}$$
, где $M_{0,j}$ --- дополнительный минор к элементу $a_{0,j}$. 


\end{definition}

Это было разложение по строке и, вообще говоря, подобная операция может быть проделяна для любой строки.
Аналогично можно использовать разложение по столбцу.


\begin{definition}[Определитель матрицы $3 \times 3$]
\begin{align*}
&\Delta =
{\begin{vmatrix}
  a_{0,0}&a_{0,1}&a_{0,2}\\
  a_{1,0}&a_{1,1}&a_{1,2}\\
  a_{2,0}&a_{2,1}&a_{2,2}
 \end{vmatrix}}
 =
  a_{0,1}{\begin{vmatrix}a_{1,1}&a_{1,2}\\a_{2,1}&a_{2,2}\end{vmatrix}}
 -a_{0,2}{\begin{vmatrix}a_{1,0}&a_{1,2}\\a_{2,0}&a_{2,2}\end{vmatrix}}
 +a_{0,3}{\begin{vmatrix}a_{1,0}&a_{1,1}\\a_{2,0}&a_{2,1}\end{vmatrix}} 
 =\\
&a_{0,0}a_{1,1}a_{2,2}-a_{0,0}a_{1,2}a_{2,1}-a_{0,1}a_{1,0}a_{2,2}+a_{0,1}a_{1,2}a_{2,0}+a_{0,2}a_{1,0}a_{2,1}-a_{0,2}a_{1,1}a_{2,0}
\end{align*}
\end{definition}


\begin{definition}[Алгебраическое дополнение]
Алгебраическим дополнением элемента $ a_{i,j}$ матрицы $A$ называется число $A_{i,j}=(-1)^{i+j}M_{i,j}$,
где $M_{i,j}$ --- дополнительный минор.
\end{definition}

\begin{example}[Определитель]
Найдём определитель матрицы Кирхгофа для нашего графа. Будем использовать разложение по 3-й строке.
\begin{align*}
&\Delta\left({
  \begin{array}{rrrrrr}
  2  & -1 & -1 & 0  \\
  -1 & 2  & -1 & 0  \\
  -1 & -1 & 3  & -1 \\
  0  & 0  & -1 & 1  \\
  \end{array}
  }\right) = (-1)^5 (-1) {\begin{vmatrix}2&-1&0\\-1&2&0\\-1&-1&-1\end{vmatrix}} + (-1)^6 1 {\begin{vmatrix}2&-1&-1\\-1&2&-1\\-1&-1&3\end{vmatrix}}=\\
  &1((2\cdot2\cdot-1) - (2\cdot0\cdot-1) - (-1\cdot-1\cdot-1) + (-1\cdot0\cdot-1) + (0\cdot-1\cdot-1) - (0\cdot2\cdot-1)) +\\
  & 1((2\cdot2\cdot3) - (2\cdot-1\cdot-1) - (-1\cdot-1\cdot3) + (-1\cdot-1\cdot-1) + (-1\cdot-1\cdot-1) - (-1\cdot2\cdot-1) ) =\\
  & (-4 + 1) + (12 - 2 - 3 - 1 - 1 - 2) = -3 + 3 = 0 
\end{align*}
\end{example}


\begin{theorem}[Матричная теорема об остовных деревьях]
Пусть $G$ --- связный простой граф с матрицей Кирхгофа $K$. Все алгебраические дополнения матрицы Кирхгофа $K$ равны между собой и их общее значение равно количеству остовных деревьев графа $G$.
\end{theorem}

\begin{example}[Количество остовных деревьев]
Из примера выше, значения миноров равно 3.

Деревья:
\begin{center}
  \begin{tikzpicture}[on grid, auto]
     \node[state] (q_0)   {$0$};
     \node[state] (q_1) [above right=1.4cm and 1.0cm of q_0] {$1$};
     \node[state] (q_2) [right=2.0cm of q_0] {$2$};
     \node[state] (q_3) [right=2.0cm of q_2] {$3$};
      \path[-]
      (q_0) edge (q_1)
      (q_2) edge (q_0)
      (q_2) edge (q_3);
  \end{tikzpicture}
  \end{center}

  \begin{center}
  \begin{tikzpicture}[on grid, auto]
     \node[state] (q_0)   {$0$};
     \node[state] (q_1) [above right=1.4cm and 1.0cm of q_0] {$1$};
     \node[state] (q_2) [right=2.0cm of q_0] {$2$};
     \node[state] (q_3) [right=2.0cm of q_2] {$3$};
      \path[-]
      (q_1) edge (q_2)
      (q_2) edge (q_0)
      (q_2) edge (q_3);
  \end{tikzpicture}
  \end{center}

  \begin{center}
  \begin{tikzpicture}[on grid, auto]
     \node[state] (q_0)   {$0$};
     \node[state] (q_1) [above right=1.4cm and 1.0cm of q_0] {$1$};
     \node[state] (q_2) [right=2.0cm of q_0] {$2$};
     \node[state] (q_3) [right=2.0cm of q_2] {$3$};
      \path[-]
      (q_0) edge (q_1)
      (q_1) edge (q_2)
      (q_2) edge (q_3);
  \end{tikzpicture}
  \end{center}
\end{example}


Сумма элементов каждой строки (столбца) матрицы Кирхгофа равна нулю: $\sum _{i=0}^{|V|-1}k_{i,j}=0$.

Определитель матрицы Кирхгофа равен нулю: $\Delta K = 0$.

Собственные числа и собственные вектора.

Нам понядобится поле $F$. 

\begin{definition}[Собственный вектор]
Ненулевой вектор $x$ называется собственным вектором матрицы $A$ для некоторого элемента $\lambda \in F$, если $Ax = \lambda x$
\end{definition}

\begin{definition}[Собственное число (Собственное значение)]
Собственным числом матрицы $A$ называется такое $\lambda \in F$, что существует ненулевое решение уравнения $Ax = \lambda x$.
\end{definition}

Как видно, собственные числа и собственные вектора ``ходят парами''.

\begin{example}[Собственные числа и вектора]

$$
A = \left({
  \begin{array}{rrrrrr}
  2  & -1 & -1 & 0  \\
  -1 & 2  & -1 & 0  \\
  -1 & -1 & 3  & -1 \\
  0  & 0  & -1 & 1  \\
  \end{array}
  }\right)
$$

По определению $Ax = \lambda x$. 
$$Ax - \lambda x = 0$$
$$(A - \lambda E) x = 0$$
$$
(\left({
  \begin{array}{rrrrrr}
  2  & -1 & -1 & 0  \\
  -1 & 2  & -1 & 0  \\
  -1 & -1 & 3  & -1 \\
  0  & 0  & -1 & 1  \\
  \end{array}
  }\right) -  \left({
  \begin{array}{rrrrrr}
  \lambda  & 0 & 0 & 0  \\
  0 & \lambda  & 0 & 0  \\
  0 & 0 & \lambda  & 0 \\
  0  & 0  & 0 & \lambda  \\
  \end{array}
  }\right)) x = 0
$$

Данное уравнение имеет ненулевое решение тогда и только тогда, когда $|A - \lambda E| = 0$
\begin{align*}
|A - \lambda E| =
  \begin{vmatrix}
  2 - \lambda & -1 & -1 & 0  \\
  -1 & 2 - \lambda & -1 & 0  \\
  -1 & -1 & 3 - \lambda & -1 \\
  0  & 0  & -1 & 1 - \lambda \\
  \end{vmatrix} = \lambda^4-8\lambda^3+19\lambda^2-12\lambda
\end{align*}
То есть надо решить уравнение 
$$
\lambda^4-8\lambda^3+19\lambda^2-12\lambda = 0
$$

$$
\lambda^4-8\lambda^3+19\lambda^2-12\lambda  = \lambda (\lambda^3 - 8\lambda^2 + 19\lambda - 12) = \lambda(\lambda - 1)(\lambda^2 - 7\lambda + 12) = \lambda(\lambda - 1)(\lambda - 3)(\lambda - 4) = 0
$$

Корни: $\lambda \in \{0,1,3,4\}$.

Далее для каждого совственного числа нужно найти соответсвующий вектор. Для этого решаем системы линейных уравнений (метод Гаусса в помощь).
$$(A - 0\cdot E) x = 0$$
$$(A - 1\cdot E) x = 0$$
$$(A - 3\cdot E) x = 0$$
$$(A - 4\cdot E) x = 0$$

$$x_0 = \left(\begin{array}{r}1\\1\\1\\1\end{array}\right)$$
$$x_1 = \left(\begin{array}{r}-\frac{1}{2}\\-\frac{1}{2}\\0\\1\end{array}\right)$$
$$x_2 = \left(\begin{array}{r}-1\\1\\0\\0\end{array}\right)$$
$$x_3 = \left(\begin{array}{r}1\\1\\-3\\1\end{array}\right)$$


\end{example}

\begin{definition}[Спектр графа]
Спектром графа называется упорядоченное по возростанию мультимножество собственных значений его матрицы смежности.
\end{definition}

Так как мы говорим о неориентированном графе, то собственные значения всегд вещественные числа (почему?).

Хотя матрица смежности и зависит от нумерации вешин, спектр является инвариантом графа (почему?).

Следствие: изоморфные графы имеют одинаковый спектр.

Графы с одинаковым спектром --- изоспектральные (или коспектральные).

\begin{theorem}
Изоморфные графы всегда изоспектральны. Обратное не верно (изоспектральные графы не обязательно изоморфны).
\end{theorem}

\subsection{Задачи}
\begin{enumerate}
\item Доказать формулу Кэли, пользуясь матричной теоремой об остовных деревьях. Формула Кэли даёт оценку числа остовных деревьев полного графа $K_{n}$: $n^{n-2}$.
\item Доказать, пользуясь матричной теоремой об остовных деревьях, что число остовных деревьев полного двудольнoгo графа $K_{m,n}$ равно $m^{n-1}\cdot n^{m-1}$.
\item Доказать, что спектр является инвариантом графа.
\end{enumerate}